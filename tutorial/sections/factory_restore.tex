% =============================================================================
\section{Factory Restore}
% =============================================================================
\label{sec:factory_restore}

The Avnet guide \emph{Restoring MiniZed to the Factory State}~\cite{Avnet_MiniZed_Restore_2018}
describes how to restore the MiniZed to factory state. This section
shows how to view the version information of U-Boot and Linux on a MiniZed,
how to calculate the checksums of the images in on-board non-volatile memory,
and how to compare those checksums to the recovery or user images.

% -----------------------------------------------------------------------------
\subsection{Viewing Tool and Software Versions}
% -----------------------------------------------------------------------------
\label{sec:restore_view_versions}

To view the MiniZed software versions, start by setting the boot mode switch
to Flash, connecting the board to your development machine using the USB
JTAG/UART interface (see Figure~\ref{fig:minized_quick_start_diagram}), and
configure a serial terminal for communication at 115200 baud 8N1.
Hit enter and the MiniZed console output (for a device purchased in March 2019) was
%
\begin{verbatim}
PetaLinux 2017.4 MiniZed /dev/ttyPS0

MiniZed login:
\end{verbatim}
%
Press the blue reset button (by the PMod connectors on the bottom right
of Figure~\ref{fig:minized_quick_start_diagram}), press the enter key to
stop U-Boot from loading Linux, and enter the \verb+version+ command to
get details on the U-Boot version. The console output was
%
\begin{verbatim}
U-Boot 2017.01 (Mar 22 2018 - 23:33:56 -0700)

Board: Xilinx Zynq
DRAM:  ECC disabled 512 MiB
MMC:   Card did not respond to voltage select!
sdhci@e0100000 - probe failed: -95
sdhci_transfer_data: Error detected in status(0x208000)!
Card did not respond to voltage select!

SF: Detected n25q128 with page size 256 Bytes, erase size 64 KiB, total 16 MiB
*** Warning - bad CRC, using default environment

In:    serial
Out:   serial
Err:   serial
U-BOOT for MiniZed

Hit any key to stop autoboot:  0
Zynq>  version

U-Boot 2017.01 (Mar 22 2018 - 23:33:56 -0700)
arm-xilinx-linux-gnueabi-gcc (Linaro GCC 6.2-2016.11) 6.2.1 20161016
GNU ld (GNU Binutils) 2.27.0.20160806
Zynq>
\end{verbatim}
%
Note that the PetaLinux prompt and U-Boot messages have different version strings.
The PetaLinux prompt implies that Xilinx Vivado and PetaLinux tools version 2017.4
were used to build PetaLinux, while the U-Boot version string implies
2017.1 tools were used to build U-Boot.

Type \verb+?+ and then press the enter key to see the U-Boot help menu. U-Boot can be
used to read and write memory locations, read and program flash, boot standalone
programs, real-time operating systems, and Linux. The example designs in this tutorial
demonstrate some of these U-Boot features.

\newpage
Boot Linux using the \verb+boot+ command. The inital lines of the console output were
%
\begin{verbatim}
Zynq> boot
boot Petalinux
reading image.ub
43267036 bytes read in 14650 ms (2.8 MiB/s)
## Loading kernel from FIT Image at 10000000 ...
   Using 'conf@1' configuration
   Trying 'kernel@0' kernel subimage
     Description:  Linux Kernel
     Type:         Kernel Image
     Compression:  uncompressed
     Data Start:   0x100000d4
     Data Size:    3925272 Bytes = 3.7 MiB
     Architecture: ARM
     OS:           Linux
     Load Address: 0x00008000
     Entry Point:  0x00008000
     Hash algo:    sha1
     Hash value:   f2ac69bcdcf755b54af38b5fd870f843fb0e9bc7
   Verifying Hash Integrity ... sha1+ OK
## Loading ramdisk from FIT Image at 10000000 ...
   Using 'conf@1' configuration
   Trying 'ramdisk@0' ramdisk subimage
     Description:  ramdisk
     Type:         RAMDisk Image
     Compression:  uncompressed
     Data Start:   0x103c28d8
     Data Size:    39323409 Bytes = 37.5 MiB
     Architecture: ARM
     OS:           Linux
     Load Address: unavailable
     Entry Point:  unavailable
     Hash algo:    sha1
     Hash value:   d57b450521fd302afdc95bfe2e7c521b08745d10
   Verifying Hash Integrity ... sha1+ OK
## Loading fdt from FIT Image at 10000000 ...
   Using 'conf@1' configuration
   Trying 'fdt@0' fdt subimage
     Description:  Flattened Device Tree blob
     Type:         Flat Device Tree
     Compression:  uncompressed
     Data Start:   0x103be6e0
     Data Size:    16712 Bytes = 16.3 KiB
     Architecture: ARM
     Hash algo:    sha1
     Hash value:   bd55217bdcb35db08dcfd467cfbbf0dd5e4cb754
\end{verbatim}
%
The hash values for the kernel, RAM disk, and device tree can be compared
to the versions in the restored image to confirm they are the same.
Save a copy of the Linux boot messages, as they maybe useful later when
building PetaLinux from scratch.

\newpage
Login at the Linux prompt as user \verb+root+ with password \verb+root+
and type a few standard Linux commands to determine some detail about
the Linux and hardware configuration;
%
\begin{verbatim}
root@MiniZed:~# uname -a
Linux MiniZed 4.9.0-xilinx-v2017.4 #1 SMP PREEMPT Thu Mar 22 22:22:16 MST 2018
armv7l GNU/Linux

root@MiniZed:~# mount
rootfs on / type rootfs (rw,size=227296k,nr_inodes=56824)
proc on /proc type proc (rw,relatime)
sysfs on /sys type sysfs (rw,relatime)
devtmpfs on /dev type devtmpfs (rw,relatime,size=227296k,nr_inodes=56824,mode=755)
tmpfs on /run type tmpfs (rw,nosuid,nodev,mode=755)
tmpfs on /var/volatile type tmpfs (rw,relatime)
/dev/mmcblk1p1 on /run/media/mmcblk1p1 type vfat (rw,relatime,gid=6,fmask=0007,
dmask=0007,allow_utime=0020,codepage=437,iocharset=iso8859-1,shortname=mixed,
errors=remount-ro)
devpts on /dev/pts type devpts (rw,relatime,gid=5,mode=620,ptmxmode=000)
/dev/mmcblk1p1 on /mnt/emmc type vfat (rw,relatime,gid=6,fmask=0007,dmask=0007,
allow_utime=0020,codepage=437,iocharset=iso8859-1,shortname=mixed,errors=remount-ro)

root@MiniZed:~# df
Filesystem           1K-blocks      Used Available Use% Mounted on
devtmpfs                227296         4    227292   0% /dev
tmpfs                   255204        84    255120   0% /run
tmpfs                   255204        44    255160   0% /var/volatile
/dev/mmcblk1p1          123089     58385     64704  47% /run/media/mmcblk1p1
/dev/mmcblk1p1          123089     58385     64704  47% /mnt/emmc

root@MiniZed:~# cat /proc/cpuinfo
processor       : 0
model name      : ARMv7 Processor rev 0 (v7l)
BogoMIPS        : 666.66
Features        : half thumb fastmult vfp edsp neon vfpv3 tls vfpd32
CPU implementer : 0x41
CPU architecture: 7
CPU variant     : 0x3
CPU part        : 0xc09
CPU revision    : 0

Hardware        : Xilinx Zynq Platform
Revision        : 0003
Serial          : 0000000000000000

root@MiniZed:~# cat /proc/version
Linux version 4.9.0-xilinx-v2017.4 (sroussea@xterra1) (gcc version 6.2.1 20161016
(Linaro GCC 6.2-2016.11) ) #1 SMP PREEMPT Thu Mar 22 22:22:16 MST 2018

root@MiniZed:~# poweroff
\end{verbatim}
%
Per the recommendation on page 30 of the \emph{Getting Started Guide}~\cite{Avnet_MiniZed_GSG_2018},
use the \verb+poweroff+ command before powering off the MiniZed to avoid corruption of the eMMC flash.

\newpage
The QSPI flash also contains a fallback Linux image. This image can be used when the eMMC flash has
been erased. The fallback image can be booted from U-Boot as follows:
%
\begin{verbatim}
Zynq> run boot_qspi
Booting backup kernel from QSPI..
SF: Detected n25q128 with page size 256 Bytes, erase size 64 KiB, total 16 MiB
device 0 offset 0x270000, size 0xd80000
SF: 14155776 bytes @ 0x270000 Read: OK
## Loading kernel from FIT Image at 10000000 ...
   Using 'conf@1' configuration
   Trying 'kernel@0' kernel subimage
     Description:  Linux Kernel
     Type:         Kernel Image
     Compression:  uncompressed
     Data Start:   0x100000d4
     Data Size:    3925320 Bytes = 3.7 MiB
     Architecture: ARM
     OS:           Linux
     Load Address: 0x00008000
     Entry Point:  0x00008000
     Hash algo:    sha1
     Hash value:   8d758a5b6f7da43eecf5324a02ff10a78e139113
   Verifying Hash Integrity ... sha1+ OK
## Loading ramdisk from FIT Image at 10000000 ...
   Using 'conf@1' configuration
   Trying 'ramdisk@0' ramdisk subimage
     Description:  ramdisk
     Type:         RAMDisk Image
     Compression:  uncompressed
     Data Start:   0x103c2908
     Data Size:    10018537 Bytes = 9.6 MiB
     Architecture: ARM
     OS:           Linux
     Load Address: unavailable
     Entry Point:  unavailable
     Hash algo:    sha1
     Hash value:   2f1d8fc5a9ce9dfbdf1ea753a23dcf7f0d881d55
   Verifying Hash Integrity ... sha1+ OK
## Loading fdt from FIT Image at 10000000 ...
   Using 'conf@1' configuration
   Trying 'fdt@0' fdt subimage
     Description:  Flattened Device Tree blob
     Type:         Flat Device Tree
     Compression:  uncompressed
     Data Start:   0x103be710
     Data Size:    16712 Bytes = 16.3 KiB
     Architecture: ARM
     Hash algo:    sha1
     Hash value:   bd55217bdcb35db08dcfd467cfbbf0dd5e4cb754
\end{verbatim}
%
The hash values of the kernel, RAM disk, and device tree are different than
the versions loaded from the eMMC.

\newpage
% -----------------------------------------------------------------------------
\subsection{Restore Image Details}
% -----------------------------------------------------------------------------
\label{sec:restore_image_details}

The MiniZed contains three non-volatile memory devices;
%
\begin{itemize}
\item 2kbit EEPROM
\item 128Mbit QSPI Flash
\item 8GB eMMC Flash
\end{itemize}
%
The 2kbit EEPROM contains the configuration for the FTDI FT2232H device.
The EEPROM data can be viewed using the
\href{https://www.ftdichip.com/Support/Utilities.htm#FT_PROG}{FT\_PROG}
tool from FTDI. All MiniZed boards ship with the same
USB JTAG/UART serial number: \verb+1234-oj1+. Under Windows, COM port
management is a lot simpler if each USB device has a unique serial number.
%
The FTDI FT\_PROG tool \emph{cannot} be used to change the MiniZed 
FTDI FT2232H serial number, as the tool does not preserve 
Xilinx/Avnet/Digilent vendor-specific bytes in the EEPROM.
%
Appendix~\ref{sec:ftdi_eeprom} contains details on the EEPROM format,
and the application \texttt{minized\_ftdi\_serial} (in the repository)
can be used to read and write the EEPROM contents.

The Avnet MiniZed documentation page has a link to a zip file containing
the restore images~\cite{Avnet_MiniZed_Restore_2018}. The zip file contains
the following files;
%
\begin{verbatim}
Restoring MiniZed to Factory Status_03
+-- Programming files
|   +-- Micron QSPI Flash
|       +-- flash_fallback_7007S.bin
|       +-- flash_only_boot_7007S.bin
|       +-- zynq_fsbl.elf
+-- Restoring_MiniZed_to_Factory_Status_03.pdf
+-- USB memory stick files
    +-- image.ub
    +-- smallboot.bin
    +-- wpa_supplicant.conf
\end{verbatim}
%
The MiniZed factory QSPI image is \verb+flash_fallback_7007S.bin+,
while the eMMC filesystem contains the files in \verb+USB memory stick files+.

The images programmed in the MiniZed board QSPI flash and eMMC could be
compared to the factory images using byte-by-byte comparisons, however,
there is a simpler (and faster) solution: calculate and compare checksums.
The MiniZed U-Boot image supports the \verb+crc32+ checksum (the U-Boot
source provides support for additional checksum algorithms), and the MiniZed Linux image
supports \verb+sha1sum+, \verb+sha256sum+, and \verb+sha512sum+.
Table~\ref{tab:factory_image_checksums} contains the \verb+crc32+
and \verb+sha1sum+ checksums for the restore images. Note that since
the checksums for  \verb+flash_fallback_7007S.bin+ and \verb+smallboot.bin+
are the same, the two files are identical.


% -----------------------------------------------------------------------------
% Factory Image Checksums
% -----------------------------------------------------------------------------
%
\begin{landscape}
\begin{table}
\caption{MiniZed Factory Restore Image Checksums.}
\label{tab:factory_image_checksums}
\begin{center}
\begin{tabular}{|l|c|c|c|}
\hline
      & Image Byte & \multicolumn{2}{c|}{Checksum}\\
\cline{3-4}
Image & Length & \verb+crc32+ & \verb+sha1+\\
\hline
\multicolumn{4}{|l|}{\bf U-Boot}\\
\hline
&&&\\
\verb+flash_fallback_7007S.bin+  & \verb+0x00FC0BE4+ & \verb+0x005B7B3C+ & \verb+0x71511941C14604A88BDD57B0C7762444D1538B5B+\\
\verb+flash_only_boot_7007S.bin+ & \verb+0x00FDA5F4+ & \verb+0x86771B07+ & \verb+0x191BE50A37CF5E7D453A67AE510264C1F2CDF89B+\\
\verb+smallboot.bin+             & \verb+0x00FC0BE4+ & \verb+0x005B7B3C+ & \verb+0x71511941C14604A88BDD57B0C7762444D1538B5B+\\
&&&\\
\hline
\multicolumn{4}{|l|}{\bf Linux}\\
\hline
&&&\\
\verb+image.ub+ & \verb+0x00FC0BE4+ & \verb+0x005B7B3C+ & \verb+0xD618217182184862C69AB1559EF806665B7D0E83+\\
&&&\\
\hline
\end{tabular}
\end{center}
\end{table}
\end{landscape}

% -----------------------------------------------------------------------------
\subsubsection{QSPI Image Details}
% -----------------------------------------------------------------------------

The image in MiniZed QSPI flash can be viewed by copying the
contents of the 16MB flash to DDR memory, and then reading the
image from DDR memory. The U-Boot commands to detect the QSPI flash
and copy 16MB (0x1000000) to DDR address 0x10000000 are
%
\begin{verbatim}
Zynq> sf probe
SF: Detected n25q128 with page size 256 Bytes, erase size 64 KiB, total 16 MiB
Zynq> sf read 10000000 0 1000000
device 0 whole chip
SF: 16777216 bytes @ 0x0 Read: OK
\end{verbatim}
%
The QSPI image header can be displayed by reading the DDR memory
%
\begin{verbatim}
Zynq> md 10000000 30
10000000: eafffffe eafffffe eafffffe eafffffe    ................
10000010: eafffffe eafffffe eafffffe eafffffe    ................
10000020: aa995566 584c4e58 00000000 01010000    fU..XNLX........
10000030: 00001700 00018008 00000000 00000000    ................
10000040: 00018008 00000001 fc164530 00000000    ........0E......
10000050: 00000000 00000000 00000000 00000000    ................
10000060: 00000000 00000000 00000000 00000000    ................
10000070: 00000000 00000000 00000000 00000000    ................
10000080: 00000000 00000000 00000000 00000000    ................
10000090: 00000000 00000000 000008c0 00000c80    ................
100000a0: ffffffff 00000000 ffffffff 00000000    ................
100000b0: ffffffff 00000000 ffffffff 00000000    ................
\end{verbatim}
%
The BootROM Image Header format is defined in Table 6-5 of the Zynq
TRM (p171~\cite{Xilinx_UG585_2018}).

The U-Boot crc32 checksum for the QSPI image copied to DDR
memory is
%
\begin{verbatim}
Zynq> crc32 10000000 fc0be4
crc32 for 10000000 ... 10fc0be3 ==> 005b7b3c
\end{verbatim}
%
The checksum matches that for
\verb+flash_fallback_7007S.bin+
in Table~\ref{tab:factory_image_checksums}.

\clearpage
% -----------------------------------------------------------------------------
\subsubsection{eMMC Image Details}
% -----------------------------------------------------------------------------

The eMMC images can be viewed from the MiniZed Linux console using:
%
\begin{verbatim}
root@MiniZed:~# ls -al /mnt/emmc/
total 58385
drwxrwx---    2 root     disk           512 Jan  1  1970 .
drwxr-xr-x    4 root     root            80 Mar 23 06:07 ..
-rwxrwx---    1 root     disk      43267036 Jul 28  2017 image.ub
-rwxrwx---    1 root     disk      16518116 Jul 28  2017 smallboot.bin
-rwxrwx---    1 root     disk           177 Jul 28  2017 wpa_supplicant.conf

root@MiniZed:~# od -Ax -tx4 -v -N 128 /mnt/emmc/image.ub
000000 edfe0dd0 dc339402 38000000 f4319402
000010 28000000 11000000 10000000 00000000
000020 74000000 bc319402 00000000 00000000
000030 00000000 00000000 01000000 00000000
000040 03000000 04000000 64000000 3c9ab45a
000050 03000000 24000000 00000000 6f422d55
000060 6620746f 6d497469 20656761 20726f66
000070 786e6c70 6d72615f 72656b20 006c656e

root@MiniZed:~# od -Ax -tx4 -v -N 192 /mnt/emmc/smallboot.bin
000000 eafffffe eafffffe eafffffe eafffffe
000010 eafffffe eafffffe eafffffe eafffffe
000020 aa995566 584c4e58 00000000 01010000
000030 00001700 00018008 00000000 00000000
000040 00018008 00000001 fc164530 00000000
000050 00000000 00000000 00000000 00000000
000060 00000000 00000000 00000000 00000000
000070 00000000 00000000 00000000 00000000
000080 00000000 00000000 00000000 00000000
000090 00000000 00000000 000008c0 00000c80
0000a0 ffffffff 00000000 ffffffff 00000000
0000b0 ffffffff 00000000 ffffffff 00000000
\end{verbatim}
%
The image checksums can be calculated via
%
\begin{verbatim}
root@MiniZed:~# sha1sum /mnt/emmc/image.ub
d618217182184862c69ab1559ef806665b7d0e83  /mnt/emmc/image.ub

root@MiniZed:~# sha1sum /mnt/emmc/smallboot.bin
71511941c14604a88bdd57b0c7762444d1538b5b  /mnt/emmc/smallboot.bin
\end{verbatim}
%
The checksums match those in Table~\ref{tab:factory_image_checksums}.
The \verb+wpa_supplicant.conf+ file on the MiniZed was different than
the restore image, however, the user has to edit that file to match
their setup, so that difference was expected.

\clearpage
% -----------------------------------------------------------------------------
\subsubsection{U-Boot Image Versions and Checksums}
% -----------------------------------------------------------------------------
\label{sec:uboot_versions}

This section contains the console output for U-Boot power-on-reset,
\verb+version+, and \verb+crc32+ commands for the three Avnet restore
images.

\subsubsection*{Image: \textcolor{blue}{\texttt{flash\_fallback\_7007S.bin}}}
%
\begin{verbatim}
U-Boot 2017.01 (Mar 22 2018 - 23:33:56 -0700)

Board: Xilinx Zynq
DRAM:  ECC disabled 512 MiB
MMC:   Card did not respond to voltage select!
sdhci@e0100000 - probe failed: -95
sdhci_transfer_data: Error detected in status(0x208000)!
Card did not respond to voltage select!

SF: Detected n25q128 with page size 256 Bytes, erase size 64 KiB, total 16 MiB
*** Warning - bad CRC, using default environment

In:    serial
Out:   serial
Err:   serial
U-BOOT for MiniZed

Hit any key to stop autoboot:  0
Zynq> version

U-Boot 2017.01 (Mar 22 2018 - 23:33:56 -0700)
arm-xilinx-linux-gnueabi-gcc (Linaro GCC 6.2-2016.11) 6.2.1 20161016
GNU ld (GNU Binutils) 2.27.0.20160806

Zynq> sf probe
SF: Detected n25q128 with page size 256 Bytes, erase size 64 KiB, total 16 MiB
Zynq> sf read 10000000 0 1000000
device 0 whole chip
SF: 16777216 bytes @ 0x0 Read: OK
Zynq> crc32 10000000 fc0be4
crc32 for 10000000 ... 10fc0be3 ==> 005b7b3c
\end{verbatim}

\subsubsection*{Image: \textcolor{blue}{\texttt{flash\_only\_boot\_7007S.bin}}}
%
\begin{verbatim}
U-Boot 2016.07 (Jul 27 2017 - 21:56:59 -0700)

DRAM:  ECC disabled 512 MiB
MMC:   sdhci@e0100000: 0, sdhci@e0101000: 1
SF: Detected N25Q128 with page size 256 Bytes, erase size 64 KiB, total 16 MiB
*** Warning - bad CRC, using default environment

In:    serial
Out:   serial
Err:   serial
U-BOOT for

Hit any key to stop autoboot:  0
Zynq> version

U-Boot 2016.07 (Jul 27 2017 - 21:56:59 -0700)
arm-linux-gnueabihf-gcc (Linaro GCC 5.2-2015.11-2) 5.2.1 20151005
GNU ld (Linaro_Binutils-) 2.25.0 Linaro 2016_02
Zynq> sf probe
SF: Detected N25Q128 with page size 256 Bytes, erase size 64 KiB, total 16 MiB
Zynq> sf read 10000000 0 1000000
device 0 whole chip
SF: 16777216 bytes @ 0x0 Read: OK
Zynq> crc32 10000000 fda5f4
crc32 for 10000000 ... 10fda5f3 ==> 86771b07
\end{verbatim}

\subsubsection*{Image: \textcolor{blue}{\texttt{smallboot.bin}}}
%
\begin{verbatim}
U-Boot 2017.01 (Mar 22 2018 - 23:33:56 -0700)

Board: Xilinx Zynq
DRAM:  ECC disabled 512 MiB
MMC:   Card did not respond to voltage select!
sdhci@e0100000 - probe failed: -95
sdhci_transfer_data: Error detected in status(0x208000)!
Card did not respond to voltage select!

SF: Detected n25q128 with page size 256 Bytes, erase size 64 KiB, total 16 MiB
*** Warning - bad CRC, using default environment

In:    serial
Out:   serial
Err:   serial
U-BOOT for MiniZed

Hit any key to stop autoboot:  0
Zynq> version

U-Boot 2017.01 (Mar 22 2018 - 23:33:56 -0700)
arm-xilinx-linux-gnueabi-gcc (Linaro GCC 6.2-2016.11) 6.2.1 20161016
GNU ld (GNU Binutils) 2.27.0.20160806

Zynq> sf probe
SF: Detected n25q128 with page size 256 Bytes, erase size 64 KiB, total 16 MiB
Zynq> sf read 10000000 0 1000000
device 0 whole chip
SF: 16777216 bytes @ 0x0 Read: OK
Zynq> crc32 10000000 fc0be4
crc32 for 10000000 ... 10fc0be3 ==> 005b7b3c
\end{verbatim}
%
The console messages from \textcolor{blue}{\texttt{flash\_fallback\_7007S.bin}}
and \textcolor{blue}{\texttt{smallboot.bin}} are identical, since they the same 
file. The easiest way to determine which image is programmed, is
to calculate the \verb+crc32+ checksum using the image length and
expected values in Table~\ref{tab:factory_image_checksums}.

\clearpage
% -----------------------------------------------------------------------------
\subsection{QSPI Flash Programming}
% -----------------------------------------------------------------------------
\label{sec:qspi_flash_programming}

The Avnet MiniZed factory restore document~\cite{Avnet_MiniZed_Restore_2018}
shows how to program the QSPI flash using the Xilinx Software Command-line
Tool (XSCT) and the Vivado GUI. The restore images were copied to the
location \verb+c:\temp\minized+ prior to the operations in the next section.

% -----------------------------------------------------------------------------
\subsubsection{XSCT \texttt{program\_flash}}
% -----------------------------------------------------------------------------

XSCT can be started under Windows by selecting the tool from
\emph{Xilinx Design Tools}$\rightarrow$\emph{SDK 2019.1}$\rightarrow$\emph{Xilinx
Software Command Line Tool}. Once the tool starts, change directory to
the folder containing the restore images and issue the \verb+program_flash+
command using the Tcl exec command, eg.,
%
\begin{verbatim}
xsct% cd {c:\temp\minized}
xsct% exec program_flash -f flash_only_boot_7007S.bin -fsbl zynq_fsbl.elf
  -flash_type qspi_single
\end{verbatim}
%
Running the \verb+program_flash+ command using the Tcl \verb+exec+ command
supresses the command console output until the flash programming completes.
More insight into the flash programming is gained if the tool is
run directly, so that the console output is seen while the flash is
programmed.

The location of the \verb+program_flash+ tool can be determined using
%
\begin{verbatim}
xsct% exec which program_flash
C:\software\Xilinx\SDK\2019.1\bin\program_flash
\end{verbatim}
%
The MiniZed flash was erased using the U-Boot command
%
\begin{verbatim}
Zynq> sf probe
SF: Detected N25Q128 with page size 256 Bytes, erase size 64 KiB, total 16 MiB
Zynq> sf erase 0 1000000
SF: 16777216 bytes @ 0x0 Erased: OK
\end{verbatim}
%
and the Flash/JTAG switch changed to the JTAG position. The flash programming
tool was then run from a Cygwin console using the full path to the
application via
%
\begin{verbatim}
$ cd c:/temp/minized
$ C:/software/Xilinx/SDK/2019.1/bin/program_flash -f flash_only_boot_7007S.bin
  -fsbl zynq_fsbl.elf -flash_type qspi_single

****** Xilinx Program Flash
****** Program Flash v2019.1 (64-bit)
  **** SW Build 2552052 on Fri May 24 14:49:42 MDT 2019
    ** Copyright 1986-2019 Xilinx, Inc. All Rights Reserved.

Connected to hw_server @ TCP:localhost:3121
Available targets and devices:
Target 0 : jsn-MiniZed V1-1234-oj1A
        Device 0: jsn-MiniZed V1-1234-oj1A-4ba00477-0

Retrieving Flash info...

Initialization done, programming the memory
===== mrd->addr=0xF800025C, data=0x00000000 =====
BOOT_MODE REG = 0x00000000
Downloading FSBL...
Running FSBL...
Finished running FSBL.
===== mrd->addr=0xF8000110, data=0x000FA220 =====
READ: ARM_PLL_CFG (0xF8000110) = 0x000FA220
===== mrd->addr=0xF8000100, data=0x00028008 =====
READ: ARM_PLL_CTRL (0xF8000100) = 0x00028008
===== mrd->addr=0xF8000120, data=0x1F000200 =====
READ: ARM_CLK_CTRL (0xF8000120) = 0x1F000200
===== mrd->addr=0xF8000118, data=0x00113220 =====
READ: IO_PLL_CFG (0xF8000118) = 0x00113220
===== mrd->addr=0xF8000108, data=0x00024008 =====
READ: IO_PLL_CTRL (0xF8000108) = 0x00024008
Info:  Remapping 256KB of on-chip-memory RAM memory to 0xFFFC0000.
===== mrd->addr=0xF8000008, data=0x00000000 =====
===== mwr->addr=0xF8000008, data=0x0000DF0D =====
MASKWRITE: addr=0xF8000008, mask=0x0000FFFF, newData=0x0000DF0D
===== mwr->addr=0xF8000910, data=0x000001FF =====
===== mrd->addr=0xF8000004, data=0x00000000 =====
===== mwr->addr=0xF8000004, data=0x0000767B =====
MASKWRITE: addr=0xF8000004, mask=0x0000FFFF, newData=0x0000767B

U-Boot 2019.01-07026-gae88108-dirty (Mar 22 2019 - 04:38:02 -0600)

Model: Zynq CSE QSPI Board
DRAM:  256 KiB
WARNING: Caches not enabled
In:    dcc
Out:   dcc
Err:   dcc
Zynq> sf probe 0 0 0
Warning: SPI speed fallback to 100 kHz
SF: Detected n25q128 with page size 256 Bytes, erase size 64 KiB, total 16 MiB
Zynq> Sector size = 65536.
f probe 0 0 0
Performing Erase Operation...
sf erase 0 FE0000
SF: 16646144 bytes @ 0x0 Erased: OK
Zynq> Erase Operation successful.
INFO: [Xicom 50-44] Elapsed time = 1 sec.
Performing Program Operation...
0%...sf write FFFC0000 0 20000
device 0 offset 0x0, size 0x20000
SF: 131072 bytes @ 0x0 Written: OK
Zynq> sf write FFFC0000 20000 20000
device 0 offset 0x20000, size 0x20000
SF: 131072 bytes @ 0x20000 Written: OK

... [more sector programming messages] ...

Zynq> sf write FFFC0000 FA0000 20000
device 0 offset 0xfa0000, size 0x20000
SF: 131072 bytes @ 0xfa0000 Written: OK
Zynq> 100%
sf write FFFC0000 FC0000 1A5F4
device 0 offset 0xfc0000, size 0x1a5f4
SF: 108020 bytes @ 0xfc0000 Written: OK
Zynq> Program Operation successful.
INFO: [Xicom 50-44] Elapsed time = 67 sec.

Flash Operation Successful
\end{verbatim}
%
The output from \verb+program_flash+ shows that it downloads the FSBL,
a U-Boot image, and then uses U-Boot commands to program the flash.
Interesting! The U-Boot flash programming occurs much faster than
when using the UART, as the U-Boot image used by
\verb+program_flash+ uses the ARM JTAG Debug Communications Channel
(DCC) to transfer sectors to U-Boot for programming to flash:
the initial U-Boot console message shows the use of \verb+dcc+ for
stdin, stdout and stderr.

% -----------------------------------------------------------------------------
\subsubsection{Vivado Flash Programming}
% -----------------------------------------------------------------------------

The Avnet MiniZed factory restore document~\cite{Avnet_MiniZed_Restore_2018}
shows how to use the \emph{Vivado Hardware Manager} to program the QSPI flash.
When I first followed the procedure using the flash images located
in the folder generated by unzipping the restore images zip file,
Vivado would pop-up an error dialog with the message
\verb+[Labtools 27-3161] Flash Programming Unsuccessful+. The Vivado
console did not provide any insight into what might be wrong.
%
Flash programming works if the restore files are first copied to
\verb+c:/temp/minized+.

The Vivado Tcl console output for programming of the image \verb+flash_fallback_7007S.bin+
was
%
\begin{verbatim}
program_hw_cfgmem -hw_cfgmem [ get_property PROGRAM.HW_CFGMEM
  [lindex [get_hw_devices xc7z007s_1] 0]]

Performing Erase Operation...
Erase Operation successful.
INFO: [Xicom 50-44] Elapsed time = 28 sec.

Performing Program Operation...
Program Operation successful.
INFO: [Xicom 50-44] Elapsed time = 68 sec.

Performing Verify Operation...
INFO: [Xicom 50-44] Elapsed time = 91 sec.
Verify Operation successful.

INFO: [Labtoolstcl 44-377] Flash programming completed successfully
\end{verbatim}
%
After programming, the Flash/JTAG boot mode switch was switched to Flash
mode, U-Boot booted, and the procedure in Section~\ref{sec:uboot_versions}
used to confirm the U-Boot version and checksum.

\clearpage
% -----------------------------------------------------------------------------
\subsection{Restore Procedure}
% -----------------------------------------------------------------------------
\label{sec:restore_procedure}

The Avnet MiniZed factory restore procedure is~\cite{Avnet_MiniZed_Restore_2018}:
%
\begin{enumerate}
\item Power-on the MiniZed and program the QSPI flash with the flash-only image.
\item Power-on the MiniZed with the additional USB power source connected and
a USB stick containing the files to program to the eMMC, boot from the flash-only
image, and run the scripts that program the eMMC.
\item Program the QSPI flash with the fallback image.
\end{enumerate}
%
To ensure the restore procedure works consistently, the MiniZed eMMC and
QSPI flash can first be erased. The following steps were performed on a
MiniZed board programmed with the QSPI U-Boot image
\verb+flash_fallback_7007S.bin+.
%
\begin{enumerate}
% -----------------------------------------------------------------------------
\item \textbf{Erase eMMC Flash}
% -----------------------------------------------------------------------------

The fallback Linux image was booted from U-Boot using the command
\verb+run boot_qspi+. The eMMC partitions are then erased using
the \verb+fdisk+ utility via
%
\begin{verbatim}
root@MiniZed:~# fdisk /dev/mmcblk1

The number of cylinders for this disk is set to 232448.
There is nothing wrong with that, but this is larger than 1024,
and could in certain setups cause problems with:
1) software that runs at boot time (e.g., old versions of LILO)
2) booting and partitioning software from other OSs
   (e.g., DOS FDISK, OS/2 FDISK)

Command (m for help): p

Disk /dev/mmcblk1: 7616 MB, 7616856064 bytes
4 heads, 16 sectors/track, 232448 cylinders
Units = cylinders of 64 * 512 = 32768 bytes

        Device Boot      Start         End      Blocks  Id System
/dev/mmcblk1p1               1        3907      125016  83 Linux

Command (m for help): d
Selected partition 1

Command (m for help): p

Disk /dev/mmcblk1: 7616 MB, 7616856064 bytes
4 heads, 16 sectors/track, 232448 cylinders
Units = cylinders of 64 * 512 = 32768 bytes

        Device Boot      Start         End      Blocks  Id System

Command (m for help): w
The partition table has been altered.
Calling ioctl() to re-read partition table
fdisk: WARNING: rereading partition table failed, kernel still uses old table:
Device or resource busy
root@MiniZed:~# reboot
\end{verbatim}
%
In this example console output, the \verb+p+ command was used to print the
partitions (there was only one), and the \verb+d+ command was used to delete
the partition.

After reboot, halt the U-Boot boot sequence, start Linux using
\verb+run boot_qspi+, and use \verb+fdisk+ to confirm there are no
partitions on the eMMC. U-Boot can also be used to view the
empty partition table via
%
\begin{verbatim}
Zynq> mmc dev 1
sdhci_transfer_data: Error detected in status(0x208000)!
switch to partitions #0, OK
mmc1(part 0) is current device
Zynq> mmc part

Partition Map for MMC device 1  --   Partition Type: DOS

Part    Start Sector    Num Sectors     UUID            Type
Zynq>
\end{verbatim}
%

% -----------------------------------------------------------------------------
\item \textbf{Erase QSPI Flash}
% -----------------------------------------------------------------------------

The U-Boot commands to erase QSPI flash are
%
\begin{verbatim}
Zynq> sf probe
SF: Detected n25q128 with page size 256 Bytes, erase size 64 KiB, total 16 MiB
Zynq> sf erase 0 1000000
SF: 16777216 bytes @ 0x0 Erased: OK
\end{verbatim}
%
Power-down the board (remove the USB connections), and change the Flash/JTAG
boot mode switch to JTAG mode.

% -----------------------------------------------------------------------------
\item \textbf{Program QSPI Flash with} \verb+flash_only_boot_7007S.bin+
% -----------------------------------------------------------------------------

Use one of the flash programming procedures outlined in
Section~\ref{sec:qspi_flash_programming} to program the image.
After programming, power-down the board, and change the Flash/JTAG boot
mode switch to Flash mode.

% -----------------------------------------------------------------------------
\item \textbf{Prepare a USB flash drive}
% -----------------------------------------------------------------------------

Format a USB flash drive in FAT or FAT32 format, and copy the restore image
files from within the zip file:
%
\begin{verbatim}
+-- USB memory stick files
    +-- image.ub
    +-- smallboot.bin
    +-- wpa_supplicant.conf
\end{verbatim}
%
onto the USB drive. Edit \verb+wpa_supplicant.conf+ to match your local
WiFi network settings.

% -----------------------------------------------------------------------------
\item \textbf{Power the MiniZed with two USB cables}
% -----------------------------------------------------------------------------

The MiniZed USB host-mode interface is used to access the USB flash drive.
The power for the USB host-mode interface is provided by the second
micro-USB connector identified with the label \emph{Auxiliary Power} in
Figure~\ref{fig:minized_quick_start_diagram}.

Insert the USB flash drive in the MiniZed USB host-mode connector, and
connect both micro-USB connectors to your development host. The flash-only
boot image contains Linux. The Linux version information is
%
\begin{verbatim}
PetaLinux 2016.4 plnx_arm /dev/ttyPS0

plnx_arm login: root
Password:
root@plnx_arm:~# uname -a
Linux plnx_arm 4.6.0-xilinx #1 SMP PREEMPT Thu Jul 27 22:24:51 PDT 2017
armv7l GNU/Linux
\end{verbatim}

% -----------------------------------------------------------------------------
\item \textbf{Restore MiniZed Linux}
% -----------------------------------------------------------------------------

The Linux commands used to restore the MiniZed eMMC Linux installation to factory state are
implemented in the script \verb+onetest.sh+ (p16~\cite{Avnet_MiniZed_Restore_2018}).
The location and contents of the script can be viewed via
%
\begin{verbatim}
root@plnx_arm:~# which onetest.sh
/usr/local/bin/onetest.sh
root@plnx_arm:~# cat /usr/local/bin/onetest.sh
\end{verbatim}
%
The script creates a DOS partition on the eMMC, formats the partition to use a FAT32
filesystem, mounts the eMMC filesystem, mounts the USB flash stick, copies the files
from the USB stick to eMMC, programs the QSPI flash with \verb+smallboot.bin+,
loads the WiFi driver, brings up the WiFi interface, runs the I2C and LED
test application, and then shuts down the system.

Run the script. Programming the QSPI flash via the \verb+flashcp+ command takes
the most time. The I2C and LED test application will run until you press a key,
and then the board will shutdown.

Press the MiniZed reset button, interrupt the U-Boot boot process, and confirm
the U-Boot checksum is that of \verb+smallboot.bin+.

% -----------------------------------------------------------------------------
\item \textbf{Program QSPI Flash with} \verb+flash_fallback_7007S.bin+
% -----------------------------------------------------------------------------

The Avnet restore procedure lists this as an optional step
(p21~\cite{Avnet_MiniZed_Restore_2018}).
Use one of the flash programming procedures outlined in
Section~\ref{sec:qspi_flash_programming} to program the fallback image.

% -----------------------------------------------------------------------------
\item \textbf{MiniZed Tests}
% -----------------------------------------------------------------------------

The Avnet restore procedure has instructions for testing the MiniZed
%
\begin{itemize}
\item Bluetooth (p18~\cite{Avnet_MiniZed_Restore_2018})
\item The Power Management controller (p19~\cite{Avnet_MiniZed_Restore_2018})
\item Motion sensor and GPIO LEDs (p20~\cite{Avnet_MiniZed_Restore_2018})
\end{itemize}

\newpage
% -----------------------------------------------------------------------------
\item \textbf{Save the U-Boot Environment to QSPI Flash}
% -----------------------------------------------------------------------------

If the flash sector containing the environment is blank, or the CRC
is incorrect, the U-Boot power-on messages contain the warning
%
\begin{verbatim}
*** Warning - bad CRC, using default environment
\end{verbatim}
%
This warning can be eliminated by saving the U-Boot environment.
%
\begin{verbatim}
Zynq> saveenv
Saving Environment to SPI Flash...
SF: Detected n25q128 with page size 256 Bytes, erase size 64 KiB, total 16 MiB
Erasing SPI flash...Writing to SPI flash...done
\end{verbatim}
%
The environment sector does not need to be erased, as U-Boot does that
before writing to the SPI flash. The environment sector can be erased via
%
\begin{verbatim}
Zynq> sf erase ff0000 10000
SF: 65536 bytes @ 0xff0000 Erased: OK
\end{verbatim}
%
After erasing the environment section, press the reset button, halt the U-Boot boot
sequence, and the CRC error message will be present in the U-Boot console
messages.
\end{enumerate}
