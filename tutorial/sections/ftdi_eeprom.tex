% =============================================================================
\section{FTDI EEPROM Programming}
% =============================================================================
\label{sec:ftdi_eeprom}

The MiniZed uses the \href{https://ftdichip.com}{FTDI} FT2232H USB interface to 
implement USB-to-JTAG and USB-to-UART interfaces. The USB device strings are
stored on a 2kbit EEPROM (93LC56B) connected to the FT2232H device. The 
FTDI \href{https://ftdichip.com/utilities}{FT\_PROG EEPROM Programming 
Utility} can be used to \emph{read} the EEPROM contents, however, this
application should not be used to write to the EEPROM, as it does not preserve
the Xilinx/Avnet/Digilent vendor-specific device identification bytes added to the 
user area of the EEPROM. Figure~\ref{fig:minized_ft_prog} shows the FT\_PROG
view for two different MiniZed boards; the 16-bit words in user area 
of the EEPROM can be seen in the bottom of the figure. 

The FTDI library 
\href{https://ftdichip.com/drivers/d2xx-drivers}{D2XX}~\cite{FTDI_D2XX_2019}
provides functions to implement USB data transfers, and to read and write
the EEPROM.
%
The open-source \href{https://www.intra2net.com/en/developer/libftdi}{libFTDI}
provides similar functionality.
%
% libftdi/doc/EEPROM-structure contains notes on the EEPROM  structure
%
The vendor-specific device identification bytes in the FTDI
EEPROM are not officially documented, however, some details have been
reverse-engineered, eg., see~\cite{Bartik_2019}.
%
Reverse-engineering the EEPROM format was made unnecessary by Digilent 
customer support inadvertantly posting their FTDI configuration tool 
\href{https://forum.digilentinc.com/topic/1816-digilent-smt1-recovery}
{FTDIConfig.exe} on their support forum.
Figure~\ref{fig:digilent_ftdi_config} shows the Digilent FTDI configuration
view for two different MiniZed boards.

The \texttt{MiniZed1} EEPROM was updated using a custom application based
on the FTDI D2XX library that;
%
read the EEPROM using \texttt{FT\_ReadEE},
replaced the default serial string, \texttt{1234-oj1}, 
with the identical length string, \texttt{MiniZed1},
updated the checksum,
and wrote the EEPROM using \texttt{FT\_WriteEE}. 
%
Vivado correctly recognized the device as \texttt{MiniZed1}, and Windows 
10 created a new COM port specific to this USB device. The same application 
was then used to reset the EEPROM to the original serial string; Vivado 
recognized the device as \texttt{1234-oj1} and TeraTerm found the device 
on the previously assigned COM port. 

The Digilent FTDI configuration tool was then used to program the MiniZed 
with the new serial string \texttt{MiniZed1}. 
%
Figure~\ref{fig:minized1_eeprom} compares the \texttt{MiniZed1} EEPROM image 
derived from the \texttt{1234-oj1} image, with the image created by the 
Digilent configuration tool. There are three differences; the 16-bit word 
at offset 0x000D  changes from \textcolor{blue}{0x0003} to 
\textcolor{red}{0x0001}, the 16-bit word at offset 0x006A changes from 
\textcolor{blue}{0x0001} to \textcolor{red}{0x0000}, and the final 
16-bit words (the checksums) change.
%
The change at offset 0x000D may be the version 3 value for FT2232H extensions 
mentioned in FTDI AN428.
%
The FTDI FT\_PROG and Digilent FTDI configuration tools were then used to 
program the MiniZed with the \emph{longer} serial string \texttt{MiniZed01}; 
the string offsets in the FTDI header bytes changed to accommodate the 
longer serial number string. 
%
Based on these tests, the application \texttt{minized\_ftdi\_serial} was 
created to read or write the MiniZed serial string.

% -----------------------------------------------------------------------------
% MiniZed EEPROM images
% -----------------------------------------------------------------------------
%
\begin{figure}[t]
\begin{minipage}{0.5\textwidth}
\begin{center}
\footnotesize
{\tt
0000: 0801 0403 6010 0700 3280 0008 0000 0E9A
0008: 18A8 12C0 0000 0000 0056 \textcolor{blue}{0003} 584A 7641
0010: 656E 0074 694D 696E 655A 2064 3156 0000
0018: 0000 0000 0000 0000 0000 0000 0000 0000
0020: 0000 0000 0000 0000 0000 0000 0000 0000
0028: 0000 0000 0000 0000 0000 0000 0000 0000
0030: 0000 0000 0000 0000 0000 0000 0000 0000
0038: 0000 0000 0000 0000 0000 0000 0000 0000
0040: 0000 0000 0000 0000 0000 0000 0000 0000
0048: 0000 0000 0000 0000 0000 030E 0058 0069
0050: 006C 0069 006E 0078 0318 004A 0054 0041
0058: 0047 002B 0053 0065 0072 0069 0061 006C
0060: 0312 004D 0069 006E 0069 005A 0065 0064
0068: 0031 0302 \textcolor{blue}{0001} 0000 0000 0000 0000 0000
0070: 0000 0000 0000 0000 0000 0000 0000 0000
0078: 0000 0000 0000 0000 0000 0000 0000 \textcolor{blue}{EC87}}
\vskip3mm
(a) Based on original MiniZed image
\end{center}
\end{minipage}
\hfil
\begin{minipage}{0.5\textwidth}
\begin{center}
\footnotesize
{\tt
0000: 0801 0403 6010 0700 3280 0008 0000 0E9A
0008: 18A8 12C0 0000 0000 0056 \textcolor{red}{0001} 584A 7641
0010: 656E 0074 694D 696E 655A 2064 3156 0000
0018: 0000 0000 0000 0000 0000 0000 0000 0000
0020: 0000 0000 0000 0000 0000 0000 0000 0000
0028: 0000 0000 0000 0000 0000 0000 0000 0000
0030: 0000 0000 0000 0000 0000 0000 0000 0000
0038: 0000 0000 0000 0000 0000 0000 0000 0000
0040: 0000 0000 0000 0000 0000 0000 0000 0000
0048: 0000 0000 0000 0000 0000 030E 0058 0069
0050: 006C 0069 006E 0078 0318 004A 0054 0041
0058: 0047 002B 0053 0065 0072 0069 0061 006C
0060: 0312 004D 0069 006E 0069 005A 0065 0064
0068: 0031 0302 \textcolor{red}{0000} 0000 0000 0000 0000 0000
0070: 0000 0000 0000 0000 0000 0000 0000 0000
0078: 0000 0000 0000 0000 0000 0000 0000 \textcolor{red}{ECAF}}
\vskip3mm
(b) Digilent FTDI programmer
\end{center}
\end{minipage}
\caption{FTDI EEPROM images for \texttt{MiniZed1}.}
\label{fig:minized1_eeprom}
\end{figure}
% -----------------------------------------------------------------------------

\clearpage
% -----------------------------------------------------------------------------
% FTDI FT_PROG
% -----------------------------------------------------------------------------
%
\begin{figure}[p]
  \begin{minipage}{0.5\textwidth}
    \begin{center}
    \includegraphics[width=0.95\textwidth]
    {figures/minized_ft_prog_1234-oj1.png}\\
    (a) Serial number: \texttt{1234-oj1}
    \end{center}
  \end{minipage}
  \hfil
  \begin{minipage}{0.5\textwidth}
    \begin{center}
    \includegraphics[width=0.95\textwidth]
    {figures/minized_ft_prog_MiniZed1.png}\\
    (b) Serial number: \texttt{MiniZed1}
    \end{center}
  \end{minipage}
  \caption{FTDI FT\_PROG view of two MiniZed boards.}
  \label{fig:minized_ft_prog}
\end{figure}
% -----------------------------------------------------------------------------

% -----------------------------------------------------------------------------
% Digilent FTDI Config
% -----------------------------------------------------------------------------
%
\begin{figure}[p]
  \begin{minipage}{0.5\textwidth}
    \begin{center}
    \includegraphics[width=0.95\textwidth]
    {figures/minized_digilent_ftdi_config_1234-oj1.png}\\
    (a) Serial number: \texttt{1234-oj1}
    \end{center}
  \end{minipage}
  \hfil
  \begin{minipage}{0.5\textwidth}
    \begin{center}
    \includegraphics[width=0.95\textwidth]
    {figures/minized_digilent_ftdi_config_MiniZed1.png}\\
    (b) Serial number: \texttt{MiniZed1}
    \end{center}
  \end{minipage}
  \caption{Digilent FTDI configuration view of two MiniZed boards.}
  \label{fig:digilent_ftdi_config}
\end{figure}
% -----------------------------------------------------------------------------
