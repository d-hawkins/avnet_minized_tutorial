%------------------------------------------------------------------------------
\subsection{PS bicolor LED control via U-Boot}
%------------------------------------------------------------------------------

The MiniZed schematic (p4~\cite{Avnet_MiniZed_Schematic_2017}) shows that the
PS bicolor LED is controlled by the Zynq GPIO signals MIO52
(\verb+PS_LED_R+) and MIO53 (\verb+PS_LED_G+).
%
The Zynq TRM describes the GPIO multiplexed I/O (MIO) and extended multiplexed
I/O (EMIO) in Chapter 14~\cite{Xilinx_UG585_2021}.
%
Figure~\ref{fig:zynq_gpio_block_diagram} shows the GPIO block diagram;
banks 0 and 1 connect to the MIO, while banks 2 and 3 connect to the EMIO.
%
Figure~\ref{fig:zynq_gpio_block_diagram} shows the GPIO control registers.
The GPIO control register definitions and location in the Zynq address map
is defined in Appendix B.19~\cite{Xilinx_UG585_2021}. The GPIO registers
are located at address 0xE000A000.

The PS bicolor LED is controlled by MIO bits 52 and 53 which are located in GPIO
bank 1. The GPIO bank 1 control registers are; \verb+DATA+ (0xE000A00C),
\verb+DIRM+ (0xE000A244), and \verb+OEN+ (0xE000A248) registers.
%
The PS bicolor LED can be controlled from U-Boot as follows:
%
\begin{verbatim}
# Set the direction
Zynq> mw e000a244 00300000

# Enable the output (the bicolor LED turns off)
Zynq> mw e000a248 00300000

# Write to the registers using the mask
Zynq> mw e000a00c ffcf0000  # off
Zynq> mw e000a00c ffcf0010  # red
Zynq> mw e000a00c ffcf0020  # green
Zynq> mw e000a00c ffcf0030  # red+green
\end{verbatim}
%
The PL LEDs can also be controlled using U-Boot commands, however, the
programmable logic first needs to be configured with a design that
either routes the EMIO to the LEDs, or implements the AXI GPIO that
controls the LEDs. Using U-Boot for read and write access to Zynq PS
registers and PL designs is a simple, but powerful, debugging tool.

\clearpage
% -----------------------------------------------------------------------------
% Zynq GPIO Block Diagram
% -----------------------------------------------------------------------------
%
\begin{figure}[p]
  \begin{center}
    \includegraphics[width=0.5\textwidth]
    {figures/zynq_gpio_block_diagram.png}
  \end{center}
  \caption{Zynq GPIO block diagram (from Figure 14-1~\cite{Xilinx_UG585_2021}).}
  \label{fig:zynq_gpio_block_diagram}
\end{figure}
% -----------------------------------------------------------------------------

% -----------------------------------------------------------------------------
% Zynq GPIO Registers
% -----------------------------------------------------------------------------
%
\begin{figure}[p]
  \begin{center}
    \includegraphics[width=0.75\textwidth]
    {figures/zynq_gpio_registers.png}
  \end{center}
  \caption{Zynq GPIO control registers (from Figure 14-2~\cite{Xilinx_UG585_2021}).}
  \label{fig:zynq_gpio_registers}
\end{figure}
% -----------------------------------------------------------------------------

\clearpage
%------------------------------------------------------------------------------
\subsection{PS bicolor LED control via GPIO MIO}
%------------------------------------------------------------------------------
\label{sec:blinky_gpio_mio}

The PS bicolor LED control via the GPIO MIO design uses only the Processor
System (PS), it does not use the Programmable Logic (PL).
%
The Xilinx Vivado tool is used to configured the Zynq Processor, and
the Xilinx Vitis tool is used to develop a bare-metal
application to blink the bicolor LED. The PS bicolor LED design duplicates
many of the steps found in Avnet MiniZed tutorials 01, 02, and
04~\cite{Avnet_MiniZed_Tutorial01_2018,Avnet_MiniZed_Tutorial02_2018,Avnet_MiniZed_Tutorial04_2018}.
The Avnet tutorials should be reviewed in conjunction with this design.

\begin{enumerate}
%------------------------------------------------------------------------------
\item \textcolor{blue}{\textbf{Vivado Project Creation}}
%------------------------------------------------------------------------------

Create the MiniZed board \verb+blinky_gpio_mio+ project:
%
\begin{enumerate}
\item Start Vivado.
\item Under \emph{Quick Start}, click on \emph{Create Project}.
\item {\bf Create a New Vivado Project}
%
\begin{itemize}
\item Click \emph{Next}
\end{itemize}
%
\item {\bf Project Name}
%
\begin{itemize}
\item \emph{Project name}: \verb+minized+
\item \emph{Project location}: \verb+c:/temp/minized/blinky_gpio_mio/vwork+
\item Uncheck \emph{Create project subdirectory}
\item Click \emph{Next}
\end{itemize}
%
The project location specifies the Vivado work directory, \verb+vwork+,
where Vivado generates the project files. After the GPIO MIO project has
been completed, a script is developed that automates the project generation.
%
\item {\bf Project Dialog}
%
\begin{itemize}
\item Select the \emph{RTL Project} bullet
\item Check \emph{Do not specify sources at this time}
\item Click \emph{Next}
\end{itemize}
%
\item {\bf Default Part}
%
\begin{itemize}
\item Click on the \emph{Boards} tab
\item On the \emph{Vendor} pull-down, select \emph{avnet.com}
\item Select the \emph{MiniZed} in the board list
\item Click \emph{Next}
\end{itemize}
%
\item {\bf New Project Summary}
%
\begin{itemize}
\item Click \emph{Finish}
\end{itemize}
%
\end{enumerate}
%
%------------------------------------------------------------------------------
\item \textcolor{blue}{\textbf{Vivado Block Diagram Creation}}
%------------------------------------------------------------------------------
%
\begin{enumerate}
%
\item In the \emph{Project Manager}, under \emph{IP Integrator},
select \emph{Create Block Design}.
%
\item {\bf Create Block Design}
%
\begin{itemize}
\item \emph{Design name}: \verb+system+
\item \emph{Directory}: \verb+<Local to Project>+
\item \emph{Specify Source set}: \verb+Design Sources+
\item Click \emph{OK}
\end{itemize}
%
\newpage
\item {\bf Add the Zynq Processor}
%
\begin{itemize}
\item Click the `+' in the middle of the design area to open the IP list
\item In the \emph{Search} bar enter \verb+Zynq+
\item Double-click on \verb+ZYNQ7 Processing System+ to add the Zynq to
the design
\item In the {\bf Block Properties} dialog, change the name from
\verb+processing_system7_0+ to \verb+zynq+.
\end{itemize}
%
\item {\bf Run Block Automation}
%
\begin{itemize}
\item Click on the blue hyperlink \textcolor{blue}{Run Block Automation}
\item Accept the block automation defaults by clicking \emph{OK}
\end{itemize}
%
\end{enumerate}
%
% -----------------------------------------------------------------------------
% PS bicolor LED GPIO MIO Design - Vivado Block Diagram
% -----------------------------------------------------------------------------
%
% Installed "A Ruler for Windows" and set the Diagram GUI detached
% window size to 5in x 10in for screen capture
%
\begin{figure}[t]
  \begin{minipage}{\textwidth}
    \begin{center}
    \includegraphics[width=0.8\textwidth]
    {figures/blinky_gpio_mio_vivado_diagram_a.png}\\
    (a) Zynq processor
    \end{center}
  \end{minipage}
  \vskip5mm
  \begin{minipage}{\textwidth}
    \begin{center}
    \includegraphics[width=0.8\textwidth]
    {figures/blinky_gpio_mio_vivado_diagram_b.png}\\
    (c) Zynq processor after block automation
    \end{center}
  \end{minipage}
  \caption{PS bicolor LED control via GPIO MIO design Vivado diagram window.}
  \label{fig:blinky_gpio_mio_vivado_diagram}
\end{figure}
% -----------------------------------------------------------------------------
%
Figure~\ref{fig:blinky_gpio_mio_vivado_diagram}(a) shows the diagram
window after adding the Zynq to the \verb+system+ design.
Figure~\ref{fig:blinky_gpio_mio_vivado_diagram}(b) shows the
diagram window after running block automation.
%
Block automation configures the Zynq processor using the board
presets, \verb+preset.xml+, which is part of the MiniZed
board definition files (BDF) downloaded from the 
\href{https://github.com/Avnet/bdf}{Avnet BDF} git repository
and copied into the Vivado board files directory.
This example uses the MiniZed Zynq processor defaults. Later design
examples customize the Zynq processor.

%------------------------------------------------------------------------------
\item \textcolor{blue}{\textbf{Vivado Synthesis}}
%------------------------------------------------------------------------------

The need for \emph{Synthesis} in a design that does not use the programmable
logic sounds redundant. However, the Zynq design flow uses Vivado to configure
the (required) Zynq processor system and the (optional) programmable logic.
%
Vivado generates the hardware design file needed by Vitis to develop software. 
Vivado is used to configure the Zynq with the MiniZed presets, so that Vitis 
can be used to create the blinky LED application. The hardware support file
needed by Vitis is generated as follows.
%
\begin{enumerate}
%
\item {\bf Create HDL Wrapper}
%
\begin{itemize}
\item In the Vivado \emph{Sources} window, highlight \verb+system (system.bd)+
\item Right-mouse-click and select \emph{Create HDL Wrapper}
\item Accept the default option \emph{Let Vivado manage wrapper and auto-update}
\item Click \emph{OK}
\item This generates the top-level design file \verb+system_wrapper.v+
\end{itemize}
%
\item {\bf Constraints}

This design uses only Zynq processor system I/Os, no programmable logic I/Os,
so pin constraints are not required. The Zynq pins are configured by the
FSBL, which is generated by Vivado as part of the synthesis step.
%
\item {\bf Generate Bitstream}
%
\begin{itemize}
\item In the Vivado \emph{Project Manager} click \emph{Generate Bitstream}
\item Vivado then runs \emph{Synthesis}, \emph{Implementation}, and 
\emph{Generate Bitstream}
\item The top-right of the Vivado GUI provides status on the process
currently running. The \emph{Design Runs} tab contains more information.
\item Synthesis and implementation generate output in the directory

\verb+c:/temp/minized/blinky_gpio_mio/vwork/minized.runs+

\item When the \emph{Bitstream Generation Completed} dialog appears, close it
by clicking \emph{OK}, which opens the \emph{Implemented Design}.

\item Run \emph{Report Utilization} and the generated \emph{Utilization} 
report shows that the Zynq design uses no logic.
\item Click on the \emph{Power} tab to see that total power estimate is 1.292W.
\end{itemize}
%
\item {\bf Export Hardware}
%
\begin{itemize}
\item Select \emph{File}$\rightarrow$\emph{Export}$\rightarrow$\emph{Export Hardware}
\item The \emph{Export Hardware Platform} GUI opens. Click \emph{Next}.
\item The \emph{Output} page contains two options;
\begin{itemize}
\item \emph{Pre-synthesis} (default selection) 
\item \emph{Include bitstream}
\end{itemize}
There is no PL in this design, so the default can be accepted by
clicking \emph{Next}.
\item Accept the default file names; 
\begin{itemize}
\item \emph{XSA file name:} \verb+system_wrapper+,
\item \emph{Export to:} \verb+c:/temp/minized/blinky_gpio_mio/vwork+,
\end{itemize}
\item Click \emph{Next}, and then \emph{Finish}
\item This step generates the XSA file used by Vitis
%
\begin{verbatim}
c:/temp/minized/blinky_gpio_mio/vwork/system_wrapper.xsa
\end{verbatim}
\end{itemize}
%
\newpage
\item {\bf XSA Analysis} (optional step)
%
\begin{itemize}
\item Create a copy of the XSA file
\item Change the extension from \verb+.xsa+ to \verb+.zip+
\item Unzip the file and view the contents of \verb+system_wrapper+:
\begin{itemize}
\item \verb+ps7_init.tcl+

Used by Vitis to initialize the processor during JTAG download

\item \verb+ps7_init*.h/c+

First stage bootloader (FSBL) source.

\end{itemize}
\item Delete the zip file and the \verb+system_wrapper+ folder
\end{itemize}
%
\end{enumerate}
%
%------------------------------------------------------------------------------
\item \textcolor{blue}{\textbf{Vitis Platform Project Creation}}
%------------------------------------------------------------------------------

Vitis can be started from within Vivado using the menu option
\emph{Tools}$\rightarrow$\emph{Launch Vitis IDE}, or it can be
started from the Windows Start Menu (or from a Linux console).

\begin{itemize}
\item Set the Vitis workspace to

\verb+c:/temp/minized/blinky_gpio_mio/vwork/workspace+

and click \emph{Launch}

\item From the Vitis \emph{Welcome} page, select \emph{Create Platform Project}
\item Set the platform project name to \verb+minized_platform+ and click \emph{Next}
\item Under \emph{Hardware Specification}, click \emph{Browse}, and 
select\\
\verb+c:/temp/minized/blinky_gpio_mio/vwork/system_wrapper.xsa+
\item The \emph{Software Specification} defaults are;
\begin{itemize}
\item Operating system: standalone
\item Processor: ps7\_cortexa9\_0
\end{itemize}
\item The \emph{Boot components} default is;
\begin{itemize}
\item (checked) Generate boot components
\end{itemize}
\item Click \emph{Finish}
\item Vitis opens and displays the \verb+minized_platform+
\item Under the \emph{Explorer} window, expand the \verb+minized_platform/hw+ folder
\item Double-click on \verb+system_wrapper.xsa+ to view the address map of the
Zynq design.
\item Table~\ref{tab:blinky_gpio_mio} shows a selection of addresses
relevant to the GPIO MIO design.

\item Double-click on \verb+ps7_init.html+ to see a detailed view
of the address map, and the register settings. The registers are configured
when Vitis runs \texttt{ps7\_init.tcl} during JTAG download, or when the
processor boots and runs the FSBL.

\item Scroll down to \textbf{MIO 52} and \textbf{MIO 53} to see that the
multiplexed I/O (MIO) used to control the red and green PS LEDs are configured
as GPIO.

\item Scroll further down and see that the PS-to-PL clocks
\textbf{FPGA0, 1, 2, 3} are configured for frequencies of
50MHz, 100MHz, 10MHz, and 10MHz. The PS-to-PL clocks are not used
in the GPIO MIO example.
\end{itemize}

% -----------------------------------------------------------------------------
% Blinky GPIO MIO Address Map
% -----------------------------------------------------------------------------
%
\begin{table}[t]
\caption{PS bicolor LED GPIO MIO Example Address Map (selected addresses).}
\label{tab:blinky_gpio_mio}
\begin{center}
\begin{tabular}{|l|l|l|}
\hline
Zynq Component & Cell & Base Address\\
\hline\hline
\textbf{Memory} &&\\
\hline
&&\\
On-Chip Memory\#0                 & \texttt{ps7\_ram\_0}          & \texttt{0x00000000}\\
DDR Memory                        & \texttt{ps7\_ddr\_0}          & \texttt{0x00100000}\\
QSPI Linear (Read-Only) Addresses & \texttt{ps7\_qspi\_linear\_0} & \texttt{0xFC000000}\\
On-Chip Memory\#1                 & \texttt{ps7\_ram\_1}          & \texttt{0xFFFF0000}\\
&&\\
\hline
\textbf{Peripheral} &&\\
\hline
&&\\
UART\#0 Registers    & \texttt{ps7\_uart\_0}         & \texttt{0xE0000000}\\
UART\#1 Registers    & \texttt{ps7\_uart\_1}         & \texttt{0xE0001000}\\
GPIO MIO  Registers  & \texttt{ps7\_gpio\_0}         & \texttt{0xE000A000}\\
QSPI Registers       & \texttt{ps7\_qspi\_0}         & \texttt{0xE000D000}\\
&&\\
DDR Registers        & \texttt{ps7\_ddrc\_0}         & \texttt{0xF8006000}\\
&&\\
\hline
\end{tabular}
\end{center}
\end{table}
% -----------------------------------------------------------------------------

\newpage
%------------------------------------------------------------------------------
\item \textcolor{blue}{\textbf{Vitis Board Support Package (BSP) Update}}
%------------------------------------------------------------------------------

The MiniZed connects the Zynq UART\#0 to the bluetooth device and UART\#1 to the
FTDI USB UART interface. The two \emph{Board Support Package} entries within
the \verb+minized_platform+ need to be configured to use UART\#1 for standard 
input/output as follows:
%
\begin{itemize}
\item In the \verb+minized_platform+ window, under \verb+minized_platform/ps7_cortexa9_0+
\item Click on \verb+zynq_fsbl/Board Support Package+
\item Click on \emph{Modify BSP Settings}
\item Select \emph{Overview}, \emph{standalone}
\item Change the pull-downs for \texttt{stdin} and \texttt{stdout} to \texttt{ps7\_uart\_1}
\item Click \emph{OK}
\item Click on \verb+standalone on ps7_cortexa9_0/Board Support Package+
and change \texttt{stdin} and \texttt{stdout} to UART\#1.
\item The \emph{Drivers} list for
\verb+standalone on ps7_cortexa9_0/Board Support Package+
contains links to documentation and example designs for the components in 
the system.
\end{itemize}

%------------------------------------------------------------------------------
\item \textcolor{blue}{\textbf{Vitis Bare-metal Application}}
%------------------------------------------------------------------------------

The PS bicolor LED control application is based on the
\emph{Hello World} example application. The \emph{Hello World} application 
configures the project to use the \texttt{platform.h/c} files, which 
define the functions for initialization and cleanup. The platform 
initialization configures the UART interface to be used for C library 
standard input and output.

\begin{itemize}
\item Select \emph{File}$\rightarrow$\emph{New}$\rightarrow$\emph{Application Project}
\item On the \emph{Create a New Application Project} page, click \emph{Next}.
\item On the \emph{Platform} page, the \verb+minized_platform+ is already 
selected, so click \emph{Next}.
\item On the \emph{Application Project Details} page configure the 
application details
\begin{itemize}
\item \emph{Application project name:} \texttt{blinky\_gpio\_mio}
\item \emph{System project name:} \texttt{blinky\_gpio\_mio\_system}
\end{itemize}
Click \emph{Next}
\item On the \emph{Domain} page, the \verb+standalone_domain+ is already 
selected, so click \emph{Next}.
\item On the \emph{Templates} page, select \emph{Hello World}, and
click \emph{Finish}.

\item Expand the application folder, so you can see the source files, i.e., in
the \emph{Explorer} window expand
\verb+blinky_gpio_mio_system/blinky_gpio_mio/src+.

\item Select \texttt{helloworld.c}
\item Right-mouse-click, select \emph{Rename}, and rename the file \texttt{blinky\_gpio\_mio.c}
\item Replace \texttt{blinky\_gpio\_mio.c} with the source code shown in
Figure~\ref{fig:blinky_gpio_mio_app}.
\item Select \texttt{blinky\_gpio\_mio\_system}, right-mouse-click, and
select \emph{Build Application}.
\item The \emph{Console} contains the build messages.
\end{itemize}
%
%------------------------------------------------------------------------------
\item \textcolor{blue}{\textbf{Vitis Download and Run the Application}}
%------------------------------------------------------------------------------
%
\begin{itemize}
\item Set the MiniZed Flash/JTAG mode switch to JTAG mode
\item Connect the MiniZed to your development machine using the USB JTAG/UART interface
\item Connect to the MiniZed UART (using TeraTerm or Minicom) at 115200 baud
\item Under \emph{Explorer}, select the \texttt{blinky\_gpio\_mio}
project
\item To download and run the application, right-mouse-click, and 
select \emph{Run As}$\rightarrow$\emph{Launch on Hardware 
(Single Application Debug)}.
\item The PL does not need to be programmed for this application to run;
the blue configuration done LED will remain off.
\item The PS bicolor LED will cycle through off, green, red, and amber (red+green).
\item The PL bicolor LED is not actively driven, so it remains amber.
\item The UART will output a message for each PS bicolor LED change, eg.,
\begin{verbatim}
MiniZed Blinky GPIO MIO Example
-------------------------------
0: Off
1: Green
2: Red
3: Amber
4: Off
5: Green
6: Red
7: Amber
\end{verbatim}
\end{itemize}

\textcolor{magenta}{\bf Congratulations!} You have run your first custom application on the MiniZed board.

%------------------------------------------------------------------------------
% blinky_gpio_mio bare-metal application code
%------------------------------------------------------------------------------
%
\begin{figure}
\begin{center}
\begin{minipage}{0.8\textwidth}
\begin{lstlisting}
#include <sleep.h>
#include "platform.h"
#include "xil_printf.h"

int main()
{
	int i = 0;

	init_platform();

	xil_printf("MiniZed Blinky GPIO MIO Example\r\n");
	xil_printf("-------------------------------\r\n");

	// Clear the data (DATA) bits
	*(volatile unsigned int *)0xE000A00C = 0xFFCF0000;

	// Set the direction (DIRM) to output
	*(volatile unsigned int *)0xE000A244 = 0x00300000;

	// Enable the outputs (OEN)
	*(volatile unsigned int *)0xE000A248 = 0x00300000;

	// Blink the bicolor LED
	while (1) {
		switch (i%4) {
			case 0:
				xil_printf("%d: Off\r\n", i);
				*(volatile unsigned int *)0xE000A00C = 0xFFCF0000;
				break;

			case 1:
				xil_printf("%d: Green\r\n", i);
				*(volatile unsigned int *)0xE000A00C = 0xFFCF0020;
				break;

			case 2:
				xil_printf("%d: Red\r\n", i);
				*(volatile unsigned int *)0xE000A00C = 0xFFCF0010;
				break;

			default:
				xil_printf("%d: Amber\r\n", i);
				*(volatile unsigned int *)0xE000A00C = 0xFFCF0030;
				break;
		}
		sleep(1);
		i++;
	}

    cleanup_platform();
    return 0;
}
\end{lstlisting}
\end{minipage}
\end{center}
\caption{PS bicolor LED GPIO MIO bare-metal application.}
\label{fig:blinky_gpio_mio_app}
\end{figure}
%------------------------------------------------------------------------------

\clearpage
%------------------------------------------------------------------------------
\item \textcolor{blue}{\textbf{Create the QSPI Flash Boot Image}}
%------------------------------------------------------------------------------
%
\begin{itemize}
\item In \emph{Explorer}, select \verb+blinky_gpio_mio_system+.
\item Right-mouse-click and select \emph{Create Boot Image}.

Figure~\ref{fig:blinky_gpio_mio_bootgen_gui} shows the 
\emph{Create Boot Image} dialog.
The defaults will generate the file \verb+BOOT.bin+. 
The Avnet MiniZed Tutorial 04 indicates that the boot image file type 
should be changed to .MCS~\cite{Avnet_MiniZed_Tutorial04_2018}, 
however, this is not necessary.

\item Click \emph{Create Image}
\end{itemize}

%------------------------------------------------------------------------------
\item \textcolor{blue}{\textbf{Program the QPSI Flash Boot Image}}
%------------------------------------------------------------------------------
%
\begin{itemize}
\item In \emph{Explorer}, select \verb+blinky_gpio_mio_system+.
\item Right-mouse-click and select \emph{Program Flash}.

Figure~\ref{fig:blinky_gpio_mio_program_flash_gui} shows the 
\emph{Program Flash Memory} dialog.

\item Click \emph{Program}

\item The Vitis console outputs flash programming messages;
\begin{verbatim}
Connected to hw_server @ TCP:127.0.0.1:3121
Available targets and devices:
Target 0 : jsn-MiniZed V1-MiniZed01A
	Device 0: jsn-MiniZed V1-MiniZed01A-4ba00477-0
Retrieving Flash info...
Initialization done, programming the memory
Using default mini u-boot image file 
[... snip ... ]
Flash Operation Successful
\end{verbatim}
\end{itemize}

%------------------------------------------------------------------------------
\item \textcolor{blue}{\textbf{Boot from QPSI Flash}}
%------------------------------------------------------------------------------
%
\begin{itemize}
\item Power-down the MiniZed, change the boot mode switch to Flash, 
and power-up.

\item The MiniZed will boot from flash and the bicolor LED will start 
cycling off, green, red, and amber. 

\item Connect to the UART and press the MiniZed reset button to reset
the Zynq (but not the FTDI USB-to-UART) and the serial console will
show the application start message and LED color changes, i.e.,
%
\begin{verbatim}
MiniZed Blinky GPIO MIO Example
-------------------------------
0: Off
1: Green
2: Red
3: Amber
4: Off
5: Green
6: Red
7: Amber
\end{verbatim}

\item The PL blue configuration done LED will remain off, as the PS 
bicolor LED project does not use the PL, so configuration done never 
asserts.

\end{itemize}

\textcolor{magenta}{\bf Congratulations!} Your MiniZed board boots and 
runs your first custom application.

The next design adds the progammable logic into the mix!

\end{enumerate} % End of blinky_gpio_mio steps
%
% -----------------------------------------------------------------------------
% PS bicolor LED GPIO MIO Design - Bootgen GUI
% -----------------------------------------------------------------------------
%
\begin{figure}[p]
  \begin{center}
    \includegraphics[width=\textwidth]
    {figures/blinky_gpio_mio_bootgen_gui.png}\\
  \end{center}
  \caption{PS GPIO MIO design \emph{Create Boot Image} window.}
  \label{fig:blinky_gpio_mio_bootgen_gui}
\end{figure}
% -----------------------------------------------------------------------------
%
% -----------------------------------------------------------------------------
% PS bicolor LED GPIO MIO Design - Program Flash GUI
% -----------------------------------------------------------------------------
%
\begin{figure}[p]
  \begin{center}
    \includegraphics[width=\textwidth]
    {figures/blinky_gpio_mio_program_flash_gui.png}\\
  \end{center}
  \caption{PS GPIO MIO design \emph{Program Flash Memory Device} window.}
  \label{fig:blinky_gpio_mio_program_flash_gui}
\end{figure}
% -----------------------------------------------------------------------------
%

