%------------------------------------------------------------------------------
\subsection{PS bicolor LED control via U-Boot}
%------------------------------------------------------------------------------

The MiniZed schematic (p4~\cite{Avnet_MiniZed_Schematic_2017}) shows that the
PS bicolor LED is controlled by the Zynq GPIO signals MIO52
(\verb+PS_LED_R+) and MIO53 (\verb+PS_LED_G+).
%
The Zynq TRM describes the GPIO multiplexed I/O (MIO) and extended multiplexed
I/O (EMIO) in Chapter 14~\cite{Xilinx_UG585_2021}.
%
Figure~\ref{fig:zynq_gpio_block_diagram} shows the GPIO block diagram;
banks 0 and 1 connect to the MIO, while banks 2 and 3 connect to the EMIO.
%
Figure~\ref{fig:zynq_gpio_block_diagram} shows the GPIO control registers.
The GPIO control register definitions and location in the Zynq address map
is defined in Appendix B.19~\cite{Xilinx_UG585_2021}. The GPIO registers
are located at address 0xE000A000.

The PS bicolor LED is controlled by MIO bits 52 and 53 which are located in GPIO
bank 1. The GPIO bank 1 control registers are; \verb+DATA+ (0xE000A00C),
\verb+DIRM+ (0xE000A244), and \verb+OEN+ (0xE000A248) registers.
%
The PS bicolor LED can be controlled from U-Boot as follows:
%
\begin{verbatim}
# Set the direction
Zynq> mw e000a244 00300000

# Enable the output (the bicolor LED turns off)
Zynq> mw e000a248 00300000

# Write to the registers using the mask
Zynq> mw e000a00c ffcf0000  # off
Zynq> mw e000a00c ffcf0010  # red
Zynq> mw e000a00c ffcf0020  # green
Zynq> mw e000a00c ffcf0030  # red+green
\end{verbatim}
%
The PL LEDs can also be controlled using U-Boot commands, however, the
programmable logic first needs to be configured with a design that
either routes the EMIO to the LEDs, or implements the AXI GPIO that
controls the LEDs. Using U-Boot for read and write access to Zynq PS
registers and PL designs is a simple, but powerful, debugging tool.

\clearpage
% -----------------------------------------------------------------------------
% Zynq GPIO Block Diagram
% -----------------------------------------------------------------------------
%
\begin{figure}[p]
  \begin{center}
    \includegraphics[width=0.5\textwidth]
    {figures/zynq_gpio_block_diagram.png}
  \end{center}
  \caption{Zynq GPIO block diagram (from Figure 14-1~\cite{Xilinx_UG585_2021}).}
  \label{fig:zynq_gpio_block_diagram}
\end{figure}
% -----------------------------------------------------------------------------

% -----------------------------------------------------------------------------
% Zynq GPIO Registers
% -----------------------------------------------------------------------------
%
\begin{figure}[p]
  \begin{center}
    \includegraphics[width=0.75\textwidth]
    {figures/zynq_gpio_registers.png}
  \end{center}
  \caption{Zynq GPIO control registers (from Figure 14-2~\cite{Xilinx_UG585_2021}).}
  \label{fig:zynq_gpio_registers}
\end{figure}
% -----------------------------------------------------------------------------

\clearpage
%------------------------------------------------------------------------------
\subsection{PS bicolor LED control via GPIO MIO}
%------------------------------------------------------------------------------
\label{sec:blinky_gpio_mio}

The PS bicolor LED control via the GPIO MIO design uses only the Processor
System (PS), it does not use the Programmable Logic (PL).
%
The Xilinx Vivado tool is used to configured the Zynq Processor, and
the Xilinx Software Development Kit (SDK) is used to develop a bare-metal
application to blink the bicolor LED. The PS bicolor LED design duplicates
many of the steps found in Avnet MiniZed tutorials 01, 02, and
04~\cite{Avnet_MiniZed_Tutorial01_2018,Avnet_MiniZed_Tutorial02_2018,Avnet_MiniZed_Tutorial04_2018}.
The Avnet tutorials should be reviewed in conjunction with this design.

\begin{enumerate}
%------------------------------------------------------------------------------
\item \textcolor{blue}{\textbf{Vivado Project Creation}}
%------------------------------------------------------------------------------

Create the MiniZed board \verb+blinky_gpio_mio+ project:
%
\begin{enumerate}
\item Start Vivado.
\item Under \emph{Quick Start}, click on \emph{Create Project}.
\item {\bf Create a New Vivado Project}
%
\begin{itemize}
\item Click \emph{Next}
\end{itemize}
%
\item {\bf Project Name}
%
\begin{itemize}
\item \emph{Project name}: \verb+blinky_gpio_mio+
\item \emph{Project location}: \verb+c:/temp/minized/blinky_gpio_mio/vwork+
\item Uncheck \emph{Create project subdirectory}
\item Click \emph{Next}
\end{itemize}
%
The project location specifies the Vivado work directory, \verb+vwork+,
where Vivado generates the project files. After the GPIO MIO project has
been completed, a script is developed that automates the project generation.
%
\item {\bf Project Dialog}
%
\begin{itemize}
\item Select the \emph{RTL Project} bullet
\item Check \emph{Do not specify sources at this time}
\item Click \emph{Next}
\end{itemize}
%
\item {\bf Default Part}
%
\begin{itemize}
\item Click on the \emph{Boards} tab
\item On the \emph{Vendor} pull-down, select \emph{em.avnet.com}
\item Select the \emph{MiniZed} in the board list
\item Click \emph{Next}
\end{itemize}
%
\item {\bf New Project Summary}
%
\begin{itemize}
\item Click \emph{Finish}
\end{itemize}
%
\end{enumerate}
%
%------------------------------------------------------------------------------
\item \textcolor{blue}{\textbf{Vivado Block Diagram Creation}}
%------------------------------------------------------------------------------
%
\begin{enumerate}
%
\item In the \emph{Project Manager}, under \emph{IP Integrator},
select \emph{Create Block Design}.
%
\item {\bf Create Block Design}
%
\begin{itemize}
\item \emph{Design name}: \verb+system+
\item \emph{Directory}: \verb+<Local to Project>+
\item \emph{Specify Source set}: \verb+Design Sources+
\item Click \emph{OK}
\end{itemize}
%
\newpage
\item {\bf Add the Zynq Processor}
%
\begin{itemize}
\item Click the `+' in the middle of the design area to open the IP list
\item In the \emph{Search} bar enter \verb+Zynq+
\item Double-click on \verb+ZYNQ7 Processing System+ to add the Zynq to
the design
\item In the {\bf Block Properties} dialog, change the name from
\verb+processing_system7_0+ to \verb+zynq+.
\end{itemize}
%
\item {\bf Run Block Automation}
%
\begin{itemize}
\item Click on the blue hyperlink \textcolor{blue}{Run Block Automation}
\item Accept the block automation defaults by clicking \emph{OK}
\end{itemize}
%
\end{enumerate}
%
% -----------------------------------------------------------------------------
% PS bicolor LED GPIO MIO Design - Vivado Block Diagram
% -----------------------------------------------------------------------------
%
% Installed "A Ruler for Windows" and set the Diagram GUI detached
% window size to 5in x 10in for screen capture
%
\begin{figure}[t]
  \begin{minipage}{\textwidth}
    \begin{center}
    \includegraphics[width=0.8\textwidth]
    {figures/blinky_gpio_mio_vivado_diagram_a.png}\\
    (a) Zynq processor
    \end{center}
  \end{minipage}
  \vskip5mm
  \begin{minipage}{\textwidth}
    \begin{center}
    \includegraphics[width=0.8\textwidth]
    {figures/blinky_gpio_mio_vivado_diagram_b.png}\\
    (c) Zynq processor after block automation
    \end{center}
  \end{minipage}
  \caption{PS bicolor LED control via GPIO MIO design Vivado diagram window.}
  \label{fig:blinky_gpio_mio_vivado_diagram}
\end{figure}
% -----------------------------------------------------------------------------
%
Figure~\ref{fig:blinky_gpio_mio_vivado_diagram}(a) shows the diagram
window after adding the Zynq to the \verb+system+ design.
Figure~\ref{fig:blinky_gpio_mio_vivado_diagram}(b) shows the
diagram window after running block automation.
%
Block automation configures the Zynq processor using the board
presets, \verb+preset.xml+, which is part of the MiniZed
board definition files (BDF) located in the directory:
%
\begin{verbatim}
C:\software\Xilinx\Vivado\2019.1\data\boards\board_files\minized\1.2\
\end{verbatim}
%
This example uses the MiniZed Zynq processor defaults. Later design examples
customize the Zynq processor.
%
%------------------------------------------------------------------------------
\item \textcolor{blue}{\textbf{Vivado Synthesis}}
%------------------------------------------------------------------------------

The need for \emph{Synthesis} in a design that does not use the programmable
logic sounds redundant. However, the Zynq design flow uses Vivado to configure
the (required) Zynq processor system and the (optional) programmable logic.
%
Vivado generates the hardware definition file (HDF) used by the SDK. Vivado
was used to configure the Zynq with the MiniZed presets, so that the SDK can
be used to create the blinky LED application. The hardware support files needed
by the SDK are generated as follows.
%
\begin{enumerate}
%
\item {\bf Create HDL Wrapper}
%
\begin{itemize}
\item In the Vivado {\bf Sources} window, highlight \verb+system (system.bd)+
\item Right-mouse-click and select \emph{Create HDL Wrapper}
\item Accept the default option \emph{Let Vivado manage wrapper and auto-update}
\item Click \emph{OK}
\item This generates the top-level design file \verb+system_wrapper.v+
\end{itemize}
%
\item {\bf Constraints}

This design uses only Zynq processor system I/Os, no programmable logic I/Os,
so pin constraints are not required. The Zynq pins are configured by the
FSBL, which is generated by Vivado as part of the synthesis step.
%
\item {\bf Run Synthesis}
%
\begin{itemize}
\item In the Vivado {\bf Project Manager} click \emph{Run Synthesis}
\item Accept the defaults in the {\bf Launch Runs} dialog
\item Click \emph{OK}
\item Synthesis generates output in the directory

\verb+vwork/blinky_gpio_mio.runs+

\item When the {\bf Synthesis Completed} dialog appears, close it
by clicking \emph{Cancel}
\end{itemize}
%
\item {\bf Export Hardware}
%
\begin{itemize}
\item Select \emph{File}$\rightarrow$\emph{Export}$\rightarrow$\emph{Export Hardware}
\item Leave the \emph{Include bitstream} checkbox unchecked
\item Accept the default \emph{Export to:} \verb+<Local to Project>+
\item Click \emph{OK}
\item This step generates the HDF file used by the SDK
%
\begin{verbatim}
vwork/blinky_gpio_mio.sdk/system_wrapper.hdf
\end{verbatim}
\end{itemize}
%
\newpage
\item {\bf HDF Analysis} (optional step)
%
\begin{itemize}
\item Create a copy of the HDF file
\item Change the extension from \verb+.hdf+ to \verb+.zip+
\item Unzip the file and view the contents of \verb+system_wrapper+:
\begin{itemize}
\item \verb+ps7_init.tcl+

Used by the SDK to initialize the processor during JTAG download

\item \verb+ps7_init*.h/c+

First stage bootloader (FSBL) source.

\item \verb+system_bd.tcl+

Block design Tcl script.
\end{itemize}
\item Delete the zip file and the \verb+system_wrapper+ folder
\end{itemize}
%
\end{enumerate}
%
%------------------------------------------------------------------------------
\item \textcolor{blue}{\textbf{SDK Project Creation}}
%------------------------------------------------------------------------------
%
\begin{enumerate}
\item \textbf{Method 1}: Start the SDK tool from Vivado
%
\begin{itemize}
\item Select the Vivado menu option \emph{File}$\rightarrow$\emph{Launch SDK}
\item Accept the {\bf Launch SDK} dialog defaults and click \emph{OK}
\item This method sets up the SDK to use the \verb+.sdk+ folder within the
Vivado work folder, i.e.,

\verb+blinky_gpio_mio\vwork\blinky_gpio_mio.sdk+

\item The SDK unzips the \verb+.hdf+ archive into the folder

\verb+blinky_gpio_mio\vwork\blinky_gpio_mio.sdk\system_wrapper_hw_platform_0+

\item The SDK opens and displays the \textbf{Address Map for processor ps7\_cortexa9\_0},
(which is generated from the settings in \verb+system.hdf+).
Table~\ref{tab:blinky_gpio_mio} shows a selection of addresses
relevant to the GPIO MIO design.

\item Double-mouse-click on \verb+ps7_init.html+ to see a detailed view
of the address map, and the register settings. The registers are configured
when the SDK runs \texttt{ps7\_init.tcl} during JTAG download, or when the
processor boots and runs the FSBL.

\item Scroll down to \textbf{MIO 52} and \textbf{MIO 53} to see that the
multiplexed I/O (MIO) used to control the red and green PS LEDs are configured
as GPIO.

\item Scroll further down and see that the PS-to-PL clocks
\textbf{FPGA0, 1, 2, 3} are configured for frequencies of
50MHz, 100MHz, 10MHz, and 10MHz. The PS-to-PL clocks are not used
in the GPIO MIO example.
\end{itemize}

\item \textbf{Method 2}: Start the SDK tool from the Windows Start Menu (or from a Linux console)

The Avnet MiniZed Tutorial 02~\cite{Avnet_MiniZed_Tutorial02_2018} shows how
to start the SDK and import a Vivado hardware platform. The Avnet tutorial shows
the SDK steps that are automated by starting the SDK from within Vivado.
The manual method of creating the SDK is useful if you want to change the
SDK project names from the defaults used by Vivado.

\end{enumerate}
%
% -----------------------------------------------------------------------------
% Blinky GPIO MIO Address Map
% -----------------------------------------------------------------------------
%
\begin{table}[t]
\caption{PS bicolor LED GPIO MIO Example Address Map (selected addresses).}
\label{tab:blinky_gpio_mio}
\begin{center}
\begin{tabular}{|l|l|l|}
\hline
Zynq Component & Cell & Base Address\\
\hline\hline
\textbf{Memory} &&\\
\hline
&&\\
On-Chip Memory\#0                 & \texttt{ps7\_ram\_0}          & \texttt{0x00000000}\\
DDR Memory                        & \texttt{ps7\_ddr\_0}          & \texttt{0x00100000}\\
QSPI Linear (Read-Only) Addresses & \texttt{ps7\_qspi\_linear\_0} & \texttt{0xFC000000}\\
On-Chip Memory\#1                 & \texttt{ps7\_ram\_1}          & \texttt{0xFFFF0000}\\
&&\\
\hline
\textbf{Peripheral} &&\\
\hline
&&\\
UART\#0 Registers    & \texttt{ps7\_uart\_0}         & \texttt{0xE0000000}\\
UART\#1 Registers    & \texttt{ps7\_uart\_1}         & \texttt{0xE0001000}\\
GPIO MIO  Registers  & \texttt{ps7\_gpio\_0}         & \texttt{0xE000A000}\\
QSPI Registers       & \texttt{ps7\_qspi\_0}         & \texttt{0xE000D000}\\
&&\\
DDR Registers        & \texttt{ps7\_ddrc\_0}         & \texttt{0xF8006000}\\
&&\\
\hline
\end{tabular}
\end{center}
\end{table}
% -----------------------------------------------------------------------------

\newpage
%------------------------------------------------------------------------------
\item \textcolor{blue}{\textbf{SDK Board Support Package (BSP) Creation}}
%------------------------------------------------------------------------------
%
\begin{enumerate}
\item Select \emph{File}$\rightarrow$\emph{New}$\rightarrow$\emph{Board Support Package}
\item Accept the default settings for the project \texttt{standalone\_bsp\_0} and
click \emph{Finish}.
\item \textbf{Board Support Package Settings}

The MiniZed connects the Zynq UART\#0 to the bluetooth device and UART\#1 to the
FTDI USB UART interface. The board support package needs to be configured to use
UART\#1 for standard input/output as follows:

\begin{itemize}
\item Select \emph{Overview}, \emph{standalone}
\item Change the pull-downs for \texttt{stdin} and \texttt{stdout} to \texttt{ps7\_uart\_1}
\item Click \emph{OK}
\end{itemize}
\end{enumerate}

After the BSP has been generated, under \emph{Project Explorer}, expand
\texttt{standalone\_bsp\_0}, and double-mouse-click on \texttt{system.mss} to
see the documentation and example designs for the components in the system.

%------------------------------------------------------------------------------
\item \textcolor{blue}{\textbf{SDK Baremetal Application Creation}}
%------------------------------------------------------------------------------

The PS bicolor LED control application is created by starting with \emph{Hello World}
example application. The \emph{Hello World} application configures the SDK
project to use the \texttt{platform.h/c} files, which define the functions for
initialization and cleanup. The platform initialization configures the
UART interface to be used for C library standard input and output.

\begin{enumerate}
\item Select \emph{File}$\rightarrow$\emph{New}$\rightarrow$\emph{Application Project}
\item \textbf{Application Project}

Configure the project name, and change the project to the BSP just generated
\begin{itemize}
\item \emph{Project name:} \texttt{blinky\_gpio\_mio}
\item \emph{OS Platform:} \texttt{standalone}
\item \emph{Hardware Platform:} \texttt{system\_wrapper\_hw\_platform\_0}
\item \emph{Language:} \texttt{C}
\item \emph{Board Support Package: Use Existing} \texttt{standalone\_bsp\_0}
\end{itemize}
\item Click \emph{Next}
\item Select \emph{Hello World}

\item Click \emph{Finish}
\item Under the \emph{Project Explorer}, expand the \texttt{blinky\_gpio\_mio} project
\item Right-mouse-click on \texttt{src}, select \texttt{helloworld.c}, right-mouse-click,
select \emph{Rename} and rename the file \texttt{blinky\_gpio\_mio.c}
\item Replace \texttt{blinky\_gpio\_mio.c} with the source code shown in
Figure~\ref{fig:blinky_gpio_mio_app}.
\end{enumerate}

The SDK will automatically build the GPIO MIO bare-metal application.
%
%------------------------------------------------------------------------------
\item \textcolor{blue}{\textbf{SDK Download and Run the Application}}
%------------------------------------------------------------------------------
%
\begin{enumerate}
\item Set the MiniZed Flash/JTAG mode switch to JTAG mode
\item Connect the MiniZed to your development machine using the USB JTAG/UART interface
\item Select the \emph{SDK Terminal}, click the green + symbol, and
setup the UART communications to the MiniZed at 115200 baud (the timeout
can be set to 1 second)
\item Under the \emph{Project Explorer}, select the \texttt{blinky\_gpio\_mio}
project, right-mouse-click, and select \emph{Run As}$\rightarrow$\emph{Launch
on Hardware (System Debugger)} to download and run the application
\item The PS LED will cycle through off, green, red, and amber (red+green)
\item Select the \emph{SDK Terminal} and you will see a message
for each LED change
\end{enumerate}

\textcolor{magenta}{Congratulations!} You have run your first custom application on the MiniZed board.

%------------------------------------------------------------------------------
% blinky_gpio_mio bare-metal application code
%------------------------------------------------------------------------------
%
\begin{figure}
\begin{center}
\begin{minipage}{0.8\textwidth}
\begin{lstlisting}
#include <sleep.h>
#include "platform.h"
#include "xil_printf.h"

int main()
{
	int i = 0;

	init_platform();

	xil_printf("MiniZed Blinky GPIO MIO Example\r\n");
	xil_printf("-------------------------------\r\n");

	// Clear the data (DATA) bits
	*(volatile unsigned int *)0xE000A00C = 0xFFCF0000;

	// Set the direction (DIRM) to output
	*(volatile unsigned int *)0xE000A244 = 0x00300000;

	// Enable the outputs (OEN)
	*(volatile unsigned int *)0xE000A248 = 0x00300000;

	// Blink the bicolor LED
	while (1) {
		switch (i%4) {
			case 0:
				xil_printf("%d: Off\r\n", i);
				*(volatile unsigned int *)0xE000A00C = 0xFFCF0000;
				break;

			case 1:
				xil_printf("%d: Green\r\n", i);
				*(volatile unsigned int *)0xE000A00C = 0xFFCF0020;
				break;

			case 2:
				xil_printf("%d: Red\r\n", i);
				*(volatile unsigned int *)0xE000A00C = 0xFFCF0010;
				break;

			default:
				xil_printf("%d: Amber\r\n", i);
				*(volatile unsigned int *)0xE000A00C = 0xFFCF0030;
				break;
		}
		sleep(1);
		i++;
	}

    cleanup_platform();
    return 0;
}
\end{lstlisting}
\end{minipage}
\end{center}
\caption{PS bicolor LED GPIO MIO bare-metal application.}
\label{fig:blinky_gpio_mio_app}
\end{figure}
%------------------------------------------------------------------------------

%
%------------------------------------------------------------------------------
\item \textcolor{blue}{\textbf{SDK Create the First Stage Boot Loader (FSBL)}}
%------------------------------------------------------------------------------

The procedure to create the FSBL is similar to the procedure used to
create a new application, however, no user customization is required, as the
FSBL uses information (register settings) defined in the hardware definition
file exported from Vivado.

\begin{enumerate}
\item Select \emph{File}$\rightarrow$\emph{New}$\rightarrow$\emph{Application Project}
\item \textbf{Application Project}

Configure the project name, and change the project to the BSP just generated
\begin{itemize}
\item \emph{Project name:} \texttt{zynq\_fsbl}
\item \emph{OS Platform:} \texttt{standalone}
\item \emph{Hardware Platform:} \texttt{system\_wrapper\_hw\_platform\_0}
\item \emph{Language:} \texttt{C}
\item \emph{Board Support Package: Create New} \texttt{zynq\_fsbl\_bsp}

A new BSP is needed as the FSBL requires \verb+xilffs+ library support.
\end{itemize}
\item Click \emph{Next}
\item Select \emph{Zynq FSBL}

\item Click \emph{Finish}

\item Modify the BSP to use \verb+ps7_uart_1+ for stdin and stdout.

\end{enumerate}

The SDK will automatically build the FSBL BSP and FSBL. The SDK Console
output should contain the message \verb+Finished building target: zynq_fsbl.elf+.

%------------------------------------------------------------------------------
\item \textcolor{blue}{\textbf{SDK Create the QSPI Flash Boot Image}}
%------------------------------------------------------------------------------
%
\begin{enumerate}
\item In the SDK, highlight the \verb+blinky_gpio_mio+ application.
\item From the SDK main menu, select \emph{Xilinx}$\rightarrow$\emph{Create Boot Image}.

Figure~\ref{fig:blinky_gpio_mio_bootgen_gui} shows the Create Boot Image dialog.
The defaults will generate the file \verb+BOOT.bin+. The Avnet MiniZed Tutorial 04
indicates that the boot image file type should be changed to
.MCS~\cite{Avnet_MiniZed_Tutorial04_2018}, however, this is not necessary.

\item Click \emph{Create Image}
\end{enumerate}

%------------------------------------------------------------------------------
\item \textcolor{blue}{\textbf{Program the QPSI Flash Boot Image}}
%------------------------------------------------------------------------------

\begin{enumerate}
\item If Vivado is not running, start Vivado, and select \emph{Open Hardware Manager}
from within the main GUI.

If Vivado has the \verb+blinky_gpio_mio+ project open, then in the \emph{Flow Navigator},
under \emph{PROGRAM AND DEBUG}, select \emph{Open Hardware Manager}

\item Use the hardware manager to connect to the target (the MiniZed board).

\item From the Vivado main menu, select \emph{Tools}$\rightarrow$\emph{Add Configuration Memory Device}.

Note that this option becomes disabled (the menu option becomes greyed out) after you have added a
configuration memory device.

\item In the \emph{Add Configuration Memory} dialog:
\begin{itemize}
\item Enter the \textbf{Filter} settings:
\begin{itemize}
\item \emph{Manufacturer:} \texttt{Micron}
\item \emph{Density (Mb):} \texttt{128}
\item \emph{Type:} \texttt{qspi}
\item \emph{Width:} \texttt{x1-single}
\end{itemize}
%
\item From the \textbf{Select Configuration Memory Part} list, select the first device:

\texttt{my25ql128-qspi-x1-single}
%
\item Click \emph{Ok}
\end{itemize}

\item In the \emph{Program Configuration Memory Device} window:
\begin{itemize}
\item For \textbf{Configuration file:} navigate to the \verb+BOOT.bin+ file in the \verb+blinky_gpio_mio+ application folder.
\item For \textbf{Zynq FSBL:} navigate to the \verb+zynq_fsbl.elf+ file in the \verb+zynq_fsbl+ application folder.

Figure~\ref{fig:blinky_gpio_mio_program_flash_gui} shows the configuration window after
the files have been selected.
\item Click \emph{Ok} to program the QSPI Flash
\end{itemize}
%
\item Unplug the MiniZed, change the boot mode switch to Flash, and plug it back into the development machine.
\end{enumerate}

The MiniZed will boot from flash and the bicolor LED will start cycling off, green, red, and amber. The
blue LED will remain off, as the PS bicolor LED project does not use the PL, so the configuration DONE
signal never asserts.

\textcolor{magenta}{Congratulations!} Your MiniZed board boots and runs your first custom application.

The next design adds the progammable logic into the mix!

\end{enumerate} % End of blinky_gpio_mio steps
%
% -----------------------------------------------------------------------------
% PS bicolor LED GPIO MIO Design - Bootgen GUI
% -----------------------------------------------------------------------------
%
\begin{figure}[p]
  \begin{center}
    \includegraphics[width=0.9\textwidth]
    {figures/blinky_gpio_mio_bootgen_gui.png}\\
  \end{center}
  \caption{PS GPIO MIO design Create Boot Image window.}
  \label{fig:blinky_gpio_mio_bootgen_gui}
\end{figure}
% -----------------------------------------------------------------------------
%
% -----------------------------------------------------------------------------
% PS bicolor LED GPIO MIO Design - Program Flash GUI
% -----------------------------------------------------------------------------
%
\begin{figure}[p]
  \begin{center}
    \includegraphics[width=0.9\textwidth]
    {figures/blinky_gpio_mio_program_flash_gui.png}\\
  \end{center}
  \caption{PS GPIO MIO design Program Coniguration Memory Device window.}
  \label{fig:blinky_gpio_mio_program_flash_gui}
\end{figure}
% -----------------------------------------------------------------------------
%

