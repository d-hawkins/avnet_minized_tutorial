% =============================================================================
\section{Zynq Power-on-Reset and Boot}
% =============================================================================
\label{sec:zynq_boot}

The Zynq processor power-on-reset and boot procedure is documented in the
\emph{Technical Reference Manual} (TRM) and \emph{Software Developer
User Guide}~\cite{Xilinx_UG585_2021,Xilinx_UG821_2015}. The Zynq
\href{https://www.xilinx.com/support/documentation-navigation/design-hubs/dh0053-zynq-7000-boot-and-config-hub.html}{boot and configuration}
design hub contains additional cross-references on the power-on-reset
and boot flow.

The MiniZed board supports two boot modes; JTAG and QSPI Flash.
%
The MiniZed Zynq processor determines the boot mode based on the
\emph{boot mode pin settings} that are read after power is valid,
when power-on-reset (\verb+PS_POR_B+) deasserts\footnote{The Zynq TRM
shows the required timing relationship between the power-on-reset
and system reset signals in Figure 6-4: \emph{Power and Reset Sequencing
Waveforms} on p165~\cite{Xilinx_UG585_2021}.}.
%
Figure~\ref{fig:zynq_boot_mode_pins} shows how seven MIO pins control
the boot mode. Figure~\ref{fig:minized_boot_mode_pins} shows the
MiniZed connections for the boot mode pins.
Table~\ref{tab:minized_boot_mode_pins} summarizes the MiniZed boot mode
pin connections.

% -----------------------------------------------------------------------------
% MiniZed Boot Mode Pins
% -----------------------------------------------------------------------------
%
\begin{table}[h]
\caption{MiniZed boot mode MIO pin settings.}
\label{tab:minized_boot_mode_pins}
\begin{center}
\begin{tabular}{|c|c|p{100mm}|}
\hline
MIO Pin(s) & Logic & Description\\
\hline\hline
&&\\
8:7  & 0b & MIO Bank 1+0 voltages are both 3.3V (pp3-4~\cite{Avnet_MiniZed_Schematic_2017})\\
&&\\
6    & 0b & Execute BootROM after the PS clock PLLs lock\\
&&\\
5    & \textcolor{blue}{0b} or \textcolor{red}{1b} & JTAG/Flash boot mode switch\\
5:3  & \textcolor{blue}{0}00b & \hspace{1mm}\textcolor{blue}{$\bullet$ JTAG boot mode}\\
     & \textcolor{red}{1}00b  & \hspace{1mm}\textcolor{red}{$\bullet$ QSPI Flash boot mode}\\
&&\\
2    & 0b       & JTAG chain is cascaded (ARM processor and FPGA logic)\\
&&\\
\hline
\end{tabular}
\end{center}
\end{table}

% -----------------------------------------------------------------------------
% Zynq Boot Mode Pins
% -----------------------------------------------------------------------------
%
\begin{figure}[p]
  \begin{center}
    \includegraphics[width=0.95\textwidth]
    {figures/zynq_boot_mode_pins.png}
  \end{center}
  \caption{Zynq boot mode MIO pins (from Table 6-4, p167~\cite{Xilinx_UG585_2021}).}
  \label{fig:zynq_boot_mode_pins}
\end{figure}
% -----------------------------------------------------------------------------
% -----------------------------------------------------------------------------
% MiniZed Boot Mode Pins
% -----------------------------------------------------------------------------
%
\begin{figure}[p]
  \begin{center}
    \includegraphics[width=0.95\textwidth]
    {figures/minized_boot_mode_pins.png}
  \end{center}
  \caption{MiniZed Zynq boot mode MIO pins (from schematic p3~\cite{Avnet_MiniZed_Schematic_2017}).}
  \label{fig:minized_boot_mode_pins}
\end{figure}
% -----------------------------------------------------------------------------

The Zynq processor JTAG boot mode is useful during software development. The SDK can
be used to configure the programmable logic, configure the processor (using \verb+ps7_init.tcl+),
download the application image, and then run or debug the application.
%
The programmable logic is configuring by the SDK using the \emph{bitstream}
included in the hardware definition file exported from Vivado.

The MiniZed QSPI Flash boot mode flow is:
%
\begin{enumerate}
\item Power is valid and power-on-reset deasserts.
\item The hardware samples the boot strap pins (QSPI Flash mode)
and enables the PS clock PLLs.
\item When the PS clock PLLs lock, the PS starts executing the on-chip BootROM code.
\item The BootROM code reads the BootROM Header (Boot Image Header) from QSPI Flash.
\item The BootROM code loads the First Stage Boot Loader (FSBL) into On-chip memory (OCM).
\item The BootROM code starts executing the FSBL (ending the BootROM execution).
\item (Optional) The FSBL code configures the programmable logic.
\item The FSBL code determines what happens next, eg., on the MiniZed it
typically loads and executes the second stage boot loader (SSBL), U-Boot.
\item The U-Boot code loads and executes Linux.
\end{enumerate}
%
The Zynq processor memory map during BootROM execution, and after handoff to the
FSBL, is given in Figure 6-11~\cite{Xilinx_UG585_2021}: the main change in the
memory map is the presence of the 64kB BootROM (at address 0x00030000) after the
192kB OCM (at address 0) during BootROM execution.

% -----------------------------------------------------------------------------
% Zynq Boot Image Header
% -----------------------------------------------------------------------------
%
\begin{figure}[p]
  \begin{center}
    \includegraphics[width=\textwidth]
    {figures/zynq_bootrom_header.png}
  \end{center}
  \caption{Zynq BootROM Image Header (from Table 6-5, pp172-173~\cite{Xilinx_UG585_2021}).}
  \label{fig:zynq_bootrom_header}
\end{figure}
% -----------------------------------------------------------------------------
%
Figure~\ref{fig:zynq_bootrom_header} shows the format of the BootROM Header
(Boot Image Header) in the QSPI flash boot image. The BootROM header is defined
in the Zynq \emph{Technical Reference Manual} (TRM)~\cite{Xilinx_UG585_2021},
while boot image creation using the \verb+bootgen+ tool is described in the
\emph{Software Developers Guide} (pp48-51, and Appendix A~\cite{Xilinx_UG821_2015}).
%
The format of the MiniZed QSPI Flash boot image is typically:
%
\begin{itemize}
\item BootROM Header (Boot Image Header)
\item First Stage Boot Loader (FSBL)
\item Programmable Logic bitstream
\item U-Boot (the second stage bootloader)
\item Additional partitions, eg., Linux and the device tree.
\end{itemize}
%
Insight into the BootROM Header can be gained by decoding an example.
The BootROM header for the MiniZed \verb+flash_fallback_7007S.bin+
image can be viewed using U-Boot via:
%
\begin{verbatim}
Zynq> sf probe
SF: Detected n25q128 with page size 256 Bytes, erase size 64 KiB, total 16 MiB
Zynq> sf read 10000000 0 1000000
device 0 whole chip
SF: 16777216 bytes @ 0x0 Read: OK
Zynq> crc32 10000000 fc0be4
crc32 for 10000000 ... 10fc0be3 ==> 005b7b3c
Zynq> md 10000000 30
10000000: eafffffe eafffffe eafffffe eafffffe    ................
10000010: eafffffe eafffffe eafffffe eafffffe    ................
10000020: aa995566 584c4e58 00000000 01010000    fU..XNLX........
10000030: 00001700 00018008 00000000 00000000    ................
10000040: 00018008 00000001 fc164530 00000000    ........0E......
10000050: 00000000 00000000 00000000 00000000    ................
10000060: 00000000 00000000 00000000 00000000    ................
10000070: 00000000 00000000 00000000 00000000    ................
10000080: 00000000 00000000 00000000 00000000    ................
10000090: 00000000 00000000 000008c0 00000c80    ................
100000a0: ffffffff 00000000 ffffffff 00000000    ................
100000b0: ffffffff 00000000 ffffffff 00000000    ................
Zynq>
\end{verbatim}
%
The U-Boot commands copy the QSPI flash to DDR memory, calculates the
image checksum (to confirm it matches the value in
Table~\ref{tab:factory_image_checksums}),
and displays the first $48\times32$-bit words (192-bytes) of the
Boot Image Header. The image details are:
%
\begin{enumerate}
\item \texttt{0x000:} \textbf{Interrupt Table for Execution-in-Place}

The interrupt table is the first $8\times32$-bit words of the header.
The 32-bit value 0xEAFFFFFE corresponds to an ARMv7 branch-to-self
instruction~\cite{ARM_ARMv7_2018}, so each interupt handler will
\emph{lock-up} the processor.

\item \texttt{0x020:} \textbf{Width Detection}

0xAA995566 is used for width detection (p173~\cite{Xilinx_UG585_2021}).

\item \texttt{0x024:} \textbf{Image Identification}

0x584C4E58 corresponds to 8-bit ASCII codes for
X, L, N, and X in little-endian format, which is why the ASCII
value displayed by U-Boot is XNLX  (p173~\cite{Xilinx_UG585_2021}).

\item \texttt{0x028:} \textbf{Encryption Status}

0x00000000 indicates the image is not encrypted.

\item \texttt{0x02C:} \textbf{FSBL/User Defined}

0x01010000 indicates the BootROM header version (p174~\cite{Xilinx_UG585_2021}).

\item \texttt{0x030:} \textbf{Source Offset}

0x00001700 offset in bytes to the FSBL image. The flash contents
near this offset are:
%
\begin{verbatim}
Zynq> md 100016e0 10
100016e0: ffffffff ffffffff ffffffff ffffffff    ................
100016f0: ffffffff ffffffff ffffffff ffffffff    ................
10001700: ea000049 ea000025 ea00002b ea00003b    I...%...+...;...
10001710: ea000032 e320f000 ea000000 ea00000f    2..... .........
\end{verbatim}
%
The locations prior to offset 0x1700 are blank, and the data starting
at offset 0x1700 look like ARMv7 instructions.

\item \texttt{0x034:} \textbf{Length of Image}

0x00018008 the image length is 96kbytes plus 8-bytes.

\item \texttt{0x038:} \textbf{FSBL Load Address}

0x00000000 load to the start of OCM.

\item \texttt{0x03C:} \textbf{Start of Execution}

0x00000000 execute from the start of OCM.

\item \texttt{0x040:} \textbf{Total Image Length}

0x00018008 is the same as \texttt{0x034:} \textbf{Length of Image}.

For non-secure boot mode, 0x040 and 0x034 must match
(p175~\cite{Xilinx_UG585_2021}).

\item \texttt{0x044:} \textbf{QSPI Configuration Word}

0x00000001 matches expected value (p175~\cite{Xilinx_UG585_2021}).

\item \texttt{0x048:} \textbf{Header Checksum}

0xFC164530 matches the inverted sum of 32-bit words 0x020 to 0x044,
i.e., $\sim$((0xAA995566 + 0x584C4E58 + 0x01010000 + 0x00001700 + 0x00018008
+ 0x00018008 + 0x00000001) \& 0xFFFFFFFF).

\item \texttt{0x04C:} \textbf{FSBL/User Defined} (76-bytes)

Bytes 0x04C to 0x097 are all zero (unused). These bytes can be defined
using \verb+bootgen+ and the \verb+udf_field+ in the BIF file
(see Appendix A~\cite{Xilinx_UG821_2015}).

\item \texttt{0x098:} \textbf{Image Header Table Offset}

0x000008C0 Image Header Table offset.

\textcolor{red}{Note: this field is missing from Table 6-5, but is listed in
the text, with the wrong name!} (see p175~\cite{Xilinx_UG585_2021}).

\item \texttt{0x09C:} \textbf{Partition Header Table Offset}

0x00000C80 Partition Header Table offset.

\textcolor{red}{Note: this field is missing from Table 6-5, but is listed in
the text, with the wrong description!} (see p176~\cite{Xilinx_UG585_2021}).

\item \texttt{0x0A0:} \textbf{Register Initialization} (2048-bytes)

The repeated pattern of 0xFFFFFFFF 0x00000000 indicates that
the register initialization block is not used (the register
address 0xFFFFFFFF terminates register initialization
(p176~\cite{Xilinx_UG585_2021}).

\item \texttt{0x8A0:} \textbf{Image Header}

Table 6-5 refers to this as the Image Header area, however, the text on
p176 indicates this is FSBL/User Defined. The Image Header Table offset
points to 0x8C0, not 0x8A0. Viewing the copy of the QSPI flash image in
DDR using U-Boot gives:
%
\begin{verbatim}
Zynq> md 100008a0 58
100008a0: ffffffff ffffffff ffffffff ffffffff    ................
100008b0: ffffffff ffffffff ffffffff ffffffff    ................
100008c0: 01020000 00000004 00000320 00000240    ........ ...@...
100008d0: 00000000 ffffffff ffffffff ffffffff    ................
100008e0: ffffffff ffffffff ffffffff ffffffff    ................
100008f0: ffffffff ffffffff ffffffff ffffffff    ................
10000900: 00000250 00000320 00000000 00000001    P... ...........
10000910: 7a796e71 5f667362 6c2e656c 66000000    qnyzbsf_le.l...f
10000920: 00000000 ffffffff ffffffff ffffffff    ................
10000930: ffffffff ffffffff ffffffff ffffffff    ................
10000940: 00000260 00000330 00000000 00000001    `...0...........
10000950: 64657369 676e5f31 5f777261 70706572    ised1_ngarw_repp
10000960: 2e626974 00000000 00000000 ffffffff    tib.............
10000970: ffffffff ffffffff ffffffff ffffffff    ................
10000980: 00000270 00000340 00000000 00000001    p...@...........
10000990: 752d626f 6f742e65 6c660000 00000000    ob-ue.to..fl....
100009a0: ffffffff ffffffff ffffffff ffffffff    ................
100009b0: ffffffff ffffffff ffffffff ffffffff    ................
100009c0: 00000000 00000350 00000000 00000001    ....P...........
100009d0: 696d6167 652e7562 00000000 00000000    gamibu.e........
100009e0: ffffffff ffffffff ffffffff ffffffff    ................
100009f0: ffffffff ffffffff ffffffff ffffffff    ................
\end{verbatim}
%
This shows that the bytes at offset 0x8A0 are unused, and the bytes
at 0x8C0 are used.
The \textbf{Image Header Table Offset} 0x098 indicates there is where
the image header table is located. The format of the image
header table is defined in Table A-4, p63~\cite{Xilinx_UG821_2015}.
The first word matches the expected value of 0x01020000. The
second word indicates there are 4 image headers. The third word
indicates the partition header is at word offset 0x320, which is
byte offset 0xC80, and that value matches the value for the
\textbf{Partition Header Table Offset} 0x09C. The forth word
indicates the first image header is at word offset 0x240, which
is byte offset 0x900. The fifth word 0x00000000 is unused, and
the values 0xFFFFFFFF pad the header table to the next 64-byte
(0x40) boundary.

\item \texttt{0x8C0:} \textbf{Partition Header}

Table 6-5 refers to this as the Partion Header area, however, the
decoding of the header indicates there is where the Image Header Table
is located. The image header was decoded in the previous step.

\item \texttt{0x900:} \textbf{FSBL Image or User Code}

The Image Header Table indicates that this is the location of the
first of the array of image headers. The Image Header format is
defined in in Table A-7, p66~\cite{Xilinx_UG821_2015}.
%
Decoding the strings for the four image names gives;
\verb+zynq_fsbl.elf+, \verb+design_1_wrapper.bit+, \verb+u-boot.elf+,
and \verb+image.ub+. This indicates that the QSPI flash fallback
image contains the FSBL, a bitfile, U-boot, and a Linux image,
i.e., exactly what we would expect to find in the fallback image.

\end{enumerate}
%
The manual decoding of the MiniZed QSPI flash fallback image has exposed
some minor inconsistencies in the Xilinx documentation, however, the
documentation provides sufficient information to interpret a boot image.

