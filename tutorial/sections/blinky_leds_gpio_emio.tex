%------------------------------------------------------------------------------
\subsection{PL bicolor and blue LED control via GPIO EMIO}
%------------------------------------------------------------------------------

The PL bicolor and blue LED control design extends the GPIO MIO design:
%
\begin{itemize}
\item 16-bits of PS GPIO EMIO are connected to the PL.
\item The 7-series \verb+STARTUPE2+ component is instantiated in the PL
to control the blue LED.
\item The PL is used to connect GPIO EMIO outputs to the
bicolor and blue LED.
\item The PL is used to connect GPIO EMIO inputs to the user
push button and switch.
\item The Vivado design uses a top-level file that is \emph{not}
managed by Vivado.
\item The Vivado block design is instantiated in the top-level design.
\item An unmanaged constraints Tcl file is used to implement pin constraints.
\end{itemize}

\noindent\textcolor{red}{\bf TODO}
\begin{itemize}
\item I have already extended the GPIO MIO design and confirmed that EMIO can control the PL bicolor LED.
\item The post-block-automation Zynq processor has 16-bits of EMIO support. All I needed to do was
to select the GPIO\_0 port, right-mouse-click and select \emph{Make External} and a port named
\texttt{GPIO\_0\_0} is created. I renamed it \texttt{GPIO}.
\item Extend the design to support the blue LED, the user switch, and push button.
\item EMIO registers:
 - DATA\_2     at E000A048  Output Data, GPIO Bank 2, EMIO
 - DATA\_2\_RO at E000A068  Input Data, GPIO Bank 2, EMIO
 - DIRM\_2     at E000A284  Direction Mode, GPIO Bank 2, EMIO
 - OEN\_2      at E000A288  Output Enable, GPIO Bank 2, EMIO
\item Describe how to use U-Boot to control the EMIO LED?
\end{itemize}


\noindent\textcolor{red}{\bf Tutorial walk-through notes}

Follow through the first tutorial and just make notes about the changes.

\begin{itemize}
\item Vivado: Create a new project \texttt{blinky\_gpio\_emio}
\item Zynq PS: After running block automation, select
GPIO\_0, \emph{Make External}, and rename to \texttt{GPIO}
\item At this point we have a top-level design that we
do not want to generate a wrapper for, rather we want a
top-level file in which we can drop the \texttt{system} design as
a component.
\item Figure XXX shows a top-level HDL component created
based on the port definitions on

\verb+vwork\blinky_gpio_emio.srcs\sources_1\bd\system\synth\system.v+

\item Add \texttt{minized.sv} to the project
\item Add the pin constraints Tcl file to the project and set it
to be used for \emph{Implementation} only.
\item Run Synthesis. No issues, so Run Implementation.
No issues, so Generate Bitstream.
\item Check the arduino pins. A3 should be pin E13 and A4 should be pin E12.

\begin{verbatim}
| E11        | arduino_a[5]      | High Range | IO_L2P_T0_AD8P_35       | BIDIR       | LVCMOS33           |      35 |          8 | SLOW |                     |                 NONE |         | FIXED      |           |          |      | NONE             |              |                   |              |
| E12        | arduino_a[4]      | High Range | IO_L2N_T0_AD8N_35       | BIDIR       | LVCMOS33           |      35 |          8 | SLOW |                     |                 NONE |         | FIXED      |           |          |      | NONE             |              |                   |              |
| E13        | arduino_a[3]      | High Range | IO_L1N_T0_AD0N_35       | BIDIR       | LVCMOS33           |      35 |          8 | SLOW |                     |                 NONE |         | FIXED      |           |          |      | NONE             |              |                   |              |
| F12        | arduino_a[2]      | High Range | IO_L1P_T0_AD0P_35       | BIDIR       | LVCMOS33           |      35 |          8 | SLOW |                     |                 NONE |         | FIXED      |           |          |      | NONE             |              |                   |              |
| F13        | arduino_a[1]      | High Range | IO_L3P_T0_DQS_AD1P_35   | BIDIR       | LVCMOS33           |      35 |          8 | SLOW |                     |                 NONE |         | FIXED      |           |          |      | NONE             |              |                   |              |
| F14        | arduino_a[0]      | High Range | IO_L3N_T0_DQS_AD1N_35   | BIDIR       | LVCMOS33           |      35 |          8 | SLOW |                     |                 NONE |         | FIXED      |           |          |      | NONE             |              |                   |              |
\end{verbatim}

\item Export hardware and include the bitstream.
\item Launch SDK, setup the BSP, and the application.
\item Since this design has a bitstream, the FPGA needs to be programmed first,
and then the application downloaded.

\item Changing the LEDs using the EMIO should just need DIRM = 1 at offset 284.
Yes. That and the data output were all that were needed.
Ok, now I have enough to write up this tutorial step.
\end{itemize}

PL bi-color LED on pin E12 PL\_LED\_R, E13 and PL\_LED\_G. Not sure about the names,
since they're also on Arduino pins.


