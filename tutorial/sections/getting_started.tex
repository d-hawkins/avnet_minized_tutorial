% =============================================================================
\section{MiniZed Getting Started (Out-of-the-Box)}
% =============================================================================

% -----------------------------------------------------------------------------
% MiniZed Quick Start Peripherals Diagram
% -----------------------------------------------------------------------------
%
\begin{figure}[t]
  \begin{center}
    \includegraphics[width=\textwidth]
    {figures/minized_quick_start_diagram.png}
  \end{center}
  \caption{MiniZed peripherals diagram (from the Quick Start Card~\cite{Avnet_MiniZed_QSC_2017}).}
  \label{fig:minized_quick_start_diagram}
\end{figure}
% -----------------------------------------------------------------------------

Figure~\ref{fig:minized_quick_start_diagram} shows a photo of the MiniZed
board from the Avnet \emph{Quick Start Card}~\cite{Avnet_MiniZed_QSC_2017}.
The Avnet MiniZed \emph{Quick Start Card} and \emph{Getting Started
Guide}~\cite{Avnet_MiniZed_GSG_2018} contain instructions
on how to start using the MiniZed. This section assumes that you have
gone through both of these documents and are ready to learn more
about how the MiniZed works.

The MiniZed is powered via the USB JTAG/UART shown in
Figure~\ref{fig:minized_quick_start_diagram} (on the left side).
The USB JTAG/UART interface is a Digilent design\footnote{Digilent provides
development boards to the Xilinx University Program, and is part of National Instruments.}.
The USB JTAG/UART circuit uses the FTDI FT2232H dual USB-to-UART/FIFO
controller: with the JTAG interface implemented on the first channel,
and the UART implemented on the second channel. The MiniZed schematic
shows the circuit on page 6~\cite{Avnet_MiniZed_Schematic_2017}.
The circuit shows that the FTDI interface can also be used to toggle the
MiniZed (power manager) Power-on-Reset (POR) and (Zynq) System Reset
(SRST)\footnote{The MiniZed hardware user guide contains a block diagram of
the reset sources in Figure 7 on page 22~\cite{Avnet_MiniZed_HW_2020}.}.

The MiniZed boots from QSPI flash, loads the Xilinx-generated
First Stage Boot Loader (FSBL), which configures the FPGA programmable
logic (causing the blue LED to turn on) and loads U-Boot
(the Second State Boot Loader), and U-Boot then loads Linux from
the eMMC card. The designs in this tutorial use both the QSPI and eMMC
flash. Appendix~\ref{sec:factory_restore} shows how to view
the MiniZed U-Boot and Linux versions, and how to restore the
board to factory condition. Restoring the board to factory condition
allows the next user to begin using the MiniZed from the out-of-the-box
starting point.

