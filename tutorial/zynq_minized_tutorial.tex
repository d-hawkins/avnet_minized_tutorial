% -----------------------------------------------------------------------------
% zynq_minized_tutorial.tex
%
% 6/1/2018 D. W. Hawkins (dwh@caltech.edu)
%
% Avnet Zynq Minized Tutorial.
%
% Latex processing performed using Windows MiKTeK 2.9.
%
% -----------------------------------------------------------------------------
% Document preamble
% -----------------------------------------------------------------------------
%
\documentclass[10pt,twoside]{article}

% Math symbols
\usepackage{amsmath}
\usepackage{amssymb}

% Headers/Footers
\usepackage{fancyhdr}

% Colors
\usepackage{color}
\usepackage{xcolor}
\definecolor{darkgreen}{rgb}{0,0.4,0}

% Importing and manipulating graphics
\usepackage{graphicx}
\usepackage{subfig}
\usepackage{pdflscape}

% Misc packages
\usepackage{verbatim}
\usepackage{dcolumn}
\usepackage{ifpdf}
\usepackage{enumerate}

% PDF Bookmarks and hyperref stuff
\usepackage[
  bookmarks=true,
  bookmarksnumbered=true,
  colorlinks=true,
  filecolor=blue,
  linkcolor=blue,
  urlcolor=blue,
  hyperfootnotes=true
  citecolor=blue
]{hyperref}

% Improved citation handling
% (include after the hyperref stuff)
\usepackage{cite}

% Pretty printing code
\usepackage{listings}

% C code syntax highlighting
\lstset{
	language=C,
	basicstyle=\small\ttfamily,
	keywordstyle=\color{blue}\ttfamily,
	stringstyle=\color{red}\ttfamily,
	commentstyle=\color{darkgreen}\ttfamily,
	tabsize=4,
	showstringspaces=false
}

% -----------------------------------------------------------------------------
% Setup the margins
% -----------------------------------------------------------------------------
% Footer Template

% Set left margin - The default is 1 inch, so the following
% command sets a 1.25-inch left margin.
\setlength{\oddsidemargin}{0.25in}
\setlength{\evensidemargin}{0.25in}

% Set width of the text - What is left will be the right
% margin. In this case, right margin is
% 8.5in - 1.25in - 6in = 1.25in.
\setlength{\textwidth}{6in}

% Set top margin - The default is 1 inch, so the following
% command sets a 0.75-inch top margin.
%\setlength{\topmargin}{-0.25in}
\setlength{\topmargin}{-0.2in}

% Set height of the header
\setlength{\headheight}{0.3in}

% Set vertical distance between the header and the text
\setlength{\headsep}{0.2in}

% Set height of the text
\setlength{\textheight}{8.5in}

% Set vertical distance between the text and the
% bottom of footer
\setlength{\footskip}{0.4in}

% -----------------------------------------------------------------------------
% Allow floats to take up more space on a page.
% -----------------------------------------------------------------------------

% see page 142 of the Companion for this stuff and the
% documentation for the fancyhdr package
\renewcommand{\textfraction}{0.05}
\renewcommand{\topfraction}{0.95}
\renewcommand{\bottomfraction}{0.95}
% dont make this too small
\renewcommand{\floatpagefraction}{0.35}
\setcounter{totalnumber}{5}

% Make sure the top/bottom rules appear on the page
\renewcommand{\headrulewidth}{0.4pt}
\renewcommand{\footrulewidth}{0.4pt}

% =============================================================================
% Document contents
% =============================================================================
%
\begin{document}
\title{Xilinx Zynq-7000 Avnet MiniZed Tutorial}
\author{D. W. Hawkins (dwh@caltech.edu)\\\textcolor{red}{Version 0.4}}
\date{\today}

% Title page
\maketitle

% Table of contents
\tableofcontents

% Switch to the fancy page style
\pagestyle{fancy}

% Start the first section on an odd page
\cleardoublepage
%\clearpage

% =============================================================================
% Main Document
% =============================================================================
%
% The \input{} command inlines the latex file referenced.
% This means that figure references are relative to this
% top-level directory, i.e., figures/*, not ../figures/*
%
% =============================================================================
\section{Introduction}
% =============================================================================

The \href{http://www.minized.org}{MiniZed} is a low-cost (US\$89) development
board from Avnet containing a Xilinx Zynq-7000 programmable system-on-chip.
%
Figure~\ref{fig:minized_schematic_block_diagram} shows a block diagram of
the MiniZed board. The MiniZed
features~\cite{Avnet_MiniZed_HW_2017,Avnet_MiniZed_Schematic_2017}:
%
\begin{itemize}
% ---------------------
\item Xilinx Zynq-7000
% ---------------------
%
\begin{itemize}
\item Zynq 7Z007S single-core ARM Cortex-A9 processor (XC7Z007S-1CLG225C)
\item 28nm Artix-7 series programmable logic
\begin{itemize}
\item 3,600 logic slices containing~\cite{Xilinx_UG585_2018}
\begin{itemize}
\item 14,400 6-input LUTs
\item 28,800 registers
\end{itemize}
\item 50 Block RAMs (BRAMs)
\item 66 DSP Blocks (DSP48E1 with pre-adder, 25$\times$18 multiplier, 48-bit accumulator)
\end{itemize}
\end{itemize}
%
% ---------------------
\item Memory
% ---------------------
%
\begin{itemize}
\item Micron 512MB DDR3L with 16-bit data bus (MT41K256M16TW-107:P)
\item Micron 128Mb (16MB) QSPI Flash (MT25QL128ABA8E12-1SIT)
\item Micron 8GB eMMC Flash (MTFC8GAKAJCN-4M IT)
\end{itemize}
%
% ---------------------
\item Interface I/O
% ---------------------
%
\begin{itemize}
\item 33.3333MHz PS reference clock (Microchip DSC1001DI1-033.3333T)
\item 1 $\times$ PS bi-color LED
\item 1 $\times$ PL bi-color LED
\item Reset push button
\item User push button
\item User switch
\end{itemize}
%
% ---------------------
\item Communications
% ---------------------
%
\begin{itemize}
\item Murata ``Type 1DX'' (LBEE5KL1DX-883) wireless module with
\begin{itemize}
\item WiFi 802.11b/g/n
\item Bluetooth 4.1 plus EDR (Enhanced Data Rate) and BLE (Bluetooth Low Energy)
\end{itemize}
\item USB 2.0 Host Controller (Microchip USB3320C ULPI PHY)
\item 2 $\times$ PMod Connectors (16 GPIO)
\item Arduino-compatible shield interface (22 GPIO)
\item Digilent USB JTAG plus USB COM port interface (based on the FTDI FT2232H)
\end{itemize}
%
% ---------------------
\item Sensors
% ---------------------
%
\begin{itemize}
\item ST Micro LIS2DS12 Motion and Temperature sensor
\item ST Micro MP34DT05 MEMS Microphone
\end{itemize}
%
% ---------------------
\item Power
% ---------------------
%
\begin{itemize}
\item Dialog DA9062 Power Management IC
\item Auxiliary microUSB connector for USB host-mode power
\end{itemize}
%
\end{itemize}
%
This tutorial is written for FPGA designers new to the Zynq processor. The
tutorial provides cross-references to Avnet and Xilinx documentation, walks
through several designs to familiarize the user with the Xilinx development tools,
Zynq processor configuration, and programmable logic synthesis.

% -----------------------------------------------------------------------------
% MiniZed Block Diagram from the schematic
% -----------------------------------------------------------------------------
%
% The MiniZed Hardware Guide has a similar block diagram (perhaps newer?).
% The 2 page product brief has the same block diagram as the HW guide.
%
\begin{landscape}
\begin{figure}
  \begin{center}
    \includegraphics[width=210mm]
    {figures/minized_schematic_block_diagram.png}
  \end{center}
  \caption{MiniZed block diagram (from the schematic~\cite{Avnet_MiniZed_Schematic_2017}).}
  \label{fig:minized_schematic_block_diagram}
\end{figure}
\end{landscape}
% -----------------------------------------------------------------------------

\clearpage
% =============================================================================
\section{MiniZed Getting Started (Out-of-the-Box)}
% =============================================================================

% -----------------------------------------------------------------------------
% MiniZed Quick Start Peripherals Diagram
% -----------------------------------------------------------------------------
%
\begin{figure}[t]
  \begin{center}
    \includegraphics[width=\textwidth]
    {figures/minized_quick_start_diagram.png}
  \end{center}
  \caption{MiniZed peripherals diagram (from the Quick Start Card~\cite{Avnet_MiniZed_QSC_2017}).}
  \label{fig:minized_quick_start_diagram}
\end{figure}
% -----------------------------------------------------------------------------

Figure~\ref{fig:minized_quick_start_diagram} shows a photo of the MiniZed
board from the Avnet \emph{Quick Start Card}~\cite{Avnet_MiniZed_QSC_2017}.
The Avnet MiniZed \emph{Quick Start Card} and \emph{Getting Started
Guide}~\cite{Avnet_MiniZed_GSG_2018} contain instructions
on how to start using the MiniZed. This section assumes that you have
gone through both of these documents and are ready to learn more
about how the MiniZed works.

The MiniZed is powered via the USB JTAG/UART shown in
Figure~\ref{fig:minized_quick_start_diagram} (on the left side).
The USB JTAG/UART interface is a Digilent design\footnote{Digilent provides
development boards to the Xilinx University Program, and is part of National Instruments.}.
The USB JTAG/UART circuit uses the FTDI FT2232H dual USB-to-UART/FIFO
controller: with the JTAG interface implemented on the first channel,
and the UART implemented on the second channel. The MiniZed schematic
shows the circuit on page 6~\cite{Avnet_MiniZed_Schematic_2017}.
The circuit shows that the FTDI interface can also be used to toggle the
MiniZed (power manager) Power-on-Reset (POR) and (Zynq) System Reset
(SRST)\footnote{The MiniZed hardware user guide contains a block diagram of
the reset sources in Figure 7 on page 22~\cite{Avnet_MiniZed_HW_2017}.}.

The MiniZed boots from QSPI flash, loads the Xilinx-generated
First Stage Boot Loader (FSBL), which configures the FPGA programmable
logic (causing the blue LED to turn on) and loads U-Boot
(the Second State Boot Loader), and U-Boot then loads Linux from
the eMMC card. The designs in this tutorial use both the QSPI and eMMC
flash. Appendix~\ref{sec:factory_restore} shows how to view
the MiniZed U-Boot and Linux versions, and how to restore the
board to factory condition. Restoring the board to factory condition
allows the next user to begin using the MiniZed from the out-of-the-box
starting point.


\clearpage
\clearpage
% =============================================================================
\section{Zynq Power-on-Reset and Boot}
% =============================================================================
\label{sec:zynq_boot}

The Zynq processor power-on-reset and boot procedure is documented in the
\emph{Technical Reference Manual} (TRM) and \emph{Software Developer
User Guide}~\cite{Xilinx_UG585_2021,Xilinx_UG821_2015}. The Zynq
\href{https://www.xilinx.com/support/documentation-navigation/design-hubs/dh0053-zynq-7000-boot-and-config-hub.html}{boot and configuration}
design hub contains additional cross-references on the power-on-reset
and boot flow.

The MiniZed board supports two boot modes; JTAG and QSPI Flash.
%
The MiniZed Zynq processor determines the boot mode based on the
\emph{boot mode pin settings} that are read after power is valid,
when power-on-reset (\verb+PS_POR_B+) deasserts\footnote{The Zynq TRM
shows the required timing relationship between the power-on-reset
and system reset signals in Figure 6-4: \emph{Power and Reset Sequencing
Waveforms} on p165~\cite{Xilinx_UG585_2021}.}.
%
Figure~\ref{fig:zynq_boot_mode_pins} shows how seven MIO pins control
the boot mode. Figure~\ref{fig:minized_boot_mode_pins} shows the
MiniZed connections for the boot mode pins.
Table~\ref{tab:minized_boot_mode_pins} summarizes the MiniZed boot mode
pin connections.

% -----------------------------------------------------------------------------
% MiniZed Boot Mode Pins
% -----------------------------------------------------------------------------
%
\begin{table}[h]
\caption{MiniZed boot mode MIO pin settings.}
\label{tab:minized_boot_mode_pins}
\begin{center}
\begin{tabular}{|c|c|p{100mm}|}
\hline
MIO Pin(s) & Logic & Description\\
\hline\hline
&&\\
8:7  & 0b & MIO Bank 1+0 voltages are both 3.3V (pp3-4~\cite{Avnet_MiniZed_Schematic_2017})\\
&&\\
6    & 0b & Execute BootROM after the PS clock PLLs lock\\
&&\\
5    & \textcolor{blue}{0b} or \textcolor{red}{1b} & JTAG/Flash boot mode switch\\
5:3  & \textcolor{blue}{0}00b & \hspace{1mm}\textcolor{blue}{$\bullet$ JTAG boot mode}\\
     & \textcolor{red}{1}00b  & \hspace{1mm}\textcolor{red}{$\bullet$ QSPI Flash boot mode}\\
&&\\
2    & 0b       & JTAG chain is cascaded (ARM processor and FPGA logic)\\
&&\\
\hline
\end{tabular}
\end{center}
\end{table}

% -----------------------------------------------------------------------------
% Zynq Boot Mode Pins
% -----------------------------------------------------------------------------
%
\begin{figure}[p]
  \begin{center}
    \includegraphics[width=0.95\textwidth]
    {figures/zynq_boot_mode_pins.png}
  \end{center}
  \caption{Zynq boot mode MIO pins (from Table 6-4, p167~\cite{Xilinx_UG585_2021}).}
  \label{fig:zynq_boot_mode_pins}
\end{figure}
% -----------------------------------------------------------------------------
% -----------------------------------------------------------------------------
% MiniZed Boot Mode Pins
% -----------------------------------------------------------------------------
%
\begin{figure}[p]
  \begin{center}
    \includegraphics[width=0.95\textwidth]
    {figures/minized_boot_mode_pins.png}
  \end{center}
  \caption{MiniZed Zynq boot mode MIO pins (from schematic p3~\cite{Avnet_MiniZed_Schematic_2017}).}
  \label{fig:minized_boot_mode_pins}
\end{figure}
% -----------------------------------------------------------------------------

The Zynq processor JTAG boot mode is useful during software development. The SDK can
be used to configure the programmable logic, configure the processor (using \verb+ps7_init.tcl+),
download the application image, and then run or debug the application.
%
The programmable logic is configuring by the SDK using the \emph{bitstream}
included in the hardware definition file exported from Vivado.

The MiniZed QSPI Flash boot mode flow is:
%
\begin{enumerate}
\item Power is valid and power-on-reset deasserts.
\item The hardware samples the boot strap pins (QSPI Flash mode)
and enables the PS clock PLLs.
\item When the PS clock PLLs lock, the PS starts executing the on-chip BootROM code.
\item The BootROM code reads the BootROM Header (Boot Image Header) from QSPI Flash.
\item The BootROM code loads the First Stage Boot Loader (FSBL) into On-chip memory (OCM).
\item The BootROM code starts executing the FSBL (ending the BootROM execution).
\item (Optional) The FSBL code configures the programmable logic.
\item The FSBL code determines what happens next, eg., on the MiniZed it
typically loads and executes the second stage boot loader (SSBL), U-Boot.
\item The U-Boot code loads and executes Linux.
\end{enumerate}
%
The Zynq processor memory map during BootROM execution, and after handoff to the
FSBL, is given in Figure 6-11~\cite{Xilinx_UG585_2021}: the main change in the
memory map is the presence of the 64kB BootROM (at address 0x00030000) after the
192kB OCM (at address 0) during BootROM execution.

% -----------------------------------------------------------------------------
% Zynq Boot Image Header
% -----------------------------------------------------------------------------
%
\begin{figure}[p]
  \begin{center}
    \includegraphics[width=\textwidth]
    {figures/zynq_bootrom_header.png}
  \end{center}
  \caption{Zynq BootROM Image Header (from Table 6-5, pp172-173~\cite{Xilinx_UG585_2021}).}
  \label{fig:zynq_bootrom_header}
\end{figure}
% -----------------------------------------------------------------------------
%
Figure~\ref{fig:zynq_bootrom_header} shows the format of the BootROM Header
(Boot Image Header) in the QSPI flash boot image. The BootROM header is defined
in the Zynq \emph{Technical Reference Manual} (TRM)~\cite{Xilinx_UG585_2021},
while boot image creation using the \verb+bootgen+ tool is described in the
\emph{Software Developers Guide} (pp48-51, and Appendix A~\cite{Xilinx_UG821_2015}).
%
The format of the MiniZed QSPI Flash boot image is typically:
%
\begin{itemize}
\item BootROM Header (Boot Image Header)
\item First Stage Boot Loader (FSBL)
\item Programmable Logic bitstream
\item U-Boot (the second stage bootloader)
\item Additional partitions, eg., Linux and the device tree.
\end{itemize}
%
Insight into the BootROM Header can be gained by decoding an example.
The BootROM header for the MiniZed \verb+flash_fallback_7007S.bin+
image can be viewed using U-Boot via:
%
\begin{verbatim}
Zynq> sf probe
SF: Detected n25q128 with page size 256 Bytes, erase size 64 KiB, total 16 MiB
Zynq> sf read 10000000 0 1000000
device 0 whole chip
SF: 16777216 bytes @ 0x0 Read: OK
Zynq> crc32 10000000 fc0be4
crc32 for 10000000 ... 10fc0be3 ==> 005b7b3c
Zynq> md 10000000 30
10000000: eafffffe eafffffe eafffffe eafffffe    ................
10000010: eafffffe eafffffe eafffffe eafffffe    ................
10000020: aa995566 584c4e58 00000000 01010000    fU..XNLX........
10000030: 00001700 00018008 00000000 00000000    ................
10000040: 00018008 00000001 fc164530 00000000    ........0E......
10000050: 00000000 00000000 00000000 00000000    ................
10000060: 00000000 00000000 00000000 00000000    ................
10000070: 00000000 00000000 00000000 00000000    ................
10000080: 00000000 00000000 00000000 00000000    ................
10000090: 00000000 00000000 000008c0 00000c80    ................
100000a0: ffffffff 00000000 ffffffff 00000000    ................
100000b0: ffffffff 00000000 ffffffff 00000000    ................
Zynq>
\end{verbatim}
%
The U-Boot commands copy the QSPI flash to DDR memory, calculates the
image checksum (to confirm it matches the value in
Table~\ref{tab:factory_image_checksums}),
and displays the first $48\times32$-bit words (192-bytes) of the
Boot Image Header. The image details are:
%
\begin{enumerate}
\item \texttt{0x000:} \textbf{Interrupt Table for Execution-in-Place}

The interrupt table is the first $8\times32$-bit words of the header.
The 32-bit value 0xEAFFFFFE corresponds to an ARMv7 branch-to-self
instruction~\cite{ARM_ARMv7_2018}, so each interupt handler will
\emph{lock-up} the processor.

\item \texttt{0x020:} \textbf{Width Detection}

0xAA995566 is used for width detection (p173~\cite{Xilinx_UG585_2021}).

\item \texttt{0x024:} \textbf{Image Identification}

0x584C4E58 corresponds to 8-bit ASCII codes for
X, L, N, and X in little-endian format, which is why the ASCII
value displayed by U-Boot is XNLX  (p173~\cite{Xilinx_UG585_2021}).

\item \texttt{0x028:} \textbf{Encryption Status}

0x00000000 indicates the image is not encrypted.

\item \texttt{0x02C:} \textbf{FSBL/User Defined}

0x01010000 indicates the BootROM header version (p174~\cite{Xilinx_UG585_2021}).

\item \texttt{0x030:} \textbf{Source Offset}

0x00001700 offset in bytes to the FSBL image. The flash contents
near this offset are:
%
\begin{verbatim}
Zynq> md 100016e0 10
100016e0: ffffffff ffffffff ffffffff ffffffff    ................
100016f0: ffffffff ffffffff ffffffff ffffffff    ................
10001700: ea000049 ea000025 ea00002b ea00003b    I...%...+...;...
10001710: ea000032 e320f000 ea000000 ea00000f    2..... .........
\end{verbatim}
%
The locations prior to offset 0x1700 are blank, and the data starting
at offset 0x1700 look like ARMv7 instructions.

\item \texttt{0x034:} \textbf{Length of Image}

0x00018008 the image length is 96kbytes plus 8-bytes.

\item \texttt{0x038:} \textbf{FSBL Load Address}

0x00000000 load to the start of OCM.

\item \texttt{0x03C:} \textbf{Start of Execution}

0x00000000 execute from the start of OCM.

\item \texttt{0x040:} \textbf{Total Image Length}

0x00018008 is the same as \texttt{0x034:} \textbf{Length of Image}.

For non-secure boot mode, 0x040 and 0x034 must match
(p175~\cite{Xilinx_UG585_2021}).

\item \texttt{0x044:} \textbf{QSPI Configuration Word}

0x00000001 matches expected value (p175~\cite{Xilinx_UG585_2021}).

\item \texttt{0x048:} \textbf{Header Checksum}

0xFC164530 matches the inverted sum of 32-bit words 0x020 to 0x044,
i.e., $\sim$((0xAA995566 + 0x584C4E58 + 0x01010000 + 0x00001700 + 0x00018008
+ 0x00018008 + 0x00000001) \& 0xFFFFFFFF).

\item \texttt{0x04C:} \textbf{FSBL/User Defined} (76-bytes)

Bytes 0x04C to 0x097 are all zero (unused). These bytes can be defined
using \verb+bootgen+ and the \verb+udf_field+ in the BIF file
(see Appendix A~\cite{Xilinx_UG821_2015}).

\item \texttt{0x098:} \textbf{Image Header Table Offset}

0x000008C0 Image Header Table offset.

\textcolor{red}{Note: this field is missing from Table 6-5, but is listed in
the text, with the wrong name!} (see p175~\cite{Xilinx_UG585_2021}).

\item \texttt{0x09C:} \textbf{Partition Header Table Offset}

0x00000C80 Partition Header Table offset.

\textcolor{red}{Note: this field is missing from Table 6-5, but is listed in
the text, with the wrong description!} (see p176~\cite{Xilinx_UG585_2021}).

\item \texttt{0x0A0:} \textbf{Register Initialization} (2048-bytes)

The repeated pattern of 0xFFFFFFFF 0x00000000 indicates that
the register initialization block is not used (the register
address 0xFFFFFFFF terminates register initialization
(p176~\cite{Xilinx_UG585_2021}).

\item \texttt{0x8A0:} \textbf{Image Header}

Table 6-5 refers to this as the Image Header area, however, the text on
p176 indicates this is FSBL/User Defined. The Image Header Table offset
points to 0x8C0, not 0x8A0. Viewing the copy of the QSPI flash image in
DDR using U-Boot gives:
%
\begin{verbatim}
Zynq> md 100008a0 58
100008a0: ffffffff ffffffff ffffffff ffffffff    ................
100008b0: ffffffff ffffffff ffffffff ffffffff    ................
100008c0: 01020000 00000004 00000320 00000240    ........ ...@...
100008d0: 00000000 ffffffff ffffffff ffffffff    ................
100008e0: ffffffff ffffffff ffffffff ffffffff    ................
100008f0: ffffffff ffffffff ffffffff ffffffff    ................
10000900: 00000250 00000320 00000000 00000001    P... ...........
10000910: 7a796e71 5f667362 6c2e656c 66000000    qnyzbsf_le.l...f
10000920: 00000000 ffffffff ffffffff ffffffff    ................
10000930: ffffffff ffffffff ffffffff ffffffff    ................
10000940: 00000260 00000330 00000000 00000001    `...0...........
10000950: 64657369 676e5f31 5f777261 70706572    ised1_ngarw_repp
10000960: 2e626974 00000000 00000000 ffffffff    tib.............
10000970: ffffffff ffffffff ffffffff ffffffff    ................
10000980: 00000270 00000340 00000000 00000001    p...@...........
10000990: 752d626f 6f742e65 6c660000 00000000    ob-ue.to..fl....
100009a0: ffffffff ffffffff ffffffff ffffffff    ................
100009b0: ffffffff ffffffff ffffffff ffffffff    ................
100009c0: 00000000 00000350 00000000 00000001    ....P...........
100009d0: 696d6167 652e7562 00000000 00000000    gamibu.e........
100009e0: ffffffff ffffffff ffffffff ffffffff    ................
100009f0: ffffffff ffffffff ffffffff ffffffff    ................
\end{verbatim}
%
This shows that the bytes at offset 0x8A0 are unused, and the bytes
at 0x8C0 are used.
The \textbf{Image Header Table Offset} 0x098 indicates there is where
the image header table is located. The format of the image
header table is defined in Table A-4, p63~\cite{Xilinx_UG821_2015}.
The first word matches the expected value of 0x01020000. The
second word indicates there are 4 image headers. The third word
indicates the partition header is at word offset 0x320, which is
byte offset 0xC80, and that value matches the value for the
\textbf{Partition Header Table Offset} 0x09C. The forth word
indicates the first image header is at word offset 0x240, which
is byte offset 0x900. The fifth word 0x00000000 is unused, and
the values 0xFFFFFFFF pad the header table to the next 64-byte
(0x40) boundary.

\item \texttt{0x8C0:} \textbf{Partition Header}

Table 6-5 refers to this as the Partion Header area, however, the
decoding of the header indicates there is where the Image Header Table
is located. The image header was decoded in the previous step.

\item \texttt{0x900:} \textbf{FSBL Image or User Code}

The Image Header Table indicates that this is the location of the
first of the array of image headers. The Image Header format is
defined in in Table A-7, p66~\cite{Xilinx_UG821_2015}.
%
Decoding the strings for the four image names gives;
\verb+zynq_fsbl.elf+, \verb+design_1_wrapper.bit+, \verb+u-boot.elf+,
and \verb+image.ub+. This indicates that the QSPI flash fallback
image contains the FSBL, a bitfile, U-boot, and a Linux image,
i.e., exactly what we would expect to find in the fallback image.

\end{enumerate}
%
The manual decoding of the MiniZed QSPI flash fallback image has exposed
some minor inconsistencies in the Xilinx documentation, however, the
documentation provides sufficient information to interpret a boot image.


%\clearpage
% =============================================================================
\section{Xilinx Hardware and Software Development Tool Versions}
% =============================================================================

The Avnet MiniZed tutorials~\cite{Avnet_MiniZed_Tutorial01_2018,
Avnet_MiniZed_Tutorial02_2018,Avnet_MiniZed_Tutorial03_2018,
Avnet_MiniZed_Tutorial04_2018,Avnet_MiniZed_Tutorial05_2018,
Avnet_MiniZed_Tutorial06_2018,Avnet_MiniZed_Tutorial07_2018,
Avnet_MiniZed_Tutorial08_2018} use;
%
\begin{itemize}
\item Xilinx Vivado 2018.1 for hardware development.
\item Xilinx Software Development Kit (SDK) 2018.1 for software development.
\item Hardware Definition Files (HDF) for hardware-to-software handoff.
\end{itemize}
%
This tutorial uses the latest version of the Xilinx tools (at 4/13/2021);
%
\begin{itemize}
\item Xilinx Vivado 2020.2 for hardware development.
\item Xilinx Vitis Unified Software Development Platform 2020.2 for software development.
\item Xilinx Support Archive (XSA) files for hardware-to-software handoff.
\end{itemize}
%


\clearpage
% =============================================================================
\section{Blinky LED Designs}
% =============================================================================

The MiniZed has three LEDs;
%
\begin{itemize}
\item A bicolor LED connected to the Processor System (PS).
\item A bicolor LED connected to the Programmable Logic (PL).
\item A blue LED connected to the programmable logic configuration controller
DONE output.
\end{itemize}
%
The design examples in this section show how to blink the LEDs using
software running on the PS. The hardware designs differ in how the
LEDs are controlled by the PS;
%
\begin{itemize}
\item PS bicolor LED control via PS GPIO MIO (no PL use).
\item PL bicolor and blue LED control via the PS GPIO EMIO.
\item PL bicolor and blue LED control using an AXI GPIO.
\end{itemize}
%
These simple designs demonstrate how the PS and PL are configured using
Vivado, and how bare-metal applications are created using the Software
Development Kit (SDK).


\clearpage
%------------------------------------------------------------------------------
\subsection{PS bicolor LED control via U-Boot}
%------------------------------------------------------------------------------

The MiniZed schematic (p4~\cite{Avnet_MiniZed_Schematic_2017}) shows that the
PS bicolor LED is controlled by the Zynq GPIO signals MIO52
(\verb+PS_LED_R+) and MIO53 (\verb+PS_LED_G+).
%
The Zynq TRM describes the GPIO multiplexed I/O (MIO) and extended multiplexed
I/O (EMIO) in Chapter 14~\cite{Xilinx_UG585_2018}.
%
Figure~\ref{fig:zynq_gpio_block_diagram} shows the GPIO block diagram;
banks 0 and 1 connect to the MIO, while banks 2 and 3 connect to the EMIO.
%
Figure~\ref{fig:zynq_gpio_block_diagram} shows the GPIO control registers.
The GPIO control register definitions and location in the Zynq address map
is defined in Appendix B.19~\cite{Xilinx_UG585_2018}. The GPIO registers
are located at address 0xE000A000.

The PS bicolor LED is controlled by MIO bits 52 and 53 which are located in GPIO
bank 1. The GPIO bank 1 control registers are; \verb+DATA+ (0xE000A00C),
\verb+DIRM+ (0xE000A244), and \verb+OEN+ (0xE000A248) registers.
%
The PS bicolor LED can be controlled from U-Boot as follows:
%
\begin{verbatim}
# Set the direction
Zynq> mw e000a244 00300000

# Enable the output (the bicolor LED turns off)
Zynq> mw e000a248 00300000

# Write to the registers using the mask
Zynq> mw e000a00c ffcf0000  # off
Zynq> mw e000a00c ffcf0010  # red
Zynq> mw e000a00c ffcf0020  # green
Zynq> mw e000a00c ffcf0030  # red+green
\end{verbatim}
%
The PL LEDs can also be controlled using U-Boot commands, however, the
programmable logic first needs to be configured with a design that
either routes the EMIO to the LEDs, or implements the AXI GPIO that
controls the LEDs. Using U-Boot for read and write access to Zynq PS
registers and PL designs is a simple, but powerful, debugging tool.

\clearpage
% -----------------------------------------------------------------------------
% Zynq GPIO Block Diagram
% -----------------------------------------------------------------------------
%
\begin{figure}[p]
  \begin{center}
    \includegraphics[width=0.5\textwidth]
    {figures/zynq_gpio_block_diagram.png}
  \end{center}
  \caption{Zynq GPIO block diagram (from Figure 14-1~\cite{Xilinx_UG585_2018}).}
  \label{fig:zynq_gpio_block_diagram}
\end{figure}
% -----------------------------------------------------------------------------

% -----------------------------------------------------------------------------
% Zynq GPIO Registers
% -----------------------------------------------------------------------------
%
\begin{figure}[p]
  \begin{center}
    \includegraphics[width=0.75\textwidth]
    {figures/zynq_gpio_registers.png}
  \end{center}
  \caption{Zynq GPIO control registers (from Figure 14-2~\cite{Xilinx_UG585_2018}).}
  \label{fig:zynq_gpio_registers}
\end{figure}
% -----------------------------------------------------------------------------

\clearpage
%------------------------------------------------------------------------------
\subsection{PS bicolor LED control via GPIO MIO}
%------------------------------------------------------------------------------
\label{sec:blinky_gpio_mio}

The PS bicolor LED control via the GPIO MIO design uses only the Processor
System (PS), it does not use the Programmable Logic (PL).
%
The Xilinx Vivado tool is used to configured the Zynq Processor, and
the Xilinx Software Development Kit (SDK) is used to develop a bare-metal
application to blink the bicolor LED. The PS bicolor LED design duplicates
many of the steps found in Avnet MiniZed tutorials 01, 02, and
04~\cite{Avnet_MiniZed_Tutorial01_2018,Avnet_MiniZed_Tutorial02_2018,Avnet_MiniZed_Tutorial04_2018}.
The Avnet tutorials should be reviewed in conjunction with this design.

\begin{enumerate}
%------------------------------------------------------------------------------
\item \textcolor{blue}{\textbf{Vivado Project Creation}}
%------------------------------------------------------------------------------

Create the MiniZed board \verb+blinky_gpio_mio+ project:
%
\begin{enumerate}
\item Start Vivado.
\item Under \emph{Quick Start}, click on \emph{Create Project}.
\item {\bf Create a New Vivado Project}
%
\begin{itemize}
\item Click \emph{Next}
\end{itemize}
%
\item {\bf Project Name}
%
\begin{itemize}
\item \emph{Project name}: \verb+blinky_gpio_mio+
\item \emph{Project location}: \verb+c:/temp/minized/blinky_gpio_mio/vwork+
\item Uncheck \emph{Create project subdirectory}
\item Click \emph{Next}
\end{itemize}
%
The project location specifies the Vivado work directory, \verb+vwork+,
where Vivado generates the project files. After the GPIO MIO project has
been completed, a script is developed that automates the project generation.
%
\item {\bf Project Dialog}
%
\begin{itemize}
\item Select the \emph{RTL Project} bullet
\item Check \emph{Do not specify sources at this time}
\item Click \emph{Next}
\end{itemize}
%
\item {\bf Default Part}
%
\begin{itemize}
\item Click on the \emph{Boards} tab
\item On the \emph{Vendor} pull-down, select \emph{em.avnet.com}
\item Select the \emph{MiniZed} in the board list
\item Click \emph{Next}
\end{itemize}
%
\item {\bf New Project Summary}
%
\begin{itemize}
\item Click \emph{Finish}
\end{itemize}
%
\end{enumerate}
%
%------------------------------------------------------------------------------
\item \textcolor{blue}{\textbf{Vivado Block Diagram Creation}}
%------------------------------------------------------------------------------
%
\begin{enumerate}
%
\item In the \emph{Project Manager}, under \emph{IP Integrator},
select \emph{Create Block Design}.
%
\item {\bf Create Block Design}
%
\begin{itemize}
\item \emph{Design name}: \verb+system+
\item \emph{Directory}: \verb+<Local to Project>+
\item \emph{Specify Source set}: \verb+Design Sources+
\item Click \emph{OK}
\end{itemize}
%
\newpage
\item {\bf Add the Zynq Processor}
%
\begin{itemize}
\item Click the `+' in the middle of the design area to open the IP list
\item In the \emph{Search} bar enter \verb+Zynq+
\item Double-click on \verb+ZYNQ7 Processing System+ to add the Zynq to
the design
\item In the {\bf Block Properties} dialog, change the name from
\verb+processing_system7_0+ to \verb+zynq+.
\end{itemize}
%
\item {\bf Run Block Automation}
%
\begin{itemize}
\item Click on the blue hyperlink \textcolor{blue}{Run Block Automation}
\item Accept the block automation defaults by clicking \emph{OK}
\end{itemize}
%
\end{enumerate}
%
% -----------------------------------------------------------------------------
% PS bicolor LED GPIO MIO Design - Vivado Block Diagram
% -----------------------------------------------------------------------------
%
% Installed "A Ruler for Windows" and set the Diagram GUI detached
% window size to 5in x 10in for screen capture
%
\begin{figure}[t]
  \begin{minipage}{\textwidth}
    \begin{center}
    \includegraphics[width=0.8\textwidth]
    {figures/blinky_gpio_mio_vivado_diagram_a.png}\\
    (a) Zynq processor
    \end{center}
  \end{minipage}
  \vskip5mm
  \begin{minipage}{\textwidth}
    \begin{center}
    \includegraphics[width=0.8\textwidth]
    {figures/blinky_gpio_mio_vivado_diagram_b.png}\\
    (c) Zynq processor after block automation
    \end{center}
  \end{minipage}
  \caption{PS bicolor LED control via GPIO MIO design Vivado diagram window.}
  \label{fig:blinky_gpio_mio_vivado_diagram}
\end{figure}
% -----------------------------------------------------------------------------
%
Figure~\ref{fig:blinky_gpio_mio_vivado_diagram}(a) shows the diagram
window after adding the Zynq to the \verb+system+ design.
Figure~\ref{fig:blinky_gpio_mio_vivado_diagram}(b) shows the
diagram window after running block automation.
%
Block automation configures the Zynq processor using the board
presets, \verb+preset.xml+, which is part of the MiniZed
board definition files (BDF) located in the directory:
%
\begin{verbatim}
C:\software\Xilinx\Vivado\2019.1\data\boards\board_files\minized\1.2\
\end{verbatim}
%
This example uses the MiniZed Zynq processor defaults. Later design examples
customize the Zynq processor.
%
%------------------------------------------------------------------------------
\item \textcolor{blue}{\textbf{Vivado Synthesis}}
%------------------------------------------------------------------------------

The need for \emph{Synthesis} in a design that does not use the programmable
logic sounds redundant. However, the Zynq design flow uses Vivado to configure
the (required) Zynq processor system and the (optional) programmable logic.
%
Vivado generates the hardware definition file (HDF) used by the SDK. Vivado
was used to configure the Zynq with the MiniZed presets, so that the SDK can
be used to create the blinky LED application. The hardware support files needed
by the SDK are generated as follows.
%
\begin{enumerate}
%
\item {\bf Create HDL Wrapper}
%
\begin{itemize}
\item In the Vivado {\bf Sources} window, highlight \verb+system (system.bd)+
\item Right-mouse-click and select \emph{Create HDL Wrapper}
\item Accept the default option \emph{Let Vivado manage wrapper and auto-update}
\item Click \emph{OK}
\item This generates the top-level design file \verb+system_wrapper.v+
\end{itemize}
%
\item {\bf Constraints}

This design uses only Zynq processor system I/Os, no programmable logic I/Os,
so pin constraints are not required. The Zynq pins are configured by the
FSBL, which is generated by Vivado as part of the synthesis step.
%
\item {\bf Run Synthesis}
%
\begin{itemize}
\item In the Vivado {\bf Project Manager} click \emph{Run Synthesis}
\item Accept the defaults in the {\bf Launch Runs} dialog
\item Click \emph{OK}
\item Synthesis generates output in the directory

\verb+vwork/blinky_gpio_mio.runs+

\item When the {\bf Synthesis Completed} dialog appears, close it
by clicking \emph{Cancel}
\end{itemize}
%
\item {\bf Export Hardware}
%
\begin{itemize}
\item Select \emph{File}$\rightarrow$\emph{Export}$\rightarrow$\emph{Export Hardware}
\item Leave the \emph{Include bitstream} checkbox unchecked
\item Accept the default \emph{Export to:} \verb+<Local to Project>+
\item Click \emph{OK}
\item This step generates the HDF file used by the SDK
%
\begin{verbatim}
vwork/blinky_gpio_mio.sdk/system_wrapper.hdf
\end{verbatim}
\end{itemize}
%
\newpage
\item {\bf HDF Analysis} (optional step)
%
\begin{itemize}
\item Create a copy of the HDF file
\item Change the extension from \verb+.hdf+ to \verb+.zip+
\item Unzip the file and view the contents of \verb+system_wrapper+:
\begin{itemize}
\item \verb+ps7_init.tcl+

Used by the SDK to initialize the processor during JTAG download

\item \verb+ps7_init*.h/c+

First stage bootloader (FSBL) source.

\item \verb+system_bd.tcl+

Block design Tcl script.
\end{itemize}
\item Delete the zip file and the \verb+system_wrapper+ folder
\end{itemize}
%
\end{enumerate}
%
%------------------------------------------------------------------------------
\item \textcolor{blue}{\textbf{SDK Project Creation}}
%------------------------------------------------------------------------------
%
\begin{enumerate}
\item \textbf{Method 1}: Start the SDK tool from Vivado
%
\begin{itemize}
\item Select the Vivado menu option \emph{File}$\rightarrow$\emph{Launch SDK}
\item Accept the {\bf Launch SDK} dialog defaults and click \emph{OK}
\item This method sets up the SDK to use the \verb+.sdk+ folder within the
Vivado work folder, i.e.,

\verb+blinky_gpio_mio\vwork\blinky_gpio_mio.sdk+

\item The SDK unzips the \verb+.hdf+ archive into the folder

\verb+blinky_gpio_mio\vwork\blinky_gpio_mio.sdk\system_wrapper_hw_platform_0+

\item The SDK opens and displays the \textbf{Address Map for processor ps7\_cortexa9\_0},
(which is generated from the settings in \verb+system.hdf+).
Table~\ref{tab:blinky_gpio_mio} shows a selection of addresses
relevant to the GPIO MIO design.

\item Double-mouse-click on \verb+ps7_init.html+ to see a detailed view
of the address map, and the register settings. The registers are configured
when the SDK runs \texttt{ps7\_init.tcl} during JTAG download, or when the
processor boots and runs the FSBL.

\item Scroll down to \textbf{MIO 52} and \textbf{MIO 53} to see that the
multiplexed I/O (MIO) used to control the red and green PS LEDs are configured
as GPIO.

\item Scroll further down and see that the PS-to-PL clocks
\textbf{FPGA0, 1, 2, 3} are configured for frequencies of
50MHz, 100MHz, 10MHz, and 10MHz. The PS-to-PL clocks are not used
in the GPIO MIO example.
\end{itemize}

\item \textbf{Method 2}: Start the SDK tool from the Windows Start Menu (or from a Linux console)

The Avnet MiniZed Tutorial 02~\cite{Avnet_MiniZed_Tutorial02_2018} shows how
to start the SDK and import a Vivado hardware platform. The Avnet tutorial shows
the SDK steps that are automated by starting the SDK from within Vivado.
The manual method of creating the SDK is useful if you want to change the
SDK project names from the defaults used by Vivado.

\end{enumerate}
%
% -----------------------------------------------------------------------------
% Blinky GPIO MIO Address Map
% -----------------------------------------------------------------------------
%
\begin{table}[t]
\caption{PS bicolor LED GPIO MIO Example Address Map (selected addresses).}
\label{tab:blinky_gpio_mio}
\begin{center}
\begin{tabular}{|l|l|l|}
\hline
Zynq Component & Cell & Base Address\\
\hline\hline
\textbf{Memory} &&\\
\hline
&&\\
On-Chip Memory\#0                 & \texttt{ps7\_ram\_0}          & \texttt{0x00000000}\\
DDR Memory                        & \texttt{ps7\_ddr\_0}          & \texttt{0x00100000}\\
QSPI Linear (Read-Only) Addresses & \texttt{ps7\_qspi\_linear\_0} & \texttt{0xFC000000}\\
On-Chip Memory\#1                 & \texttt{ps7\_ram\_1}          & \texttt{0xFFFF0000}\\
&&\\
\hline
\textbf{Peripheral} &&\\
\hline
&&\\
UART\#0 Registers    & \texttt{ps7\_uart\_0}         & \texttt{0xE0000000}\\
UART\#1 Registers    & \texttt{ps7\_uart\_1}         & \texttt{0xE0001000}\\
GPIO MIO  Registers  & \texttt{ps7\_gpio\_0}         & \texttt{0xE000A000}\\
QSPI Registers       & \texttt{ps7\_qspi\_0}         & \texttt{0xE000D000}\\
&&\\
DDR Registers        & \texttt{ps7\_ddrc\_0}         & \texttt{0xF8006000}\\
&&\\
\hline
\end{tabular}
\end{center}
\end{table}
% -----------------------------------------------------------------------------

\newpage
%------------------------------------------------------------------------------
\item \textcolor{blue}{\textbf{SDK Board Support Package (BSP) Creation}}
%------------------------------------------------------------------------------
%
\begin{enumerate}
\item Select \emph{File}$\rightarrow$\emph{New}$\rightarrow$\emph{Board Support Package}
\item Accept the default settings for the project \texttt{standalone\_bsp\_0} and
click \emph{Finish}.
\item \textbf{Board Support Package Settings}

The MiniZed connects the Zynq UART\#0 to the bluetooth device and UART\#1 to the
FTDI USB UART interface. The board support package needs to be configured to use
UART\#1 for standard input/output as follows:

\begin{itemize}
\item Select \emph{Overview}, \emph{standalone}
\item Change the pull-downs for \texttt{stdin} and \texttt{stdout} to \texttt{ps7\_uart\_1}
\item Click \emph{OK}
\end{itemize}
\end{enumerate}

After the BSP has been generated, under \emph{Project Explorer}, expand
\texttt{standalone\_bsp\_0}, and double-mouse-click on \texttt{system.mss} to
see the documentation and example designs for the components in the system.

%------------------------------------------------------------------------------
\item \textcolor{blue}{\textbf{SDK Baremetal Application Creation}}
%------------------------------------------------------------------------------

The PS bicolor LED control application is created by starting with \emph{Hello World}
example application. The \emph{Hello World} application configures the SDK
project to use the \texttt{platform.h/c} files, which define the functions for
initialization and cleanup. The platform initialization configures the
UART interface to be used for C library standard input and output.

\begin{enumerate}
\item Select \emph{File}$\rightarrow$\emph{New}$\rightarrow$\emph{Application Project}
\item \textbf{Application Project}

Configure the project name, and change the project to the BSP just generated
\begin{itemize}
\item \emph{Project name:} \texttt{blinky\_gpio\_mio}
\item \emph{OS Platform:} \texttt{standalone}
\item \emph{Hardware Platform:} \texttt{system\_wrapper\_hw\_platform\_0}
\item \emph{Language:} \texttt{C}
\item \emph{Board Support Package: Use Existing} \texttt{standalone\_bsp\_0}
\end{itemize}
\item Click \emph{Next}
\item Select \emph{Hello World}

\item Click \emph{Finish}
\item Under the \emph{Project Explorer}, expand the \texttt{blinky\_gpio\_mio} project
\item Right-mouse-click on \texttt{src}, select \texttt{helloworld.c}, right-mouse-click,
select \emph{Rename} and rename the file \texttt{blinky\_gpio\_mio.c}
\item Replace \texttt{blinky\_gpio\_mio.c} with the source code shown in
Figure~\ref{fig:blinky_gpio_mio_app}.
\end{enumerate}

The SDK will automatically build the GPIO MIO bare-metal application.
%
%------------------------------------------------------------------------------
\item \textcolor{blue}{\textbf{SDK Download and Run the Application}}
%------------------------------------------------------------------------------
%
\begin{enumerate}
\item Set the MiniZed Flash/JTAG mode switch to JTAG mode
\item Connect the MiniZed to your development machine using the USB JTAG/UART interface
\item Select the \emph{SDK Terminal}, click the green + symbol, and
setup the UART communications to the MiniZed at 115200 baud (the timeout
can be set to 1 second)
\item Under the \emph{Project Explorer}, select the \texttt{blinky\_gpio\_mio}
project, right-mouse-click, and select \emph{Run As}$\rightarrow$\emph{Launch
on Hardware (System Debugger)} to download and run the application
\item The PS LED will cycle through off, green, red, and amber (red+green)
\item Select the \emph{SDK Terminal} and you will see a message
for each LED change
\end{enumerate}

\textcolor{magenta}{Congratulations!} You have run your first custom application on the MiniZed board.

%------------------------------------------------------------------------------
% blinky_gpio_mio bare-metal application code
%------------------------------------------------------------------------------
%
\begin{figure}
\begin{center}
\begin{minipage}{0.8\textwidth}
\begin{lstlisting}
#include <sleep.h>
#include "platform.h"
#include "xil_printf.h"

int main()
{
	int i = 0;

	init_platform();

	xil_printf("MiniZed Blinky GPIO MIO Example\r\n");
	xil_printf("-------------------------------\r\n");

	// Clear the data (DATA) bits
	*(volatile unsigned int *)0xE000A00C = 0xFFCF0000;

	// Set the direction (DIRM) to output
	*(volatile unsigned int *)0xE000A244 = 0x00300000;

	// Enable the outputs (OEN)
	*(volatile unsigned int *)0xE000A248 = 0x00300000;

	// Blink the bicolor LED
	while (1) {
		switch (i%4) {
			case 0:
				xil_printf("%d: Off\r\n", i);
				*(volatile unsigned int *)0xE000A00C = 0xFFCF0000;
				break;

			case 1:
				xil_printf("%d: Green\r\n", i);
				*(volatile unsigned int *)0xE000A00C = 0xFFCF0020;
				break;

			case 2:
				xil_printf("%d: Red\r\n", i);
				*(volatile unsigned int *)0xE000A00C = 0xFFCF0010;
				break;

			default:
				xil_printf("%d: Amber\r\n", i);
				*(volatile unsigned int *)0xE000A00C = 0xFFCF0030;
				break;
		}
		sleep(1);
		i++;
	}

    cleanup_platform();
    return 0;
}
\end{lstlisting}
\end{minipage}
\end{center}
\caption{PS bicolor LED GPIO MIO bare-metal application.}
\label{fig:blinky_gpio_mio_app}
\end{figure}
%------------------------------------------------------------------------------

%
%------------------------------------------------------------------------------
\item \textcolor{blue}{\textbf{SDK Create the First Stage Boot Loader (FSBL)}}
%------------------------------------------------------------------------------

The procedure to create the FSBL is similar to the procedure used to
create a new application, however, no user customization is required, as the
FSBL uses information (register settings) defined in the hardware definition
file exported from Vivado.

\begin{enumerate}
\item Select \emph{File}$\rightarrow$\emph{New}$\rightarrow$\emph{Application Project}
\item \textbf{Application Project}

Configure the project name, and change the project to the BSP just generated
\begin{itemize}
\item \emph{Project name:} \texttt{zynq\_fsbl}
\item \emph{OS Platform:} \texttt{standalone}
\item \emph{Hardware Platform:} \texttt{system\_wrapper\_hw\_platform\_0}
\item \emph{Language:} \texttt{C}
\item \emph{Board Support Package: Create New} \texttt{zynq\_fsbl\_bsp}

A new BSP is needed as the FSBL requires \verb+xilffs+ library support.
\end{itemize}
\item Click \emph{Next}
\item Select \emph{Zynq FSBL}

\item Click \emph{Finish}

\item Modify the BSP to use \verb+ps7_uart_1+ for stdin and stdout.

\end{enumerate}

The SDK will automatically build the FSBL BSP and FSBL. The SDK Console
output should contain the message \verb+Finished building target: zynq_fsbl.elf+.

%------------------------------------------------------------------------------
\item \textcolor{blue}{\textbf{SDK Create the QSPI Flash Boot Image}}
%------------------------------------------------------------------------------
%
\begin{enumerate}
\item In the SDK, highlight the \verb+blinky_gpio_mio+ application.
\item From the SDK main menu, select \emph{Xilinx}$\rightarrow$\emph{Create Boot Image}.

Figure~\ref{fig:blinky_gpio_mio_bootgen_gui} shows the Create Boot Image dialog.
The defaults will generate the file \verb+BOOT.bin+. The Avnet MiniZed Tutorial 04
indicates that the boot image file type should be changed to
.MCS~\cite{Avnet_MiniZed_Tutorial04_2018}, however, this is not necessary.

\item Click \emph{Create Image}
\end{enumerate}

%------------------------------------------------------------------------------
\item \textcolor{blue}{\textbf{Program the QPSI Flash Boot Image}}
%------------------------------------------------------------------------------

\begin{enumerate}
\item If Vivado is not running, start Vivado, and select \emph{Open Hardware Manager}
from within the main GUI.

If Vivado has the \verb+blinky_gpio_mio+ project open, then in the \emph{Flow Navigator},
under \emph{PROGRAM AND DEBUG}, select \emph{Open Hardware Manager}

\item Use the hardware manager to connect to the target (the MiniZed board).

\item From the Vivado main menu, select \emph{Tools}$\rightarrow$\emph{Add Configuration Memory Device}.

Note that this option becomes disabled (the menu option becomes greyed out) after you have added a
configuration memory device.

\item In the \emph{Add Configuration Memory} dialog:
\begin{itemize}
\item Enter the \textbf{Filter} settings:
\begin{itemize}
\item \emph{Manufacturer:} \texttt{Micron}
\item \emph{Density (Mb):} \texttt{128}
\item \emph{Type:} \texttt{qspi}
\item \emph{Width:} \texttt{x1-single}
\end{itemize}
%
\item From the \textbf{Select Configuration Memory Part} list, select the first device:

\texttt{my25ql128-qspi-x1-single}
%
\item Click \emph{Ok}
\end{itemize}

\item In the \emph{Program Configuration Memory Device} window:
\begin{itemize}
\item For \textbf{Configuration file:} navigate to the \verb+BOOT.bin+ file in the \verb+blinky_gpio_mio+ application folder.
\item For \textbf{Zynq FSBL:} navigate to the \verb+zynq_fsbl.elf+ file in the \verb+zynq_fsbl+ application folder.

Figure~\ref{fig:blinky_gpio_mio_program_flash_gui} shows the configuration window after
the files have been selected.
\item Click \emph{Ok} to program the QSPI Flash
\end{itemize}
%
\item Unplug the MiniZed, change the boot mode switch to Flash, and plug it back into the development machine.
\end{enumerate}

The MiniZed will boot from flash and the bicolor LED will start cycling off, green, red, and amber. The
blue LED will remain off, as the PS bicolor LED project does not use the PL, so the configuration DONE
signal never asserts.

\textcolor{magenta}{Congratulations!} Your MiniZed board boots and runs your first custom application.

The next design adds the progammable logic into the mix!

\end{enumerate} % End of blinky_gpio_mio steps
%
% -----------------------------------------------------------------------------
% PS bicolor LED GPIO MIO Design - Bootgen GUI
% -----------------------------------------------------------------------------
%
\begin{figure}[p]
  \begin{center}
    \includegraphics[width=0.9\textwidth]
    {figures/blinky_gpio_mio_bootgen_gui.png}\\
  \end{center}
  \caption{PS GPIO MIO design Create Boot Image window.}
  \label{fig:blinky_gpio_mio_bootgen_gui}
\end{figure}
% -----------------------------------------------------------------------------
%
% -----------------------------------------------------------------------------
% PS bicolor LED GPIO MIO Design - Program Flash GUI
% -----------------------------------------------------------------------------
%
\begin{figure}[p]
  \begin{center}
    \includegraphics[width=0.9\textwidth]
    {figures/blinky_gpio_mio_program_flash_gui.png}\\
  \end{center}
  \caption{PS GPIO MIO design Program Coniguration Memory Device window.}
  \label{fig:blinky_gpio_mio_program_flash_gui}
\end{figure}
% -----------------------------------------------------------------------------
%


\clearpage
%------------------------------------------------------------------------------
\subsection{PL bicolor and blue LED control via GPIO EMIO}
%------------------------------------------------------------------------------

The PL bicolor and blue LED control design extends the GPIO MIO design:
%
\begin{itemize}
\item 16-bits of PS GPIO EMIO are connected to the PL.
\item The 7-series \verb+STARTUPE2+ component is instantiated in the PL
to control the blue LED.
\item The PL is used to connect GPIO EMIO outputs to the
bicolor and blue LED.
\item The PL is used to connect GPIO EMIO inputs to the user
push button and switch.
\item The Vivado design uses a top-level file that is \emph{not}
managed by Vivado.
\item The Vivado block design is instantiated in the top-level design.
\item An unmanaged constraints Tcl file is used to implement pin constraints.
\end{itemize}

\noindent\textcolor{red}{\bf TODO}
\begin{itemize}
\item I have already extended the GPIO MIO design and confirmed that EMIO can control the PL bicolor LED.
\item The post-block-automation Zynq processor has 16-bits of EMIO support. All I needed to do was
to select the GPIO\_0 port, right-mouse-click and select \emph{Make External} and a port named
\texttt{GPIO\_0\_0} is created. I renamed it \texttt{GPIO}.
\item Extend the design to support the blue LED, the user switch, and push button.
\item EMIO registers:
 - DATA\_2     at E000A048  Output Data, GPIO Bank 2, EMIO
 - DATA\_2\_RO at E000A068  Input Data, GPIO Bank 2, EMIO
 - DIRM\_2     at E000A284  Direction Mode, GPIO Bank 2, EMIO
 - OEN\_2      at E000A288  Output Enable, GPIO Bank 2, EMIO
\item Describe how to use U-Boot to control the EMIO LED?
\end{itemize}


\noindent\textcolor{red}{\bf Tutorial walk-through notes}

Follow through the first tutorial and just make notes about the changes.

\begin{itemize}
\item Vivado: Create a new project \texttt{blinky\_gpio\_emio}
\item Zynq PS: After running block automation, select
GPIO\_0, \emph{Make External}, and rename to \texttt{GPIO}
\item At this point we have a top-level design that we
do not want to generate a wrapper for, rather we want a
top-level file in which we can drop the \texttt{system} design as
a component.
\item Figure XXX shows a top-level HDL component created
based on the port definitions on

\verb+vwork\blinky_gpio_emio.srcs\sources_1\bd\system\synth\system.v+

\item Add \texttt{minized.sv} to the project
\item Add the pin constraints Tcl file to the project and set it
to be used for \emph{Implementation} only.
\item Run Synthesis. No issues, so Run Implementation.
No issues, so Generate Bitstream.
\item Check the arduino pins. A3 should be pin E13 and A4 should be pin E12.

\begin{verbatim}
| E11        | arduino_a[5]      | High Range | IO_L2P_T0_AD8P_35       | BIDIR       | LVCMOS33           |      35 |          8 | SLOW |                     |                 NONE |         | FIXED      |           |          |      | NONE             |              |                   |              |
| E12        | arduino_a[4]      | High Range | IO_L2N_T0_AD8N_35       | BIDIR       | LVCMOS33           |      35 |          8 | SLOW |                     |                 NONE |         | FIXED      |           |          |      | NONE             |              |                   |              |
| E13        | arduino_a[3]      | High Range | IO_L1N_T0_AD0N_35       | BIDIR       | LVCMOS33           |      35 |          8 | SLOW |                     |                 NONE |         | FIXED      |           |          |      | NONE             |              |                   |              |
| F12        | arduino_a[2]      | High Range | IO_L1P_T0_AD0P_35       | BIDIR       | LVCMOS33           |      35 |          8 | SLOW |                     |                 NONE |         | FIXED      |           |          |      | NONE             |              |                   |              |
| F13        | arduino_a[1]      | High Range | IO_L3P_T0_DQS_AD1P_35   | BIDIR       | LVCMOS33           |      35 |          8 | SLOW |                     |                 NONE |         | FIXED      |           |          |      | NONE             |              |                   |              |
| F14        | arduino_a[0]      | High Range | IO_L3N_T0_DQS_AD1N_35   | BIDIR       | LVCMOS33           |      35 |          8 | SLOW |                     |                 NONE |         | FIXED      |           |          |      | NONE             |              |                   |              |
\end{verbatim}

\item Export hardware and include the bitstream.
\item Launch SDK, setup the BSP, and the application.
\item Since this design has a bitstream, the FPGA needs to be programmed first,
and then the application downloaded.

\item Changing the LEDs using the EMIO should just need DIRM = 1 at offset 284.
Yes. That and the data output were all that were needed.
Ok, now I have enough to write up this tutorial step.
\end{itemize}

PL bi-color LED on pin E12 PL\_LED\_R, E13 and PL\_LED\_G. Not sure about the names,
since they're also on Arduino pins.



%------------------------------------------------------------------------------
\subsection{PL bicolor and blue LED control via AXI GPIO}
%------------------------------------------------------------------------------

Create a Zynq processor system with;
%
\begin{itemize}
\item PS bicolor LED control
\item PL bicolor LED control from AXI GPIO
\item PL blue DONE lED control from AXI GPIO
\item PL switch readback via AXI GPIO
\item PL push button readback via AXI GPIO
\item Do not use the Vivado block design wrapper file.
\item Describe the use of unmanaged constraints.
\end{itemize}

%------------------------------------------------------------------------------
\subsection{Scripting the Blinky LED Designs}
%------------------------------------------------------------------------------


\clearpage
% =============================================================================
\section{Programmable Logic (PL) Only Designs}
% =============================================================================

Programmable Logic Only.

Blink an LED from the internal oscillator, the external oscillator (none),
and an PS oscillator.

Select which oscillator using the select switch?

Or just blink the blue LED using the internal oscillator clock and
the bi-color LED using the PS clock.

If the QSPI flash is erased, or the PS design does not enable the FCLK,
then the counter clocked by the PS does not increment.

Describe the use of unmanaged constraints.


\subsection{Using the FPGA internal oscillator}

\subsection{Using the PS-to-PL clocks}

\clearpage
\section{Peripheral Designs}
% =============================================================================
\subsection{QSPI Flash}
% =============================================================================

Create an application that uses the QSPI PS library to access the flash.
The application should support all the flash features via the UART or
WiFi interface. Write a PC-side application that allows you to read
and write the contents of flash. Use this to read the contents of
a MiniZed that has never been programmed, and one restored to factory
defaults.

Use the Xilinx QSPI code to read the flash and calculate the CRC32 using
the same code I used under Cygwin. This would demonstrate a pretty basic
use of the library.

Extend this application to access the eMMC too?
\clearpage
% =============================================================================
\subsection{I2C to PMIC and Sensors}
% =============================================================================

\clearpage
% =============================================================================
\subsection{SPI to PMod ADC and DAC}
% =============================================================================


\appendix
% =============================================================================
% Appendices
% =============================================================================
\clearpage
% =============================================================================
\section{Factory Restore}
% =============================================================================
\label{sec:factory_restore}

The Avnet guide \emph{Restoring MiniZed to the Factory State}~\cite{Avnet_MiniZed_Restore_2018}
describes how to restore the MiniZed to factory state. This section
shows how to view the version information of U-Boot and Linux on a MiniZed,
how to calculate the checksums of the images in on-board non-volatile memory,
and how to compare those checksums to the recovery or user images.

% -----------------------------------------------------------------------------
\subsection{Viewing Tool and Software Versions}
% -----------------------------------------------------------------------------
\label{sec:restore_view_versions}

To view the MiniZed software versions, start by setting the boot mode switch
to Flash, connecting the board to your development machine using the USB
JTAG/UART interface (see Figure~\ref{fig:minized_quick_start_diagram}), and
configure a serial terminal for communication at 115200 baud 8N1.
Hit enter and the MiniZed console output (for a device purchased in March 2019) was
%
\begin{verbatim}
PetaLinux 2017.4 MiniZed /dev/ttyPS0

MiniZed login:
\end{verbatim}
%
Press the blue reset button (by the PMod connectors on the bottom right
of Figure~\ref{fig:minized_quick_start_diagram}), press the enter key to
stop U-Boot from loading Linux, and enter the \verb+version+ command to
get details on the U-Boot version. The console output was
%
\begin{verbatim}
U-Boot 2017.01 (Mar 22 2018 - 23:33:56 -0700)

Board: Xilinx Zynq
DRAM:  ECC disabled 512 MiB
MMC:   Card did not respond to voltage select!
sdhci@e0100000 - probe failed: -95
sdhci_transfer_data: Error detected in status(0x208000)!
Card did not respond to voltage select!

SF: Detected n25q128 with page size 256 Bytes, erase size 64 KiB, total 16 MiB
*** Warning - bad CRC, using default environment

In:    serial
Out:   serial
Err:   serial
U-BOOT for MiniZed

Hit any key to stop autoboot:  0
Zynq>  version

U-Boot 2017.01 (Mar 22 2018 - 23:33:56 -0700)
arm-xilinx-linux-gnueabi-gcc (Linaro GCC 6.2-2016.11) 6.2.1 20161016
GNU ld (GNU Binutils) 2.27.0.20160806
Zynq>
\end{verbatim}
%
Note that the PetaLinux prompt and U-Boot messages have different version strings.
The PetaLinux prompt implies that Xilinx Vivado and PetaLinux tools version 2017.4
were used to build PetaLinux, while the U-Boot version string implies
2017.1 tools were used to build U-Boot.

Type \verb+?+ and then press the enter key to see the U-Boot help menu. U-Boot can be
used to read and write memory locations, read and program flash, boot standalone
programs, real-time operating systems, and Linux. The example designs in this tutorial
demonstrate some of these U-Boot features.

\newpage
Boot Linux using the \verb+boot+ command. The inital lines of the console output were
%
\begin{verbatim}
Zynq> boot
boot Petalinux
reading image.ub
43267036 bytes read in 14650 ms (2.8 MiB/s)
## Loading kernel from FIT Image at 10000000 ...
   Using 'conf@1' configuration
   Trying 'kernel@0' kernel subimage
     Description:  Linux Kernel
     Type:         Kernel Image
     Compression:  uncompressed
     Data Start:   0x100000d4
     Data Size:    3925272 Bytes = 3.7 MiB
     Architecture: ARM
     OS:           Linux
     Load Address: 0x00008000
     Entry Point:  0x00008000
     Hash algo:    sha1
     Hash value:   f2ac69bcdcf755b54af38b5fd870f843fb0e9bc7
   Verifying Hash Integrity ... sha1+ OK
## Loading ramdisk from FIT Image at 10000000 ...
   Using 'conf@1' configuration
   Trying 'ramdisk@0' ramdisk subimage
     Description:  ramdisk
     Type:         RAMDisk Image
     Compression:  uncompressed
     Data Start:   0x103c28d8
     Data Size:    39323409 Bytes = 37.5 MiB
     Architecture: ARM
     OS:           Linux
     Load Address: unavailable
     Entry Point:  unavailable
     Hash algo:    sha1
     Hash value:   d57b450521fd302afdc95bfe2e7c521b08745d10
   Verifying Hash Integrity ... sha1+ OK
## Loading fdt from FIT Image at 10000000 ...
   Using 'conf@1' configuration
   Trying 'fdt@0' fdt subimage
     Description:  Flattened Device Tree blob
     Type:         Flat Device Tree
     Compression:  uncompressed
     Data Start:   0x103be6e0
     Data Size:    16712 Bytes = 16.3 KiB
     Architecture: ARM
     Hash algo:    sha1
     Hash value:   bd55217bdcb35db08dcfd467cfbbf0dd5e4cb754
\end{verbatim}
%
The hash values for the kernel, RAM disk, and device tree can be compared
to the versions in the restored image to confirm they are the same.
Save a copy of the Linux boot messages, as they maybe useful later when
building PetaLinux from scratch.

\newpage
Login at the Linux prompt as user \verb+root+ with password \verb+root+
and type a few standard Linux commands to determine some detail about
the Linux and hardware configuration;
%
\begin{verbatim}
root@MiniZed:~# uname -a
Linux MiniZed 4.9.0-xilinx-v2017.4 #1 SMP PREEMPT Thu Mar 22 22:22:16 MST 2018
armv7l GNU/Linux

root@MiniZed:~# mount
rootfs on / type rootfs (rw,size=227296k,nr_inodes=56824)
proc on /proc type proc (rw,relatime)
sysfs on /sys type sysfs (rw,relatime)
devtmpfs on /dev type devtmpfs (rw,relatime,size=227296k,nr_inodes=56824,mode=755)
tmpfs on /run type tmpfs (rw,nosuid,nodev,mode=755)
tmpfs on /var/volatile type tmpfs (rw,relatime)
/dev/mmcblk1p1 on /run/media/mmcblk1p1 type vfat (rw,relatime,gid=6,fmask=0007,
dmask=0007,allow_utime=0020,codepage=437,iocharset=iso8859-1,shortname=mixed,
errors=remount-ro)
devpts on /dev/pts type devpts (rw,relatime,gid=5,mode=620,ptmxmode=000)
/dev/mmcblk1p1 on /mnt/emmc type vfat (rw,relatime,gid=6,fmask=0007,dmask=0007,
allow_utime=0020,codepage=437,iocharset=iso8859-1,shortname=mixed,errors=remount-ro)

root@MiniZed:~# df
Filesystem           1K-blocks      Used Available Use% Mounted on
devtmpfs                227296         4    227292   0% /dev
tmpfs                   255204        84    255120   0% /run
tmpfs                   255204        44    255160   0% /var/volatile
/dev/mmcblk1p1          123089     58385     64704  47% /run/media/mmcblk1p1
/dev/mmcblk1p1          123089     58385     64704  47% /mnt/emmc

root@MiniZed:~# cat /proc/cpuinfo
processor       : 0
model name      : ARMv7 Processor rev 0 (v7l)
BogoMIPS        : 666.66
Features        : half thumb fastmult vfp edsp neon vfpv3 tls vfpd32
CPU implementer : 0x41
CPU architecture: 7
CPU variant     : 0x3
CPU part        : 0xc09
CPU revision    : 0

Hardware        : Xilinx Zynq Platform
Revision        : 0003
Serial          : 0000000000000000

root@MiniZed:~# cat /proc/version
Linux version 4.9.0-xilinx-v2017.4 (sroussea@xterra1) (gcc version 6.2.1 20161016
(Linaro GCC 6.2-2016.11) ) #1 SMP PREEMPT Thu Mar 22 22:22:16 MST 2018

root@MiniZed:~# poweroff
\end{verbatim}
%
Per the recommendation on page 30 of the \emph{Getting Started Guide}~\cite{Avnet_MiniZed_GSG_2018},
use the \verb+poweroff+ command before powering off the MiniZed to avoid corruption of the eMMC flash.

\newpage
The QSPI flash also contains a fallback Linux image. This image can be used when the eMMC flash has
been erased. The fallback image can be booted from U-Boot as follows:
%
\begin{verbatim}
Zynq> run boot_qspi
Booting backup kernel from QSPI..
SF: Detected n25q128 with page size 256 Bytes, erase size 64 KiB, total 16 MiB
device 0 offset 0x270000, size 0xd80000
SF: 14155776 bytes @ 0x270000 Read: OK
## Loading kernel from FIT Image at 10000000 ...
   Using 'conf@1' configuration
   Trying 'kernel@0' kernel subimage
     Description:  Linux Kernel
     Type:         Kernel Image
     Compression:  uncompressed
     Data Start:   0x100000d4
     Data Size:    3925320 Bytes = 3.7 MiB
     Architecture: ARM
     OS:           Linux
     Load Address: 0x00008000
     Entry Point:  0x00008000
     Hash algo:    sha1
     Hash value:   8d758a5b6f7da43eecf5324a02ff10a78e139113
   Verifying Hash Integrity ... sha1+ OK
## Loading ramdisk from FIT Image at 10000000 ...
   Using 'conf@1' configuration
   Trying 'ramdisk@0' ramdisk subimage
     Description:  ramdisk
     Type:         RAMDisk Image
     Compression:  uncompressed
     Data Start:   0x103c2908
     Data Size:    10018537 Bytes = 9.6 MiB
     Architecture: ARM
     OS:           Linux
     Load Address: unavailable
     Entry Point:  unavailable
     Hash algo:    sha1
     Hash value:   2f1d8fc5a9ce9dfbdf1ea753a23dcf7f0d881d55
   Verifying Hash Integrity ... sha1+ OK
## Loading fdt from FIT Image at 10000000 ...
   Using 'conf@1' configuration
   Trying 'fdt@0' fdt subimage
     Description:  Flattened Device Tree blob
     Type:         Flat Device Tree
     Compression:  uncompressed
     Data Start:   0x103be710
     Data Size:    16712 Bytes = 16.3 KiB
     Architecture: ARM
     Hash algo:    sha1
     Hash value:   bd55217bdcb35db08dcfd467cfbbf0dd5e4cb754
\end{verbatim}
%
The hash values of the kernel, RAM disk, and device tree are different than
the versions loaded from the eMMC.

\newpage
% -----------------------------------------------------------------------------
\subsection{Restore Image Details}
% -----------------------------------------------------------------------------
\label{sec:restore_image_details}

The MiniZed contains three non-volatile memory devices;
%
\begin{itemize}
\item 2kbit EEPROM
\item 128Mbit QSPI Flash
\item 8GB eMMC Flash
\end{itemize}
%
The 2kbit EEPROM contains the configuration for the FTDI FT2232H device.
The Avnet restore procedure does not include restoring the EEPROM contents.
The EEPROM contents can be read and saved using the
\href{https://www.ftdichip.com/Support/Utilities.htm#FT_PROG}{FT\_PROG}
tool from FTDI. All MiniZed boards ship with the same
USB JTAG/UART serial number: \verb+1234-oj1+. Connecting multiple MiniZed
boards to the same development machine at the same time requires the use
of unique USB serial numbers, as under Windows, each USB serial number defines
a unique COM port number. The FTDI FT\_PROG tool can be used to change the
MiniZed FTDI FT2232H serial number. Do not change any other FT2232H parameters,
otherwise the Digilent JTAG device will not be detected by the Xilinx tools
as a JTAG programmer.

The Avnet MiniZed documentation page has a link to a zip file containing
the restore images~\cite{Avnet_MiniZed_Restore_2018}. The zip file contains
the following files;
%
\begin{verbatim}
Restoring MiniZed to Factory Status_03
+-- Programming files
|   +-- Micron QSPI Flash
|       +-- flash_fallback_7007S.bin
|       +-- flash_only_boot_7007S.bin
|       +-- zynq_fsbl.elf
+-- Restoring_MiniZed_to_Factory_Status_03.pdf
+-- USB memory stick files
    +-- image.ub
    +-- smallboot.bin
    +-- wpa_supplicant.conf
\end{verbatim}
%
The MiniZed factory QSPI image is \verb+flash_fallback_7007S.bin+,
while the eMMC filesystem contains the files in \verb+USB memory stick files+.

The images programmed in the MiniZed board QSPI flash and eMMC could be
compared to the factory images using byte-by-byte comparisons, however,
there is a simpler (and faster) solution: calculate and compare checksums.
The MiniZed U-Boot image supports the \verb+crc32+ checksum (the U-Boot
source provides support for additional checksum algorithms), and the MiniZed Linux image
supports \verb+sha1sum+, \verb+sha256sum+, and \verb+sha512sum+.
Table~\ref{tab:factory_image_checksums} contains the \verb+crc32+
and \verb+sha1sum+ checksums for the restore images.

% -----------------------------------------------------------------------------
% Factory Image Checksums
% -----------------------------------------------------------------------------
%
\begin{landscape}
\begin{table}
\caption{MiniZed Factory Restore Image Checksums.}
\label{tab:factory_image_checksums}
\begin{center}
\begin{tabular}{|l|c|c|c|}
\hline
      & Image Byte & \multicolumn{2}{c|}{Checksum}\\
\cline{3-4}
Image & Length & \verb+crc32+ & \verb+sha1+\\
\hline
\multicolumn{4}{|l|}{\bf U-Boot}\\
\hline
&&&\\
\verb+flash_fallback_7007S.bin+  & \verb+0x00FC0BE4+ & \verb+0x005B7B3C+ & \verb+0x71511941C14604A88BDD57B0C7762444D1538B5B+\\
\verb+flash_only_boot_7007S.bin+ & \verb+0x00FDA5F4+ & \verb+0x86771B07+ & \verb+0x191BE50A37CF5E7D453A67AE510264C1F2CDF89B+\\
\verb+smallboot.bin+             & \verb+0x00FC0BE4+ & \verb+0x005B7B3C+ & \verb+0x71511941C14604A88BDD57B0C7762444D1538B5B+\\
&&&\\
\hline
\multicolumn{4}{|l|}{\bf Linux}\\
\hline
&&&\\
\verb+image.ub+ & \verb+0x00FC0BE4+ & \verb+0x005B7B3C+ & \verb+0xD618217182184862C69AB1559EF806665B7D0E83+\\
&&&\\
\hline
\end{tabular}
\end{center}
\end{table}
\end{landscape}

% -----------------------------------------------------------------------------
\subsubsection{QSPI Image Details}
% -----------------------------------------------------------------------------

The image in MiniZed QSPI flash can be viewed by copying the
contents of the 16MB flash to DDR memory, and then reading the
image from DDR memory. The U-Boot commands to detect the QSPI flash
and copy 16MB (0x1000000) to DDR address 0x10000000 are
%
\begin{verbatim}
Zynq> sf probe
SF: Detected n25q128 with page size 256 Bytes, erase size 64 KiB, total 16 MiB
Zynq> sf read 10000000 0 1000000
device 0 whole chip
SF: 16777216 bytes @ 0x0 Read: OK
\end{verbatim}
%
The QSPI image header can be displayed by reading the DDR memory
%
\begin{verbatim}
Zynq> md 10000000 30
10000000: eafffffe eafffffe eafffffe eafffffe    ................
10000010: eafffffe eafffffe eafffffe eafffffe    ................
10000020: aa995566 584c4e58 00000000 01010000    fU..XNLX........
10000030: 00001700 00018008 00000000 00000000    ................
10000040: 00018008 00000001 fc164530 00000000    ........0E......
10000050: 00000000 00000000 00000000 00000000    ................
10000060: 00000000 00000000 00000000 00000000    ................
10000070: 00000000 00000000 00000000 00000000    ................
10000080: 00000000 00000000 00000000 00000000    ................
10000090: 00000000 00000000 000008c0 00000c80    ................
100000a0: ffffffff 00000000 ffffffff 00000000    ................
100000b0: ffffffff 00000000 ffffffff 00000000    ................
\end{verbatim}
%
The BootROM Image Header format is defined in Table 6-5 of the Zynq
TRM (p171~\cite{Xilinx_UG585_2018}).

The U-Boot crc32 checksum for the QSPI image copied to DDR
memory is
%
\begin{verbatim}
Zynq> crc32 10000000 fc0be4
crc32 for 10000000 ... 10fc0be3 ==> 005b7b3c
\end{verbatim}
%
The checksum matches that for
\verb+flash_fallback_7007S.bin+
in Table~\ref{tab:factory_image_checksums}.

\clearpage
% -----------------------------------------------------------------------------
\subsubsection{eMMC Image Details}
% -----------------------------------------------------------------------------

The eMMC images can be viewed from the MiniZed Linux console using:
%
\begin{verbatim}
root@MiniZed:~# ls -al /mnt/emmc/
total 58385
drwxrwx---    2 root     disk           512 Jan  1  1970 .
drwxr-xr-x    4 root     root            80 Mar 23 06:07 ..
-rwxrwx---    1 root     disk      43267036 Jul 28  2017 image.ub
-rwxrwx---    1 root     disk      16518116 Jul 28  2017 smallboot.bin
-rwxrwx---    1 root     disk           177 Jul 28  2017 wpa_supplicant.conf

root@MiniZed:~# od -Ax -tx4 -v -N 128 /mnt/emmc/image.ub
000000 edfe0dd0 dc339402 38000000 f4319402
000010 28000000 11000000 10000000 00000000
000020 74000000 bc319402 00000000 00000000
000030 00000000 00000000 01000000 00000000
000040 03000000 04000000 64000000 3c9ab45a
000050 03000000 24000000 00000000 6f422d55
000060 6620746f 6d497469 20656761 20726f66
000070 786e6c70 6d72615f 72656b20 006c656e

root@MiniZed:~# od -Ax -tx4 -v -N 192 /mnt/emmc/smallboot.bin
000000 eafffffe eafffffe eafffffe eafffffe
000010 eafffffe eafffffe eafffffe eafffffe
000020 aa995566 584c4e58 00000000 01010000
000030 00001700 00018008 00000000 00000000
000040 00018008 00000001 fc164530 00000000
000050 00000000 00000000 00000000 00000000
000060 00000000 00000000 00000000 00000000
000070 00000000 00000000 00000000 00000000
000080 00000000 00000000 00000000 00000000
000090 00000000 00000000 000008c0 00000c80
0000a0 ffffffff 00000000 ffffffff 00000000
0000b0 ffffffff 00000000 ffffffff 00000000
\end{verbatim}
%
The image checksums can be calculated via
%
\begin{verbatim}
root@MiniZed:~# sha1sum /mnt/emmc/image.ub
d618217182184862c69ab1559ef806665b7d0e83  /mnt/emmc/image.ub

root@MiniZed:~# sha1sum /mnt/emmc/smallboot.bin
71511941c14604a88bdd57b0c7762444d1538b5b  /mnt/emmc/smallboot.bin
\end{verbatim}
%
The checksums match those in Table~\ref{tab:factory_image_checksums}.
The \verb+wpa_supplicant.conf+ file on the MiniZed was different than
the restore image, however, the user has to edit that file to match
their setup, so that difference was expected.

\clearpage
% -----------------------------------------------------------------------------
\subsubsection{U-Boot Image Versions and Checksums}
% -----------------------------------------------------------------------------
\label{sec:uboot_versions}

This section contains the console output for U-Boot power-on-reset,
\verb+version+, and \verb+crc32+ commands for the three Avnet restore
images.

\subsubsection*{Image: \textcolor{blue}{\texttt{flash\_fallback\_7007S.bin}}}
%
\begin{verbatim}
U-Boot 2017.01 (Mar 22 2018 - 23:33:56 -0700)

Board: Xilinx Zynq
DRAM:  ECC disabled 512 MiB
MMC:   Card did not respond to voltage select!
sdhci@e0100000 - probe failed: -95
sdhci_transfer_data: Error detected in status(0x208000)!
Card did not respond to voltage select!

SF: Detected n25q128 with page size 256 Bytes, erase size 64 KiB, total 16 MiB
*** Warning - bad CRC, using default environment

In:    serial
Out:   serial
Err:   serial
U-BOOT for MiniZed

Hit any key to stop autoboot:  0
Zynq> version

U-Boot 2017.01 (Mar 22 2018 - 23:33:56 -0700)
arm-xilinx-linux-gnueabi-gcc (Linaro GCC 6.2-2016.11) 6.2.1 20161016
GNU ld (GNU Binutils) 2.27.0.20160806

Zynq> sf probe
SF: Detected n25q128 with page size 256 Bytes, erase size 64 KiB, total 16 MiB
Zynq> sf read 10000000 0 1000000
device 0 whole chip
SF: 16777216 bytes @ 0x0 Read: OK
Zynq> crc32 10000000 fc0be4
crc32 for 10000000 ... 10fc0be3 ==> 005b7b3c
\end{verbatim}

\subsubsection*{Image: \textcolor{blue}{\texttt{flash\_only\_boot\_7007S.bin}}}
%
\begin{verbatim}
U-Boot 2016.07 (Jul 27 2017 - 21:56:59 -0700)

DRAM:  ECC disabled 512 MiB
MMC:   sdhci@e0100000: 0, sdhci@e0101000: 1
SF: Detected N25Q128 with page size 256 Bytes, erase size 64 KiB, total 16 MiB
*** Warning - bad CRC, using default environment

In:    serial
Out:   serial
Err:   serial
U-BOOT for

Hit any key to stop autoboot:  0
Zynq> version

U-Boot 2016.07 (Jul 27 2017 - 21:56:59 -0700)
arm-linux-gnueabihf-gcc (Linaro GCC 5.2-2015.11-2) 5.2.1 20151005
GNU ld (Linaro_Binutils-) 2.25.0 Linaro 2016_02
Zynq> sf probe
SF: Detected N25Q128 with page size 256 Bytes, erase size 64 KiB, total 16 MiB
Zynq> sf read 10000000 0 1000000
device 0 whole chip
SF: 16777216 bytes @ 0x0 Read: OK
Zynq> crc32 10000000 fda5f4
crc32 for 10000000 ... 10fda5f3 ==> 86771b07
\end{verbatim}

\subsubsection*{Image: \textcolor{blue}{\texttt{smallboot.bin}}}
%
\begin{verbatim}
U-Boot 2017.01 (Mar 22 2018 - 23:33:56 -0700)

Board: Xilinx Zynq
DRAM:  ECC disabled 512 MiB
MMC:   Card did not respond to voltage select!
sdhci@e0100000 - probe failed: -95
sdhci_transfer_data: Error detected in status(0x208000)!
Card did not respond to voltage select!

SF: Detected n25q128 with page size 256 Bytes, erase size 64 KiB, total 16 MiB
*** Warning - bad CRC, using default environment

In:    serial
Out:   serial
Err:   serial
U-BOOT for MiniZed

Hit any key to stop autoboot:  0
Zynq> version

U-Boot 2017.01 (Mar 22 2018 - 23:33:56 -0700)
arm-xilinx-linux-gnueabi-gcc (Linaro GCC 6.2-2016.11) 6.2.1 20161016
GNU ld (GNU Binutils) 2.27.0.20160806

Zynq> sf probe
SF: Detected n25q128 with page size 256 Bytes, erase size 64 KiB, total 16 MiB
Zynq> sf read 10000000 0 1000000
device 0 whole chip
SF: 16777216 bytes @ 0x0 Read: OK
Zynq> crc32 10000000 fc0be4
crc32 for 10000000 ... 10fc0be3 ==> 005b7b3c
\end{verbatim}
%
The console messages from \textcolor{blue}{\texttt{smallboot.bin}} are
identical to \textcolor{blue}{\texttt{flash\_fallback\_7007S.bin}}. The
easiest way to determine which image is programmed, is
to calculate the \verb+crc32+ checksum using the image length and
expected values in Table~\ref{tab:factory_image_checksums}.

\clearpage
% -----------------------------------------------------------------------------
\subsection{QSPI Flash Programming}
% -----------------------------------------------------------------------------
\label{sec:qspi_flash_programming}

The Avnet MiniZed factory restore document~\cite{Avnet_MiniZed_Restore_2018}
shows how to program the QSPI flash using the Xilinx Software Command-line
Tool (XSCT) and the Vivado GUI. The restore images were copied to the
location \verb+c:\temp\minized+ prior to the operations in the next section.

% -----------------------------------------------------------------------------
\subsubsection{XSCT \texttt{program\_flash}}
% -----------------------------------------------------------------------------

XSCT can be started under Windows by selecting the tool from
\emph{Xilinx Design Tools}$\rightarrow$\emph{SDK 2019.1}$\rightarrow$\emph{Xilinx
Software Command Line Tool}. Once the tool starts, change directory to
the folder containing the restore images and issue the \verb+program_flash+
command using the Tcl exec command, eg.,
%
\begin{verbatim}
xsct% cd {c:\temp\minized}
xsct% exec program_flash -f flash_only_boot_7007S.bin -fsbl zynq_fsbl.elf
  -flash_type qspi_single
\end{verbatim}
%
Running the \verb+program_flash+ command using the Tcl \verb+exec+ command
supresses the command console output until the flash programming completes.
More insight into the flash programming is gained if the tool is
run directly, so that the console output is seen while the flash is
programmed.

The location of the \verb+program_flash+ tool can be determined using
%
\begin{verbatim}
xsct% exec which program_flash
C:\software\Xilinx\SDK\2019.1\bin\program_flash
\end{verbatim}
%
The MiniZed flash was erased using the U-Boot command
%
\begin{verbatim}
Zynq> sf probe
SF: Detected N25Q128 with page size 256 Bytes, erase size 64 KiB, total 16 MiB
Zynq> sf erase 0 1000000
SF: 16777216 bytes @ 0x0 Erased: OK
\end{verbatim}
%
and the Flash/JTAG switch changed to the JTAG position. The flash programming
tool was then run from a Cygwin console using the full path to the
application via
%
\begin{verbatim}
$ cd c:/temp/minized
$ C:/software/Xilinx/SDK/2019.1/bin/program_flash -f flash_only_boot_7007S.bin
  -fsbl zynq_fsbl.elf -flash_type qspi_single

****** Xilinx Program Flash
****** Program Flash v2019.1 (64-bit)
  **** SW Build 2552052 on Fri May 24 14:49:42 MDT 2019
    ** Copyright 1986-2019 Xilinx, Inc. All Rights Reserved.

Connected to hw_server @ TCP:localhost:3121
Available targets and devices:
Target 0 : jsn-MiniZed V1-1234-oj1A
        Device 0: jsn-MiniZed V1-1234-oj1A-4ba00477-0

Retrieving Flash info...

Initialization done, programming the memory
===== mrd->addr=0xF800025C, data=0x00000000 =====
BOOT_MODE REG = 0x00000000
Downloading FSBL...
Running FSBL...
Finished running FSBL.
===== mrd->addr=0xF8000110, data=0x000FA220 =====
READ: ARM_PLL_CFG (0xF8000110) = 0x000FA220
===== mrd->addr=0xF8000100, data=0x00028008 =====
READ: ARM_PLL_CTRL (0xF8000100) = 0x00028008
===== mrd->addr=0xF8000120, data=0x1F000200 =====
READ: ARM_CLK_CTRL (0xF8000120) = 0x1F000200
===== mrd->addr=0xF8000118, data=0x00113220 =====
READ: IO_PLL_CFG (0xF8000118) = 0x00113220
===== mrd->addr=0xF8000108, data=0x00024008 =====
READ: IO_PLL_CTRL (0xF8000108) = 0x00024008
Info:  Remapping 256KB of on-chip-memory RAM memory to 0xFFFC0000.
===== mrd->addr=0xF8000008, data=0x00000000 =====
===== mwr->addr=0xF8000008, data=0x0000DF0D =====
MASKWRITE: addr=0xF8000008, mask=0x0000FFFF, newData=0x0000DF0D
===== mwr->addr=0xF8000910, data=0x000001FF =====
===== mrd->addr=0xF8000004, data=0x00000000 =====
===== mwr->addr=0xF8000004, data=0x0000767B =====
MASKWRITE: addr=0xF8000004, mask=0x0000FFFF, newData=0x0000767B

U-Boot 2019.01-07026-gae88108-dirty (Mar 22 2019 - 04:38:02 -0600)

Model: Zynq CSE QSPI Board
DRAM:  256 KiB
WARNING: Caches not enabled
In:    dcc
Out:   dcc
Err:   dcc
Zynq> sf probe 0 0 0
Warning: SPI speed fallback to 100 kHz
SF: Detected n25q128 with page size 256 Bytes, erase size 64 KiB, total 16 MiB
Zynq> Sector size = 65536.
f probe 0 0 0
Performing Erase Operation...
sf erase 0 FE0000
SF: 16646144 bytes @ 0x0 Erased: OK
Zynq> Erase Operation successful.
INFO: [Xicom 50-44] Elapsed time = 1 sec.
Performing Program Operation...
0%...sf write FFFC0000 0 20000
device 0 offset 0x0, size 0x20000
SF: 131072 bytes @ 0x0 Written: OK
Zynq> sf write FFFC0000 20000 20000
device 0 offset 0x20000, size 0x20000
SF: 131072 bytes @ 0x20000 Written: OK

... [more sector programming messages] ...

Zynq> sf write FFFC0000 FA0000 20000
device 0 offset 0xfa0000, size 0x20000
SF: 131072 bytes @ 0xfa0000 Written: OK
Zynq> 100%
sf write FFFC0000 FC0000 1A5F4
device 0 offset 0xfc0000, size 0x1a5f4
SF: 108020 bytes @ 0xfc0000 Written: OK
Zynq> Program Operation successful.
INFO: [Xicom 50-44] Elapsed time = 67 sec.

Flash Operation Successful
\end{verbatim}
%
The output from \verb+program_flash+ shows that it downloads the FSBL,
a U-Boot image, and then uses U-Boot commands to program the flash.
Interesting! The U-Boot flash programming occurs much faster than
when using the UART, as the U-Boot image used by
\verb+program_flash+ uses the ARM JTAG Debug Communications Channel
(DCC) to transfer sectors to U-Boot for programming to flash:
the initial U-Boot console message shows the use of \verb+dcc+ for
stdin, stdout and stderr.

% -----------------------------------------------------------------------------
\subsubsection{Vivado Flash Programming}
% -----------------------------------------------------------------------------

The Avnet MiniZed factory restore document~\cite{Avnet_MiniZed_Restore_2018}
shows how to use the \emph{Vivado Hardware Manager} to program the QSPI flash.
When I first followed the procedure using the flash images located
in the folder generated by unzipping the restore images zip file,
Vivado would pop-up an error dialog with the message
\verb+[Labtools 27-3161] Flash Programming Unsuccessful+. The Vivado
console did not provide any insight into what might be wrong.
%
Flash programming works if the restore files are first copied to
\verb+c:/temp/minized+.

The Vivado Tcl console output for programming of the image \verb+flash_fallback_7007S.bin+
was
%
\begin{verbatim}
program_hw_cfgmem -hw_cfgmem [ get_property PROGRAM.HW_CFGMEM
  [lindex [get_hw_devices xc7z007s_1] 0]]

Performing Erase Operation...
Erase Operation successful.
INFO: [Xicom 50-44] Elapsed time = 28 sec.

Performing Program Operation...
Program Operation successful.
INFO: [Xicom 50-44] Elapsed time = 68 sec.

Performing Verify Operation...
INFO: [Xicom 50-44] Elapsed time = 91 sec.
Verify Operation successful.

INFO: [Labtoolstcl 44-377] Flash programming completed successfully
\end{verbatim}
%
After programming, the Flash/JTAG boot mode switch was switched to Flash
mode, U-Boot booted, and the procedure in Section~\ref{sec:uboot_versions}
used to confirm the U-Boot version and checksum.

\clearpage
% -----------------------------------------------------------------------------
\subsection{Restore Procedure}
% -----------------------------------------------------------------------------
\label{sec:restore_procedure}

The Avnet MiniZed factory restore procedure is~\cite{Avnet_MiniZed_Restore_2018}:
%
\begin{enumerate}
\item Power-on the MiniZed and program the QSPI flash with the flash-only image.
\item Power-on the MiniZed with the additional USB power source connected and
a USB stick containing the files to program to the eMMC, boot from the flash-only
image, and run the scripts that program the eMMC.
\item Program the QSPI flash with the fallback image.
\end{enumerate}
%
To ensure the restore procedure works consistently, the MiniZed eMMC and
QSPI flash can first be erased. The following steps were performed on a
MiniZed board programmed with the QSPI U-Boot image
\verb+flash_fallback_7007S.bin+.
%
\begin{enumerate}
% -----------------------------------------------------------------------------
\item \textbf{Erase eMMC Flash}
% -----------------------------------------------------------------------------

The fallback Linux image was booted from U-Boot using the command
\verb+run boot_qspi+. The eMMC partitions are then erased using
the \verb+fdisk+ utility via
%
\begin{verbatim}
root@MiniZed:~# fdisk /dev/mmcblk1

The number of cylinders for this disk is set to 232448.
There is nothing wrong with that, but this is larger than 1024,
and could in certain setups cause problems with:
1) software that runs at boot time (e.g., old versions of LILO)
2) booting and partitioning software from other OSs
   (e.g., DOS FDISK, OS/2 FDISK)

Command (m for help): p

Disk /dev/mmcblk1: 7616 MB, 7616856064 bytes
4 heads, 16 sectors/track, 232448 cylinders
Units = cylinders of 64 * 512 = 32768 bytes

        Device Boot      Start         End      Blocks  Id System
/dev/mmcblk1p1               1        3907      125016  83 Linux

Command (m for help): d
Selected partition 1

Command (m for help): p

Disk /dev/mmcblk1: 7616 MB, 7616856064 bytes
4 heads, 16 sectors/track, 232448 cylinders
Units = cylinders of 64 * 512 = 32768 bytes

        Device Boot      Start         End      Blocks  Id System

Command (m for help): w
The partition table has been altered.
Calling ioctl() to re-read partition table
fdisk: WARNING: rereading partition table failed, kernel still uses old table:
Device or resource busy
root@MiniZed:~# reboot
\end{verbatim}
%
In this example console output, the \verb+p+ command was used to print the
partitions (there was only one), and the \verb+d+ command was used to delete
the partition.

After reboot, halt the U-Boot boot sequence, start Linux using
\verb+run boot_qspi+, and use \verb+fdisk+ to confirm there are no
partitions on the eMMC. U-Boot can also be used to view the
empty partition table via
%
\begin{verbatim}
Zynq> mmc dev 1
sdhci_transfer_data: Error detected in status(0x208000)!
switch to partitions #0, OK
mmc1(part 0) is current device
Zynq> mmc part

Partition Map for MMC device 1  --   Partition Type: DOS

Part    Start Sector    Num Sectors     UUID            Type
Zynq>
\end{verbatim}
%

% -----------------------------------------------------------------------------
\item \textbf{Erase QSPI Flash}
% -----------------------------------------------------------------------------

The U-Boot commands to erase QSPI flash are
%
\begin{verbatim}
Zynq> sf probe
SF: Detected n25q128 with page size 256 Bytes, erase size 64 KiB, total 16 MiB
Zynq> sf erase 0 1000000
SF: 16777216 bytes @ 0x0 Erased: OK
\end{verbatim}
%
Power-down the board (remove the USB connections), and change the Flash/JTAG
boot mode switch to JTAG mode.

% -----------------------------------------------------------------------------
\item \textbf{Program QSPI Flash with} \verb+flash_only_boot_7007S.bin+
% -----------------------------------------------------------------------------

Use one of the flash programming procedures outlined in
Section~\ref{sec:qspi_flash_programming} to program the image.
After programming, power-down the board, and change the Flash/JTAG boot
mode switch to Flash mode.

% -----------------------------------------------------------------------------
\item \textbf{Prepare a USB flash drive}
% -----------------------------------------------------------------------------

Format a USB flash drive in FAT or FAT32 format, and copy the restore image
files from within the zip file:
%
\begin{verbatim}
+-- USB memory stick files
    +-- image.ub
    +-- smallboot.bin
    +-- wpa_supplicant.conf
\end{verbatim}
%
onto the USB drive. Edit \verb+wpa_supplicant.conf+ to match your local
WiFi network settings.

% -----------------------------------------------------------------------------
\item \textbf{Power the MiniZed with two USB cables}
% -----------------------------------------------------------------------------

The MiniZed USB host-mode interface is used to access the USB flash drive.
The power for the USB host-mode interface is provided by the second
micro-USB connector identified with the label \emph{Auxiliary Power} in
Figure~\ref{fig:minized_quick_start_diagram}.

Insert the USB flash drive in the MiniZed USB host-mode connector, and
connect both micro-USB connectors to your development host. The flash-only
boot image contains Linux. The Linux version information is
%
\begin{verbatim}
PetaLinux 2016.4 plnx_arm /dev/ttyPS0

plnx_arm login: root
Password:
root@plnx_arm:~# uname -a
Linux plnx_arm 4.6.0-xilinx #1 SMP PREEMPT Thu Jul 27 22:24:51 PDT 2017
armv7l GNU/Linux
\end{verbatim}

% -----------------------------------------------------------------------------
\item \textbf{Restore MiniZed Linux}
% -----------------------------------------------------------------------------

The Linux commands used to restore the MiniZed eMMC Linux installation to factory state are
implemented in the script \verb+onetest.sh+ (p16~\cite{Avnet_MiniZed_Restore_2018}).
The location and contents of the script can be viewed via
%
\begin{verbatim}
root@plnx_arm:~# which onetest.sh
/usr/local/bin/onetest.sh
root@plnx_arm:~# cat /usr/local/bin/onetest.sh
\end{verbatim}
%
The script creates a DOS partition on the eMMC, formats the partition to use a FAT32
filesystem, mounts the eMMC filesystem, mounts the USB flash stick, copies the files
from the USB stick to eMMC, programs the QSPI flash with \verb+smallboot.bin+,
loads the WiFi driver, brings up the WiFi interface, runs the I2C and LED
test application, and then shuts down the system.

Run the script. Programming the QSPI flash via the \verb+flashcp+ command takes
the most time. The I2C and LED test application will run until you press a key,
and then the board will shutdown.

Press the MiniZed reset button, interrupt the U-Boot boot process, and confirm
the U-Boot checksum is that of \verb+smallboot.bin+.

% -----------------------------------------------------------------------------
\item \textbf{Program QSPI Flash with} \verb+flash_fallback_7007S.bin+
% -----------------------------------------------------------------------------

The Avnet restore procedure lists this as an optional step
(p21~\cite{Avnet_MiniZed_Restore_2018}).
Use one of the flash programming procedures outlined in
Section~\ref{sec:qspi_flash_programming} to program the fallback image.

% -----------------------------------------------------------------------------
\item \textbf{MiniZed Tests}
% -----------------------------------------------------------------------------

The Avnet restore procedure has instructions for testing the MiniZed
%
\begin{itemize}
\item Bluetooth (p18~\cite{Avnet_MiniZed_Restore_2018})
\item The Power Management controller (p19~\cite{Avnet_MiniZed_Restore_2018})
\item Motion sensor and GPIO LEDs (p20~\cite{Avnet_MiniZed_Restore_2018})
\end{itemize}

\newpage
% -----------------------------------------------------------------------------
\item \textbf{Save the U-Boot Environment to QSPI Flash}
% -----------------------------------------------------------------------------

If the flash sector containing the environment is blank, or the CRC
is incorrect, the U-Boot power-on messages contain the warning
%
\begin{verbatim}
*** Warning - bad CRC, using default environment
\end{verbatim}
%
This warning can be eliminated by saving the U-Boot environment.
%
\begin{verbatim}
Zynq> saveenv
Saving Environment to SPI Flash...
SF: Detected n25q128 with page size 256 Bytes, erase size 64 KiB, total 16 MiB
Erasing SPI flash...Writing to SPI flash...done
\end{verbatim}
%
The environment sector does not need to be erased, as U-Boot does that
before writing to the SPI flash. The environment sector can be erased via
%
\begin{verbatim}
Zynq> sf erase ff0000 10000
SF: 65536 bytes @ 0xff0000 Erased: OK
\end{verbatim}
%
After erasing the environment section, press the reset button, halt the U-Boot boot
sequence, and the CRC error message will be present in the U-Boot console
messages.
\end{enumerate}

\clearpage
% =============================================================================
\section{FTDI EEPROM Programming}
% =============================================================================
\label{sec:ftdi_eeprom}

The MiniZed uses the \href{https://ftdichip.com}{FTDI} FT2232H USB interface to 
implement USB-to-JTAG and USB-to-UART interfaces. The USB device strings are
stored on a 2kbit EEPROM (93LC56B) connected to the FT2232H device. The 
FTDI \href{https://ftdichip.com/utilities}{FT\_PROG EEPROM Programming 
Utility} can be used to \emph{read} the EEPROM contents, however, this
application should not be used to write to the EEPROM, as it does not preserve
the Xilinx/Avnet/Digilent vendor-specific device identification bytes added to the 
user area of the EEPROM. Figure~\ref{fig:minized_ft_prog} shows the FT\_PROG
view for two different MiniZed boards; the 16-bit words in user area 
of the EEPROM can be seen in the bottom of the figure. 

The FTDI library 
\href{https://ftdichip.com/drivers/d2xx-drivers}{D2XX}~\cite{FTDI_D2XX_2019}
provides functions to implement USB data transfers, and to read and write
the EEPROM.
%
The open-source \href{https://www.intra2net.com/en/developer/libftdi}{libFTDI}
provides similar functionality.
%
% libftdi/doc/EEPROM-structure contains notes on the EEPROM  structure
%
The vendor-specific device identification bytes in the FTDI
EEPROM are not officially documented, however, some details have been
reverse-engineered, eg., see~\cite{Bartik_2019}.
%
Reverse-engineering the EEPROM format was made unnecessary by Digilent 
customer support inadvertantly posting their FTDI configuration tool 
\href{https://forum.digilentinc.com/topic/1816-digilent-smt1-recovery}
{FTDIConfig.exe} on their support forum.
Figure~\ref{fig:digilent_ftdi_config} shows the Digilent FTDI configuration
view for two different MiniZed boards.

The \texttt{MiniZed1} EEPROM was updated using a custom application based
on the FTDI D2XX library that;
%
read the EEPROM using \texttt{FT\_ReadEE},
replaced the default serial string, \texttt{1234-oj1}, 
with the identical length string, \texttt{MiniZed1},
updated the checksum,
and wrote the EEPROM using \texttt{FT\_WriteEE}. 
%
Vivado correctly recognized the device as \texttt{MiniZed1}, and Windows 
10 created a new COM port specific to this USB device. The same application 
was then used to reset the EEPROM to the original serial string; Vivado 
recognized the device as \texttt{1234-oj1} and TeraTerm found the device 
on the previously assigned COM port. 

The Digilent FTDI configuration tool was then used to program the MiniZed 
with the new serial string \texttt{MiniZed1}. 
%
Figure~\ref{fig:minized1_eeprom} compares the \texttt{MiniZed1} EEPROM image 
derived from the \texttt{1234-oj1} image, with the image created by the 
Digilent configuration tool. There are three differences; the 16-bit word 
at offset 0x000D  changes from \textcolor{blue}{0x0003} to 
\textcolor{red}{0x0001}, the 16-bit word at offset 0x006A changes from 
\textcolor{blue}{0x0001} to \textcolor{red}{0x0000}, and the final 
16-bit words (the checksums) change.
%
The change at offset 0x000D may be the version 3 value for FT2232H extensions 
mentioned in FTDI AN428.
%
The FTDI FT\_PROG and Digilent FTDI configuration tools were then used to 
program the MiniZed with the \emph{longer} serial string \texttt{MiniZed01}; 
the string offsets in the FTDI header bytes changed to accommodate the 
longer serial number string. 
%
Based on these tests, the application \texttt{minized\_ftdi\_serial} was 
created to read or write the MiniZed serial string.

% -----------------------------------------------------------------------------
% MiniZed EEPROM images
% -----------------------------------------------------------------------------
%
\begin{figure}[t]
\begin{minipage}{0.5\textwidth}
\begin{center}
\footnotesize
{\tt
0000: 0801 0403 6010 0700 3280 0008 0000 0E9A
0008: 18A8 12C0 0000 0000 0056 \textcolor{blue}{0003} 584A 7641
0010: 656E 0074 694D 696E 655A 2064 3156 0000
0018: 0000 0000 0000 0000 0000 0000 0000 0000
0020: 0000 0000 0000 0000 0000 0000 0000 0000
0028: 0000 0000 0000 0000 0000 0000 0000 0000
0030: 0000 0000 0000 0000 0000 0000 0000 0000
0038: 0000 0000 0000 0000 0000 0000 0000 0000
0040: 0000 0000 0000 0000 0000 0000 0000 0000
0048: 0000 0000 0000 0000 0000 030E 0058 0069
0050: 006C 0069 006E 0078 0318 004A 0054 0041
0058: 0047 002B 0053 0065 0072 0069 0061 006C
0060: 0312 004D 0069 006E 0069 005A 0065 0064
0068: 0031 0302 \textcolor{blue}{0001} 0000 0000 0000 0000 0000
0070: 0000 0000 0000 0000 0000 0000 0000 0000
0078: 0000 0000 0000 0000 0000 0000 0000 \textcolor{blue}{EC87}}
\vskip3mm
(a) Based on original MiniZed image
\end{center}
\end{minipage}
\hfil
\begin{minipage}{0.5\textwidth}
\begin{center}
\footnotesize
{\tt
0000: 0801 0403 6010 0700 3280 0008 0000 0E9A
0008: 18A8 12C0 0000 0000 0056 \textcolor{red}{0001} 584A 7641
0010: 656E 0074 694D 696E 655A 2064 3156 0000
0018: 0000 0000 0000 0000 0000 0000 0000 0000
0020: 0000 0000 0000 0000 0000 0000 0000 0000
0028: 0000 0000 0000 0000 0000 0000 0000 0000
0030: 0000 0000 0000 0000 0000 0000 0000 0000
0038: 0000 0000 0000 0000 0000 0000 0000 0000
0040: 0000 0000 0000 0000 0000 0000 0000 0000
0048: 0000 0000 0000 0000 0000 030E 0058 0069
0050: 006C 0069 006E 0078 0318 004A 0054 0041
0058: 0047 002B 0053 0065 0072 0069 0061 006C
0060: 0312 004D 0069 006E 0069 005A 0065 0064
0068: 0031 0302 \textcolor{red}{0000} 0000 0000 0000 0000 0000
0070: 0000 0000 0000 0000 0000 0000 0000 0000
0078: 0000 0000 0000 0000 0000 0000 0000 \textcolor{red}{ECAF}}
\vskip3mm
(b) Digilent FTDI programmer
\end{center}
\end{minipage}
\caption{FTDI EEPROM images for \texttt{MiniZed1}.}
\label{fig:minized1_eeprom}
\end{figure}
% -----------------------------------------------------------------------------

\clearpage
% -----------------------------------------------------------------------------
% FTDI FT_PROG
% -----------------------------------------------------------------------------
%
\begin{figure}[p]
  \begin{minipage}{0.5\textwidth}
    \begin{center}
    \includegraphics[width=0.95\textwidth]
    {figures/minized_ft_prog_1234-oj1.png}\\
    (a) Serial number: \texttt{1234-oj1}
    \end{center}
  \end{minipage}
  \hfil
  \begin{minipage}{0.5\textwidth}
    \begin{center}
    \includegraphics[width=0.95\textwidth]
    {figures/minized_ft_prog_MiniZed1.png}\\
    (b) Serial number: \texttt{MiniZed1}
    \end{center}
  \end{minipage}
  \caption{FTDI FT\_PROG view of two MiniZed boards.}
  \label{fig:minized_ft_prog}
\end{figure}
% -----------------------------------------------------------------------------

% -----------------------------------------------------------------------------
% Digilent FTDI Config
% -----------------------------------------------------------------------------
%
\begin{figure}[p]
  \begin{minipage}{0.5\textwidth}
    \begin{center}
    \includegraphics[width=0.95\textwidth]
    {figures/minized_digilent_ftdi_config_1234-oj1.png}\\
    (a) Serial number: \texttt{1234-oj1}
    \end{center}
  \end{minipage}
  \hfil
  \begin{minipage}{0.5\textwidth}
    \begin{center}
    \includegraphics[width=0.95\textwidth]
    {figures/minized_digilent_ftdi_config_MiniZed1.png}\\
    (b) Serial number: \texttt{MiniZed1}
    \end{center}
  \end{minipage}
  \caption{Digilent FTDI configuration view of two MiniZed boards.}
  \label{fig:digilent_ftdi_config}
\end{figure}
% -----------------------------------------------------------------------------

\clearpage
% =============================================================================
\section{Unique Board Identification}
% =============================================================================
\label{sec:unique_id}

% -----------------------------------------------------------------------------
% MiniZed Device DNA
% -----------------------------------------------------------------------------
%
\begin{figure}[t]
  \begin{minipage}{0.5\textwidth}
    \begin{center}
    \includegraphics[width=0.7\textwidth]
    {figures/minized_device_dna_MiniZed01.png}\\
    (a) \texttt{MiniZed01}
    \end{center}
  \end{minipage}
  \hfil
  \begin{minipage}{0.5\textwidth}
    \begin{center}
    \includegraphics[width=0.7\textwidth]
    {figures/minized_device_dna_MiniZed02.png}\\
    (b) \texttt{Minized02}
    \end{center}
  \end{minipage}
  \caption{MiniZed device DNA.}
  \label{fig:minized_device_dna}
\end{figure}
% -----------------------------------------------------------------------------

MiniZed boards all ship with the same FTDI EEPROM image, so they are
not uniquely identifiable based on their USB identification information
(at least until the FTDI EEPROM is reprogrammed). The Xilinx programmable
logic contains a unique ID referred to as the device DNA, and the Murata
1DX WiFi chip contains a unique MAC address. 
%
The device DNA can be read using the Vivado hardware manager.
Figure~\ref{fig:minized_device_dna} shows the device DNA read from
\texttt{MiniZed01} and \texttt{MiniZed02}.
%
The WiFi MAC address can be read using PetaLinux by first running the WiFi 
script, \texttt{wifi.sh}, then \texttt{ifconfig wlan0}, and the MAC address
is reported as \texttt{HWaddr}. The Bluetooth script, \texttt{bt.sh},
reports the MAC address as \texttt{BDADDR}.
The MAC address is stored in the one-time programmable (OTP) memory
of the Infineon/Cypress/Broadcom BCM4343W chipset on the Murata 1DX
WiFi/Bluetooth module.
%
Table~\ref{tab:minized_unique_id} shows some example board identifiers.
The \href{https://www.wireshark.org/tools/oui-lookup.html}{Wireshark
OUI Lookup Tool} indicates that MAC prefixes \texttt{A0:CC:2B} and
\texttt{44:91:60} both belong to \texttt{Murata Manufacturing Co., Ltd.}

% -----------------------------------------------------------------------------
% MiniZed unique board identification
% -----------------------------------------------------------------------------
%
\begin{table}[b]
\caption{MiniZed unique board identification}
\label{tab:minized_unique_id}
\begin{center}
\begin{tabular}{|l|c|c|}
\hline
FTDI Name & Device DNA & WiFi/Bluetooth MAC\\
\hline\hline
&&\\
\texttt{MiniZed01} & \texttt{2A11A4C684B90A53} & \texttt{A0:CC:2B:FF:E9:4B}\\
\texttt{MiniZed02} & \texttt{3A1102EA74B20A97} & \texttt{44:91:60:17:59:47}\\
\texttt{MiniZed03} & \texttt{3A1102EA74B0226F} & \texttt{44:91:60:17:D5:CD}\\
&&\\
\hline
\end{tabular}
\end{center}
\end{table}
% -----------------------------------------------------------------------------



\clearpage
% =============================================================================
\section{Resources}
% =============================================================================

% -----------------------------------------------------------------------------
\subsection{Avnet}
% -----------------------------------------------------------------------------
%
\begin{itemize}
\item \href{https://www.avnet.com/wps/portal/us/products/avnet-boards/avnet-board-families/minized}{Avnet MiniZed product page}

The MiniZed documentation links take you to the Element14 web site.
\item \href{https://www.element14.com/community/docs/DOC-95639}{Element14 MiniZed product page}
\begin{itemize}
\item \emph{Technical Documents} has links to the documentation
\item \emph{Reference Designs} has links to the tutorials
\end{itemize}
%
\item \emph{MiniZed Schematic}~\cite{Avnet_MiniZed_Schematic_2017}
\item \emph{MiniZed Quickstart Guide}~\cite{Avnet_MiniZed_QSC_2017}
\item \emph{MiniZed Getting Started Guide}~\cite{Avnet_MiniZed_GSG_2018}
\item \emph{MiniZed Hardware User Guide}~\cite{Avnet_MiniZed_HW_2020}
\item \emph{Restoring MiniZed to the Factory State}~\cite{Avnet_MiniZed_Restore_2018}
\end{itemize}

% -----------------------------------------------------------------------------
\subsection{Xilinx}
% -----------------------------------------------------------------------------
%
\begin{itemize}
\item Xilinx Design Hubs
\begin{itemize}
\item \href{https://www.xilinx.com/support/documentation-navigation/design-hubs.html}
{https://www.xilinx.com/support/documentation-navigation/design-hubs.html}
\end{itemize}
%
The \emph{Zynq-7000 SoC Design} section links to multiple Zynq-7000 design hubs:
\begin{itemize}
\item \href{https://www.xilinx.com/support/documentation-navigation/design-hubs/dh0050-zynq-7000-design-overview-hub.html}{Design Overview}
\item \href{https://www.xilinx.com/support/documentation-navigation/design-hubs/dh0051-zynq-7000-data-movement-hub.html}{Data Movers}
\item \href{https://www.xilinx.com/support/documentation-navigation/design-hubs/dh0052-zynq-7000-power-management-hub.html}{Power Management}
\item \href{https://www.xilinx.com/support/documentation-navigation/design-hubs/dh0053-zynq-7000-boot-and-config-hub.html}{Boot and Configuration}
\item \href{https://www.xilinx.com/support/documentation-navigation/design-hubs/dh0056-zynq-7000-security-hub.html}{Security}
\item \href{https://www.xilinx.com/support/documentation-navigation/design-hubs/dh0054-zynq-7000-performance-and-benchmarks-hub.html}{Performance and Acceleration}
\item \href{https://www.xilinx.com/support/documentation-navigation/design-hubs/dh0055-zynq-7000-debug-hub.html}{Debug}
\end{itemize}
%
\item Zynq-7000 and MPSoC Embedded Design Tutorials
\begin{itemize}
\item \href{https://xilinx.github.io/Embedded-Design-Tutorials/master/docs/index.html}{https://xilinx.github.io/Embedded-Design-Tutorials/master/docs/index.html}
\end{itemize}
%
\item Zynq-7000 Technical Reference Manual~\cite{Xilinx_UG585_2021}
\item Zynq-7000 Software Developers Guide~\cite{Xilinx_UG821_2015}
\end{itemize}


\clearpage

%------------------------------------------------------------------------------
% Do the bibliography
%------------------------------------------------------------------------------
%
% Note, you can't have spaces in the list of bibliography files
%
\bibliography{sections/refs}
\bibliographystyle{plain}

%------------------------------------------------------------------------------
\end{document}

